\documentclass{beamer}

\usepackage[utf8]{inputenc}
\usetheme{Madrid}
\usecolortheme{seahorse}

\title{Fase A: Visi Arsitektur dalam Metode Pengembangan Arsitektur TOGAF}
\author{Alfa Yohannis}
\date{\today}

\begin{document}

\frame{\titlepage}

\begin{frame}
\frametitle{Tujuan}
\begin{itemize}
\item Menetapkan ruang lingkup proyek dan mendapatkan persetujuan dari pemangku kepentingan
\item Mengidentifikasi stakeholder utama dan memahami tujuan mereka
\item Mendefinisikan visi arsitektur tingkat tinggi
\end{itemize}
\end{frame}

\begin{frame}
\frametitle{Input}
\begin{itemize}
\item Pernyataan Kebutuhan Arsitektur
\item Strategi Bisnis Organisasi
\end{itemize}
\end{frame}

\begin{frame}
\frametitle{Langkah-langkah}
\begin{enumerate}
\item Mengidentifikasi stakeholder dan pengguna
\item Mendefinisikan visi dan tujuan proyek
\item Mengembangkan arsitektur visi
\item Mendapatkan persetujuan dari stakeholder
\end{enumerate}
\end{frame}

\begin{frame}
\frametitle{Output}
\begin{itemize}
\item Visi Arsitektur
\item Pernyataan Kebutuhan Arsitektur yang disetujui
\end{itemize}
\end{frame}

\begin{frame}
\frametitle{Contoh}
\begin{itemize}
\item Contoh Visi Arsitektur: Diagram atau narasi yang menjelaskan tentang visi arsitektur masa depan
\end{itemize}
\end{frame}

\begin{frame}
\frametitle{Ringkasan}
Fase Visi Arsitektur merupakan langkah awal dalam Metode Pengembangan Arsitektur TOGAF, dimana kita menetapkan ruang lingkup, mengidentifikasi stakeholder, dan mendefinisikan visi arsitektur tingkat tinggi. Fase ini sangat penting untuk memastikan bahwa proyek arsitektur memiliki dukungan dan arah yang tepat dari awal.
\end{frame}

\end{document}
