\documentclass{beamer}

\usepackage[utf8]{inputenc}
\usetheme{Madrid}
\usecolortheme{seahorse}

\title{Fase C: Arsitektur Data dalam Metode Pengembangan Arsitektur TOGAF}
\author{Alfa Yohannis}
\date{\today}

\begin{document}

\frame{\titlepage}

\begin{frame}
\frametitle{Tujuan}
\begin{itemize}
\item Mendefinisikan struktur data yang diperlukan untuk mendukung bisnis
\item Mengintegrasikan data dari berbagai sumber dalam organisasi
\end{itemize}
\end{frame}

\begin{frame}
\frametitle{Input}
\begin{itemize}
\item Arsitektur Bisnis
\item Pernyataan Kebutuhan Arsitektur
\end{itemize}
\end{frame}

\begin{frame}
\frametitle{Langkah-langkah}
\begin{enumerate}
\item Mengembangkan Baseline Arsitektur Data
\item Mengembangkan Arsitektur Sasaran Data
\item Melakukan Gap Analysis
\item Mendefinisikan peta jalan data
\end{enumerate}
\end{frame}

\begin{frame}
\frametitle{Output}
\begin{itemize}
\item Baseline Arsitektur Data
\item Arsitektur Sasaran Data
\item Peta jalan data
\end{itemize}
\end{frame}

\begin{frame}
\frametitle{Contoh}
\begin{itemize}
\item Contoh Baseline Arsitektur Data: Diagram arsitektur data saat ini
\item Contoh Arsitektur Sasaran Data: Diagram arsitektur data masa depan
\end{itemize}
\end{frame}

\begin{frame}
\frametitle{Ringkasan}
Fase Arsitektur Data memungkinkan kita untuk mendefinisikan dan memvalidasi struktur data yang diperlukan untuk mendukung arsitektur bisnis. Fase ini penting untuk memastikan bahwa data dan integrasinya sesuai dengan kebutuhan bisnis.
\end{frame}

\end{document}
