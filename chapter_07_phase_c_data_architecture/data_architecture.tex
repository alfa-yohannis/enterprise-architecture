\documentclass[aspectratio=169, table]{beamer}

%\usepackage[beamertheme=./praditatheme]{Pradita}
\usepackage[utf8]{inputenc}

\usetheme{Pradita}

\subtitle{MTI102-Information Systems \&\\Technology Architecture}

\title{\Large Phase C: Data Architecture\\in TOGAF
    Architecture\\Development Method (ADM)}
\date[Serial]{\scriptsize {PRU/SPMI/FR-BM-18/0222}}
\author[Pradita]{\small {\textbf{Alfa Yohannis}}}

\begin{document}

    \frame{\titlepage}

    \begin{frame}
        \frametitle{Goals}
        % \framesubtitle{\hspace{1cm}}
        \begin{itemize}
            \item Develop Target Data Architectures that enable the Business Architecture and Architectural Vision to be achieved while addressing Architectural Work Requests and \textit{concerns} from \textit{stakeholders}.
            \item Identify candidate Architecture Roadmap components based on gaps between Baseline and Target Data Architectures
        \end{itemize}
    \end{frame}

    \begin{frame}
        \frametitle{Input (1)}
        % \framesubtitle{\hspace{1cm}}
        \vspace{20pt}
        \begin{enumerate}
            \item Architectural Job Request
            \item Capability Assessment
            \item Communication Plan
            \item Organizational Models for Enterprise Architecture
            \item Customized Architectural Framework
            \item Data principles, if any
            \item Statement of Architectural Work
            \item Draft Architectural Requirements Specification Design:
            \begin{enumerate}
                \item Gap Analysis Results
                \item Other relevant technical requirements
            \end{enumerate}
        \end{enumerate}
    \end{frame}


    \begin{frame}
        \frametitle{Input (2)}
        % \framesubtitle{\hspace{1cm}}
        \vspace{20pt}
        \begin{enumerate}
            \setcounter{enumi}{8}
            \item Architectural Vision
            \item Architectural Repository
            \item Draft Architectural Definition Document:
            \begin{enumerate}
                \item Baseline Business Architecture (detailed)
                \item Target Business Architecture (detailed)
                \item Baseline Data Architecture (outline)
                \item Target Data Architecture (outline)
                \item Baseline Application Architecture (outline)
                \item Target Application Architecture (outline)
                \item Baseline Technology Architecture (outline)
                \item Target Technology Architecture (outline)
            \end{enumerate}
            \item Business Architecture Components of the Architecture Roadmap
        \end{enumerate}
    \end{frame}


    \begin{frame}
        \frametitle{Steps}
        \vspace{20pt}
        \begin{enumerate}
            \item Select a reference model, viewpoint, and tool
            \item Develop Baseline Data Architecture Description
            \item Develop Target Data Architecture Description
            \item Perform a Gap Analysis
            \item Specifies candidate roadmap components
            \item Resolve impact in Architectural Landscape
            \item Conduct a formal stakeholder review
            \item Complete Data Architecture
            \item Create an Architecture Definition Document
        \end{enumerate}
    \end{frame}

    \begin{frame}
        \frametitle{Output}
        % \framesubtitle{\hspace{1cm}}
        \begin{enumerate}
            \item Architectural Statement of Work, updated as necessary
            \item Validated data principles, or new data principles
            \item Draft Architectural Definition Document, with updated content
            \item Design Architectural Requirements Specifications, including those that have been updated
            \item Data Architecture Components of the Architecture Roadmap
        \end{enumerate}
    \end{frame}

    \begin{frame}
        \frametitle{Data Principles}
        % \framesubtitle{\hspace{1cm}}
        \begin{enumerate}
            \item \textbf{Data is an Asset}.
            Data is an asset that has value to a company and is managed accordingly.
            \item \textbf{Data is shared}.
            Users can access the data necessary to carry out their tasks; therefore, data is shared across company functions
            and organizations.
            \item \textbf{Accessible Data}. Data can be accessed by users to carry out their functions.
        \end{enumerate}
    \end{frame}

    \begin{frame}
        \frametitle{Data Principles (2)}
        % \framesubtitle{\hspace{1cm}}
        \begin{enumerate}
            \setcounter{enumi}{3}
            \item \textbf{Vocabulary Definitions and Shared Data}. Data is defined consistently across the enterprise, and the definitions are understandable and available to all users.
            \item \textbf{Data Security}.
            Data is protected from unauthorized use and disclosure. In addition to traditional aspects of national security
            classifications, these include, but are not limited to, pre-decision protection, sensitive, sensitive source selection, and
            proprietary information.
        \end{enumerate}
    \end{frame}


    \begin{frame}
        \frametitle{ Data Architecture Catalogs, Matrices, and Diagrams}
        \framesubtitle{\hspace{1cm}}
        \begin{small}
            \begin{table}[]
                \begin{tabular}{|c|c|c|}
                    \hline
                    \textbf{Catalogs} & \textbf{Matrices} & \textbf{Diagrams} \\ \hline
                    Data Entity/Data & Data Entity/Business Function matrix & Conceptual Data diagram \\
                    Component catalog & Application/Data matrix & Logical Data diagram \\
                    & & Data Dissemination diagram \\
                    & & Data Lifecycle diagram \\
                    & & Data Security diagram \\
                    & & Data Migration diagram \\ \hline
                \end{tabular}
            \end{table}
        \end{small}
    \end{frame}

	\begin{frame}
	\frametitle{Komponen Dokumen Definisi Arsitektur}
	Topik yang harus dibahas dalam Dokumen Definisi Arsitektur yang terkait dengan Arsitektur Data adalah sebagai berikut:
	\begin{itemize}
		\item Arsitektur Data Baseline, jika memungkinkan
		\item Arsitektur Data Target, termasuk model untuk data bisnis, data logis, dan proses manajemen data, serta matriks Entitas Data/Fungsi Bisnis
		\item Pandangan Arsitektur Data yang sesuai dengan sudut pandang yang dipilih, yang mengatasi kekhawatiran pemangku kepentingan kunci
	\end{itemize}
\end{frame}


\begin{frame}
	\frametitle{Komponen Spesifikasi Kebutuhan Arsitektur}
	\label{sec:data_komponen_spesifikasi_kebutuhan}
	Kebutuhan Arsitektur Data yang mengisi Spesifikasi Kebutuhan Arsitektur pada Fase C meliputi:
	\begin{itemize}
		\item Hasil Analisis Kesenjangan
		\item Kebutuhan interoperabilitas data (misalnya, skema XML, kebijakan keamanan)
		\item Area di mana Arsitektur Bisnis mungkin perlu berubah untuk mematuhi perubahan dalam Arsitektur Data
		\item Pembatasan pada Arsitektur Teknologi yang akan dirancang
		\item Kebutuhan bisnis yang telah diperbarui, jika ada
		\item Kebutuhan aplikasi yang telah diperbarui, jika ada
	\end{itemize}
\end{frame}

    \begin{frame}
        \frametitle{Some Concerns Regarding Remote Work}
        \begin{enumerate}
            \item Offline data migration (physical, digital) to online (accessed via the internet)
            \item Transformation of physical and unstructured data to digital and structured
            \item Data security: encryption, password management, access rights, etc
            \item Data governance: authorization and CRUD (create, read, update, delete) data settings, how long data is stored before being deleted
            \item How to store, share, access, process data
            \item backup and accident/disaster recovery
            \item Data costs: financial, speed and load of processing, transfer, and storage
        \end{enumerate}
    \end{frame}





    \begin{frame}
        \frametitle{Data Related Innovation}
        % \framesubtitle{\hspace{1cm}}
        \begin{columns}
            \begin{column}{0.5\textwidth}
                \begin{center}
                    \begin{enumerate}
                        \item Text and Binary Files
                        \item Relational Database Systems (RDBMS)
                        \item Online transaction processing (OLTP) Data
                        \item Online analytical processing (OLAP) Data
                        \item NoSQL and Unstructured Data
                        \item graph database
                        \item file-based database
                    \end{enumerate}
                \end{center}
            \end{column}
            \begin{column}{0.5\textwidth}
                \begin{center}
                    \begin{enumerate}
                        \setcounter{enumi}{7}
                        \item column-based database
                        \item in-memory database
                        \item Sequential vs parallel processing
                        \item Centralized vs distributed data
                        \item Centralized vs federated data
                        \item Centralized vs Decentralized data (Web3 movement) please search Filecoin, SOLID POD
                    \end{enumerate}
                \end{center}
            \end{column}
        \end{columns}
    \end{frame}


    \begin{frame}
        \frametitle{Summary}
        % \framesubtitle{\hspace{1cm}}
        \begin{itemize}
            \item The Data Architecture phase allows us to define and validate the data structures required to support the business architecture.
            \item This phase is important to ensure that the data and its integration meet business needs.
        \end{itemize}
    \end{frame}



\end{document}