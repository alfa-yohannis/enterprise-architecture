\documentclass[aspectratio=169, table]{beamer}

%\usepackage[beamertheme=./praditatheme]{Pradita}
\usepackage[utf8]{inputenc}

\usetheme{Pradita}

\subtitle{MTI102-Information System \&\\Technology Architecture}

\title{Fase C: Arsitektur Data dalam Metode \&\\ Pengembangan Arsitektur TOGAF}
\date[Serial]{\scriptsize {PRU/SPMI/FR-BM-18/0222}}
\author[Pradita]{\small {\textbf{Alfa Yohannis}}}

\begin{document}

	\frame{\titlepage}

	\begin{frame}
		\frametitle{Tujuan}
%		\framesubtitle{\hspace{1cm}}
		\begin{itemize}
			\item Mengembangkan Arsitektur Data Target yang memampukan Arsitektur Bisnis dan Visi Arsitektur tercapai, sambil menangani Permintaan Pekerjaan Arsitektur dan \textit{concerns} dari \textit{stakeholders}.
			\item Identifikasi kandidat komponen Peta Jalan Arsitektur berdasarkan gap antara Baseline dan Target Data Architectures
		\end{itemize}
	\end{frame}

	\begin{frame}
		\frametitle{Input (1)}
%		\framesubtitle{\hspace{1cm}}
			\vspace{20pt}
					\begin{enumerate}
						\item Permintaan Pekerjaan Arsitektur
						\item Penilaian Kapabilitas
						\item Rencana Komunikasi
						\item Model Organisasi untuk Arsitektur Perusahaan
						\item Kerangka Kerja Arsitektur yang sudah Disesuaikan
						\item Prinsip data, jika ada
						\item Pernyataan Pekerjaan Arsitektur
						\item Draf Spesifikasi Persyaratan Arsitektur Rancangan:
						\begin{enumerate}
							\item Hasil Analisis Gap
							\item Persyaratan teknis relevan lain
						\end{enumerate}
					\end{enumerate}
	\end{frame}


	\begin{frame}
		\frametitle{Input (2)}
%		\framesubtitle{\hspace{1cm}}
			\vspace{20pt}
					\begin{enumerate}
						\setcounter{enumi}{8}
						\item Visi Arsitektur
						\item Repositori Arsitektur
						\item Draf Dokumen Definisi Arsitektur:
						\begin{enumerate}
							\item Arsitektur Bisnis Dasar (detil)
							\item Target Arsitektur Bisnis (detil)
							\item Arsitektur Data Dasar (tingkat tinggi)
							\item Arsitektur Data Target (garis besar)
							\item Arsitektur Aplikasi Dasar (garis besar)
							\item Arsitektur Aplikasi Target (garis besar)
							\item Arsitektur Teknologi Dasar (garis besar)
							\item Arsitektur Teknologi Target (garis besar)
						\end{enumerate}
						\item Komponen Arsitektur Bisnis dari Peta Jalan Arsitektur
					\end{enumerate}
	\end{frame}


	\begin{frame}
		\frametitle{Langkah-langkah}
        \vspace{20pt}
		\begin{enumerate}
			\item Pilih model referensi, sudut pandang, dan alat
			\item Kembangkan Deskripsi Arsitektur Data Baseline
			\item Kembangkan Deskripsi Arsitektur Data Target
			\item Lakukan Analisis Gap
			\item Menentukan kandidat komponen peta jalan
			\item Selesaikan dampak di Lanskap Arsitektur
			\item Lakukan tinjauan formal pemangku kepentingan
			\item Menyelesaikan Arsitektur Data
			\item Buat Dokumen Definisi Arsitektur
		\end{enumerate}
	\end{frame}

	\begin{frame}
		\frametitle{Output}
%		\framesubtitle{\hspace{1cm}}
		\begin{enumerate}
			\item Pernyataan Pekerjaan Arsitektur, diperbarui jika diperlukan
			\item Prinsip data tervalidasi, atau prinsip data baru
			\item Draf Dokumen Definisi Arsitektur, dengan isi yang telah diperbarui
			\item Spesifikasi Persyaratan Arsitektur Rancangan, termasuk yang sudah diperbarui
			\item Komponen Arsitektur Data dari Peta Jalan Arsitektur
		\end{enumerate}
	\end{frame}

	\begin{frame}
		\frametitle{Prinsip-prinsip Data}
%		\framesubtitle{\hspace{1cm}}
		\begin{enumerate}
			\item \textbf{Data adalah Aset}.
			Data adalah aset yang memiliki nilai bagi perusahaan dan dikelola sesuai dengan itu.
			\item \textbf{Data is shared}.
			Pengguna memiliki akses ke data yang diperlukan untuk menjalankan tugasnya; oleh karena itu, data dibagikan di seluruh fungsi perusahaan
			dan organisasi.
			\item \textbf{Data Dapat Diakses}. Data dapat diakses oleh pengguna untuk menjalankan fungsinya.
		\end{enumerate}
	\end{frame}

	\begin{frame}
		\frametitle{Prinsip-prinsip Data (2)}
%		\framesubtitle{\hspace{1cm}}
		\begin{enumerate}
			\setcounter{enumi}{3}
			\item \textbf{Definisi Kosakata dan Data Bersama}. Data didefinisikan secara konsisten di seluruh perusahaan, dan definisi tersebut dapat dimengerti dan tersedia untuk semua pengguna.
			\item \textbf{Keamanan Data}.
			Data dilindungi dari penggunaan dan pengungkapan yang tidak sah. Selain aspek tradisional keamanan nasional
			klasifikasi, ini termasuk, namun tidak terbatas pada, perlindungan pra-keputusan, sensitif, pemilihan sumber sensitif, dan
			informasi hak milik.
		\end{enumerate}
	\end{frame}


	\begin{frame}
		\frametitle{ Data Architecture Catalogs, Matrices, and Diagrams}
		\framesubtitle{\hspace{1cm}}
        \begin{small}
		\begin{table}[]
			\begin{tabular}{|c|c|c|}
				\hline
				\textbf{Catalogs} & \textbf{Matrices}   & \textbf{Diagrams} \\ \hline
				Data Entity/Data                   & Data Entity/Business Function matrix & Conceptual Data diagram            \\
				Component catalog                  & Application/Data matrix              & Logical Data diagram               \\
				&                                      & Data Dissemination diagram         \\
				&                                      & Data Lifecycle diagram             \\
				&                                      & Data Security diagram              \\
				&                                      & Data Migration diagram             \\ \hline
			\end{tabular}
		\end{table}
        \end{small}
	\end{frame}


	\begin{frame}
		\frametitle{Beberapa Hal yang Perlu Diperhatikan Khususnya }
		\framesubtitle{untuk Kerja Jarak Jauh}
		\begin{enumerate}
			\item Migrasi data offline (fisik, digital) ke online (diakses melalui internet)
			\item Transformasi data fisik dan tidak terstruktur ke digital dan terstruktur
			\item Kemananan data: enkripsi, manajemen password, hak akses, dll
			\item Tata kelola data: otorisasi dan pengaturan CRUD (create, read, update, delete) data, berapa lama data disimpan sebelum dihapus
			\item Bagaimana menyimpan, membagikan, mengakses, memperoses data
			\item backup dan accident/disaster recovery
			\item Biaya data: finansial, kecepatan dan beban proses, transfer, dan penyimpanan
		\end{enumerate}
	\end{frame}





	\begin{frame}
		\frametitle{Inovasi Terkait Data}
%		\framesubtitle{\hspace{1cm}}
		\begin{columns}
			\begin{column}{0.5\textwidth}
				\begin{center}
					\begin{enumerate}
						\item Text and Binary Files
						\item Relational Database Systems (RDBMS)
						\item Online transaction processing (OLTP) Data
						\item Online analytical processing (OLAP) Data

					\end{enumerate}
				\end{center}
			\end{column}
			\begin{column}{0.5\textwidth}
				\begin{center}
					\begin{enumerate}
						\setcounter{enumi}{4}

						\item Sequential vs parallel processing
						\item Centralised vs distributed data
						\item Centralised vs federated data
						\item Centralised vs Decentralised data (Web3 movement) please search Filecoin, SOLID POD
					\end{enumerate}
				\end{center}
			\end{column}
		\end{columns}
	\end{frame}

		\begin{frame}
		\frametitle{Inovasi Terkait Data (2)}
%		\framesubtitle{\hspace{1cm}}
		\begin{columns}
			\begin{column}{0.5\textwidth}
				\begin{center}
					\begin{enumerate}
						\setcounter{enumi}{8}
						\item Text and Binary Files
						\item Relational Database Systems (RDBMS)
						\item Online transaction processing (OLTP) Data
						\item Online analytical processing (OLAP) Data
					\end{enumerate}
				\end{center}
			\end{column}
			\begin{column}{0.5\textwidth}
				\begin{center}
					\begin{enumerate}
						\setcounter{enumi}{12}
						\item Sequential vs parallel processing
						\item Centralised vs distributed data
						\item Centralised vs federated data
						\item Centralised vs Decentralised data (Web3 movement) please search Filecoin, SOLID POD
					\end{enumerate}
				\end{center}
			\end{column}
		\end{columns}
	\end{frame}



	\begin{frame}
		\frametitle{Penutup}
%		\framesubtitle{\hspace{1cm}}
		\begin{itemize}
			\item Fase Arsitektur Data memungkinkan kita untuk mendefinisikan dan memvalidasi struktur data yang diperlukan untuk mendukung arsitektur bisnis.
			\item Fase ini penting untuk memastikan bahwa data dan integrasinya sesuai dengan kebutuhan bisnis.
		\end{itemize}
	\end{frame}



\end{document}
