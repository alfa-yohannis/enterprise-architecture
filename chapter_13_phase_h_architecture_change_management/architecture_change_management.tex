\documentclass{beamer}

\usetheme{Madrid}
\usecolortheme{seahorse}

% Author
\author{Alfa Yohannis}

\begin{document}
	
	\begin{frame}
		\title{Fase H: Manajemen Perubahan Arsitektur}
		\author{}
		\date{}
		\titlepage
	\end{frame}
	
	\section{Tentang Fase Manajemen Perubahan Arsitektur}
	\begin{frame}
		\frametitle{Tentang Fase Manajemen Perubahan Arsitektur}
		\begin{itemize}
		\item Fase Manajemen Perubahan Arsitektur (Phase G) adalah salah satu fase dalam metode pengembangan arsitektur TOGAF yang bertujuan untuk mengelola perubahan pada arsitektur organisasi. 
		\item Fase ini memastikan bahwa perubahan yang terjadi didasarkan pada alasan yang jelas, sesuai dengan strategi bisnis, serta disetujui oleh para stakeholder yang terkait.
		\end{itemize}
	\end{frame}
	
	\section{Tujuan}
	\begin{frame}
		\frametitle{Tujuan}
		\begin{itemize}
			\item Memastikan bahwa perubahan pada arsitektur didasarkan pada alasan yang jelas dan sesuai dengan strategi bisnis.
			\item Menyediakan mekanisme untuk mengelola dan mengawasi perubahan arsitektur.
			\item Melibatkan stakeholder terkait dalam proses pengambilan keputusan terkait perubahan arsitektur.
		\end{itemize}
	\end{frame}
	
	\section{Langkah-langkah}
	\begin{frame}
		\frametitle{Langkah-langkah}
		Fase Manajemen Perubahan Arsitektur melibatkan beberapa langkah, antara lain:
		\begin{enumerate}
			\item Analisis Dampak Perubahan
			\item Validasi Perubahan Arsitektur
			\item Pemantauan dan Evaluasi Kinerja Arsitektur yang Diubah
		\end{enumerate}
	\end{frame}
	
	\section{Keluaran}
	\begin{frame}
		\frametitle{Keluaran}
		Fase Manajemen Perubahan Arsitektur menghasilkan beberapa keluaran, antara lain:
		\begin{itemize}
			\item Perubahan pada arsitektur yang terdokumentasi.
			\item Perubahan pada arsitektur aplikasi dan teknologi.
			\item Pembaruan rencana arsitektur.
		\end{itemize}
	\end{frame}
	
	\section{Contoh}
	\begin{frame}
		\frametitle{Contoh}
		Berikut ini adalah beberapa contoh perubahan arsitektur yang dapat terjadi dalam Fase Manajemen Perubahan Arsitektur:
		\begin{itemize}
			\item Migrasi dari sistem legacy ke sistem modern
			\item Penggunaan teknologi baru dalam infrastruktur IT
			\item Perubahan kebijakan keamanan IT
				\item Penerapan solusi cloud computing sebagai bagian dari transformasi digital.
				\item Integrasi sistem yang ada dengan aplikasi pihak ketiga untuk meningkatkan fungsionalitas.
				\item Penyesuaian arsitektur untuk mendukung pengembangan produk baru atau peluncuran layanan baru.
			\end{itemize}
		\end{frame}
	
\section{Kesimpulan}
\begin{frame}
	\frametitle{Kesimpulan}
	\begin{itemize}
	
	\item Dalam fase Manajemen Perubahan Arsitektur, langkah-langkah seperti analisis dampak perubahan, validasi perubahan arsitektur, dan pemantauan kinerja arsitektur yang diubah dilakukan untuk memastikan perubahan yang terjadi sesuai dengan rencana dan memenuhi persyaratan yang telah ditetapkan.
	
	\item Fase Manajemen Perubahan Arsitektur menghasilkan output berupa perubahan yang terdokumentasi pada arsitektur organisasi, termasuk perubahan pada arsitektur aplikasi dan teknologi.
	
	\item Fase ini penting dalam menghadapi perubahan bisnis dan teknologi yang terus berkembang. Dengan mengelola perubahan arsitektur dengan baik, organisasi dapat tetap relevan dan bersaing di pasar yang semakin kompetitif.
\end{itemize}
\end{frame}

\end{document}


