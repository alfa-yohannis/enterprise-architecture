\documentclass{beamer}

\usepackage[utf8]{inputenc}
\usetheme{Madrid}
\usecolortheme{seahorse}

\title{Fase B: Arsitektur Bisnis dalam Metode Pengembangan Arsitektur TOGAF}
\author{Alfa Yohannis}
\date{\today}

\begin{document}

\frame{\titlepage}

\begin{frame}
\frametitle{Tujuan}
\begin{itemize}
\item Mendefinisikan arsitektur bisnis yang diperlukan untuk mendukung strategi bisnis
\item Menyelaraskan arsitektur bisnis dengan kebutuhan stakeholder
\end{itemize}
\end{frame}

\begin{frame}
\frametitle{Input}
\begin{itemize}
\item Pernyataan Kebutuhan Arsitektur
\item Strategi Bisnis Organisasi
\end{itemize}
\end{frame}

\begin{frame}
\frametitle{Langkah-langkah}
\begin{enumerate}
\item Mengembangkan Baseline Arsitektur Bisnis
\item Mengembangkan Arsitektur Sasaran Bisnis
\item Melakukan Gap Analysis
\item Mendefinisikan peta jalan bisnis
\end{enumerate}
\end{frame}

\begin{frame}
\frametitle{Output}
\begin{itemize}
\item Baseline Arsitektur Bisnis
\item Arsitektur Sasaran Bisnis
\item Peta jalan bisnis
\end{itemize}
\end{frame}

\begin{frame}
\frametitle{Contoh}
\begin{itemize}
\item Contoh Baseline Arsitektur Bisnis: Diagram arsitektur bisnis saat ini
\item Contoh Arsitektur Sasaran Bisnis: Diagram arsitektur bisnis masa depan
\end{itemize}
\end{frame}

\begin{frame}
\frametitle{Ringkasan}
Fase Arsitektur Bisnis memungkinkan kita untuk mendefinisikan dan memvalidasi arsitektur bisnis yang diperlukan untuk mendukung strategi bisnis. Fase ini penting untuk memastikan bahwa arsitektur bisnis sejalan dengan kebutuhan stakeholder.
\end{frame}

\end{document}
