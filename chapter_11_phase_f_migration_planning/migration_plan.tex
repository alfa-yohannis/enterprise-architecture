\documentclass{beamer}
\usepackage[utf8]{inputenc}

\usetheme{Madrid}
\usecolortheme{seahorse}

\title{Fase F: Perencanaan Migrasi dalam ADM TOGAF}
\author{Alfa Yohannis}
\date{13 May 2023}

\begin{document}
	
	\begin{frame}
		\titlepage
	\end{frame}
	
	\begin{frame}
		\frametitle{Perencanaan Migrasi dalam ADM TOGAF}
		\begin{itemize}
			\item Ini adalah fase F dalam ADM TOGAF.
			\item Tujuan dari fase ini adalah mengembangkan rencana detail untuk memindahkan sistem dan proses yang ada ke arsitektur yang diinginkan.
		\end{itemize}
	\end{frame}
	
	\begin{frame}
		\frametitle{Input Perencanaan Migrasi}
		\begin{itemize}
			\item Model Arsitektur Bisnis (BAM): Memberikan pemahaman tentang proses bisnis dan tujuan organisasi.
			\item Arsitektur yang Diusulkan: Merupakan hasil dari fase Desain Arsitektur yang mencakup solusi arsitektur yang diusulkan.
			\item Evaluasi Performa Bisnis: Menyediakan wawasan tentang kinerja bisnis saat ini dan potensial peningkatan di masa depan.
			\item Rencana Transformasi Arsitektur: Memberikan panduan tentang perubahan organisasi dan teknologi yang direncanakan.
		\end{itemize}
	\end{frame}
	
	\begin{frame}
		\frametitle{Langkah-langkah Perencanaan Migrasi}
		\begin{enumerate}
			\item Menganalisis Gap: Mengidentifikasi perbedaan antara arsitektur saat ini dan arsitektur yang diinginkan.
			\item Menentukan Pendekatan Migrasi: Mengembangkan strategi dan pendekatan untuk memindahkan sistem dan proses yang ada.
			\item Menentukan Prioritas dan Urutan Migrasi: Menetapkan urutan dan prioritas migrasi berdasarkan kepentingan bisnis dan kompleksitas teknis.
			\item Mengembangkan Rencana Pelaksanaan Migrasi: Mengembangkan rencana rinci yang mencakup jadwal, alokasi sumber daya, dan tanggung jawab.
			\item Mengidentifikasi Risiko dan Mitigasi: Mengidentifikasi risiko migrasi dan mengembangkan rencana untuk mengurangi atau menghilangkan risiko tersebut.
		\end{enumerate}
	\end{frame}
	
	\begin{frame}
		\frametitle{Output Perencanaan Migrasi}
		\begin{itemize}
			\item Rencana Migrasi Arsitektur: Rencana detail yang mencakup langkah-langkah migrasi, jadwal, biaya, dan sumber daya yang diperlukan.
			\item Rencana Pelaksanaan Migrasi: Rincian tugas, tanggung jawab, dan jadwal untuk melaksanakan migrasi.
			\item Rencana Komunikasi: Rencana untuk berkomunikasi dengan pemangku terkait dan anggota tim terkait migrasi.
\item Rencana Pelatihan: Rencana untuk melatih anggota tim terkait migrasi tentang sistem baru dan proses yang akan diimplementasikan.
\item Evaluasi Risiko: Evaluasi risiko yang terkait dengan migrasi dan rekomendasi mitigasi risiko.
\end{itemize}
\end{frame}

\begin{frame}
\frametitle{Contoh Rencana Migrasi}
Berikut adalah contoh format rencana migrasi yang dapat digunakan:

\begin{itemize}
\item Deskripsi Proyek: Menjelaskan latar belakang dan tujuan dari proyek migrasi.
\item Analisis Gap: Mengidentifikasi perbedaan antara arsitektur saat ini dan arsitektur yang diinginkan.
\item Strategi Migrasi: Menjelaskan strategi dan pendekatan yang akan digunakan untuk memindahkan sistem dan proses yang ada.
\item Rencana Pelaksanaan: Menyajikan rencana terperinci yang mencakup jadwal, alokasi sumber daya, dan tanggung jawab.
\item Risiko dan Mitigasi: Mengidentifikasi risiko yang terkait dengan migrasi dan mengembangkan rencana mitigasi risiko.
\end{itemize}
\end{frame}

\begin{frame}
\frametitle{Ringkasan}
\begin{itemize}
\item Perencanaan Migrasi adalah langkah penting dalam ADM TOGAF yang melibatkan pengembangan rencana detail untuk memindahkan sistem dan proses yang ada ke arsitektur yang diinginkan.
\item Langkah-langkah dalam perencanaan migrasi termasuk menganalisis gap, menentukan pendekatan migrasi, menentukan prioritas dan urutan migrasi, mengembangkan rencana pelaksanaan migrasi, dan mengidentifikasi risiko dan mitigasi.
\item Output dari perencanaan migrasi termasuk rencana migrasi arsitektur, rencana pelaksanaan migrasi, rencana komunikasi, rencana pelatihan, dan evaluasi risiko.
\end{itemize}
\end{frame}

\begin{frame}
\frametitle{Terima Kasih}
\centering
Terima kasih atas perhatiannya. Apakah ada pertanyaan?
\end{frame}

\end{document}

terkait dan anggota tim terkait migrasi.
			\item Rencana Pelatihan: Rencana untuk melatih anggota tim terkait migrasi tentang sistem baru dan proses yang akan diimplementasikan.
			\item Evaluasi Risiko: Evaluasi risiko yang terkait dengan migrasi dan rekomendasi mitigasi risiko.
		\end{itemize}
	\end{frame}

\begin{frame}
\frametitle{Contoh Rencana Migrasi}
Berikut adalah contoh format rencana migrasi yang dapat digunakan:

\begin{itemize}
\item Deskripsi Proyek: Menjelaskan latar belakang dan tujuan dari proyek migrasi.
\item Analisis Gap: Mengidentifikasi perbedaan antara arsitektur saat ini dan arsitektur yang diinginkan.
\item Strategi Migrasi: Menjelaskan strategi dan pendekatan yang akan digunakan untuk memindahkan sistem dan proses yang ada.
\item Rencana Pelaksanaan: Menyajikan rencana terperinci yang mencakup jadwal, alokasi sumber daya, dan tanggung jawab.
\item Risiko dan Mitigasi: Mengidentifikasi risiko yang terkait dengan migrasi dan mengembangkan rencana mitigasi risiko.
\end{itemize}
\end{frame}

\begin{frame}
\frametitle{Ringkasan}
\begin{itemize}
\item Perencanaan Migrasi adalah langkah penting dalam ADM TOGAF yang melibatkan pengembangan rencana detail untuk memindahkan sistem dan proses yang ada ke arsitektur yang diinginkan.
\item Langkah-langkah dalam perencanaan migrasi termasuk menganalisis gap, menentukan pendekatan migrasi, menentukan prioritas dan urutan migrasi, mengembangkan rencana pelaksanaan migrasi, dan mengidentifikasi risiko dan mitigasi.
\item Output dari perencanaan migrasi termasuk rencana migrasi arsitektur, rencana pelaksanaan migrasi, rencana komunikasi, rencana pelatihan, dan evaluasi risiko.
\end{itemize}
\end{frame}

\begin{frame}
\frametitle{Terima Kasih}
\centering
Terima kasih atas perhatiannya. Apakah ada pertanyaan?
\end{frame}

\end{document}

