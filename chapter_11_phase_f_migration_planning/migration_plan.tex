\documentclass[aspectratio=169, table]{beamer}

%\usepackage[beamertheme=./praditatheme]{Pradita}
\usepackage[utf8]{inputenc}

\usetheme{Pradita}

\subtitle{MTI102-Information System \&\\Technology Architecture}
\title{\Large Phase F: Migration Plan\\in TOGAF
	Architecture\\Development Method (ADM)}
\date[Serial]{\scriptsize {PRU/SPMI/FR-BM-18/0222}}
\author[Pradita]{\small {\textbf{Alfa Yohannis}}}

\begin{document}

	\frame{\titlepage}

	\begin{frame}
		\frametitle{Aims}
		\begin{enumerate}
			\item Finalize the Architecture Roadmap and the supporting Implementation and Migration Plan
			\item Ensure that the Implementation and Migration Plan is coordinated with the enterprise’s approach to managing and implementing change in the enterprise’s overall change portfolio
			\item Ensure that the business value and cost of work packages and Transition Architectures are understood by key stakeholders
		\end{enumerate}
	\end{frame}

	\begin{frame}
		\frametitle{Inputs}
		\vspace{22pt}
		\begin{columns}[onlytextwidth]
			\begin{column}{0.45\textwidth}
				\begin{enumerate}
					\item Request for Architecture Work
					\item Communications Plan
					\item Organizational Model for Enterprise Architecture
					\item Governance Models and Frameworks
					\item Tailored Architecture Framework
					\item Statement of Architecture Work
					\item Architecture Vision
					\item Architecture Repository
					\item Draft Architecture Requirements Specification
				\end{enumerate}

			\end{column}
			\begin{column}{0.55\textwidth}
				\begin{enumerate}
					\setcounter{enumi}{8}
					\item Draft Architecture Definition Document, including:
					\begin{enumerate}
						\item Transition Architectures, if any
					\end{enumerate}
					\item Change Requests for existing programs and projects
					\item Architecture Roadmap, including:
					\begin{enumerate}
						\item Identification of work packages
						\item Identification of Transition Architectures
						\item Implementation Factor Assessment and Deduction Matrix
					\end{enumerate}
					\item Capability Assessment
					\item Implementation and Migration Plan (outline)
				\end{enumerate}
			\end{column}
		\end{columns}
	\end{frame}


	\begin{frame}
		\frametitle{Steps}
		\vspace{20pt}
		\begin{enumerate}
			\item Confirm management framework interactions for the Implementation and Migration Plan
			\item Assign a business value to each work package
			\item Estimate resource requirements, project timings, and availability/delivery vehicle
			\item Prioritize the migration projects through the conduct of a cost/benefit assessment and risk validation
			\item Confirm Architecture Roadmap and update Architecture Definition Document
			\item Complete the Implementation and Migration Plan
			\item Complete the architecture development cycle and document lessons learned
		\end{enumerate}
	\end{frame}


	\begin{frame}
		\frametitle{Outputs}
		\vspace{20pt}
		\begin{enumerate}
			\item Implementation and Migration Plan (detailed)
			\item Finalized Architecture Definition Document, including:
			\begin{enumerate}
				\item Finalized Transition Architectures, if any
			\end{enumerate}
			\item Finalized Architecture Requirements Specification
			\item Finalized Architecture Roadmap
			\item Re-Usable Architecture Building Blocks (ABBs)
			\item Requests for Architecture Work for a new iteration of the ADM cycle (if any)
			\item Implementation Governance Model
			\item Change Requests for the Architecture Capability arising from lessons learned
		\end{enumerate}
	\end{frame}


\section{Implementation and Migration Plan}

\begin{frame}{Implementation and Migration Plan}
	\begin{itemize}
		\item The Implementation and Migration Plan is developed in Phases E and F.
		\item Provides a project schedule for implementing the Target Architecture.
		\item Includes executable projects grouped into managed portfolios and programs.
		\item The Implementation and Migration Strategy is a key element, identifying the approach to change.
	\end{itemize}
\end{frame}

\begin{frame}{\LARGE{Contents of Implementation and Migration Plan}}
	\vspace{15pt}
	\begin{columns}[T] % Aligns the columns at the top
		\begin{column}{0.48\textwidth}
			\begin{itemize}
				\item \textbf{Implementation and Migration Strategy:}
				\begin{itemize}
					\item Strategic direction for implementation
					\item Approach to implementation sequencing
				\end{itemize}
				\item \textbf{Project and portfolio details:}
				\begin{itemize}
					\item Allocation of work packages to projects and portfolios
					\item Capabilities provided by the projects
					\item Milestones and schedules
					\item Work breakdown structure
				\end{itemize}
			\end{itemize}
		\end{column}
		
		\begin{column}{0.48\textwidth}
			\begin{itemize}
				\item \textbf{Possible project documentation:}
				\begin{itemize}
					\item Included work packages
					\item Business value
					\item Risks, issues, assumptions, and dependencies
					\item Resource and cost requirements
					\item Migration benefits (including mapping to business needs)
					\item Estimated costs of migration options
				\end{itemize}
			\end{itemize}
		\end{column}
	\end{columns}
\end{frame}


\section{Architecture Definition Document, including Transition Architecture if required}

\begin{frame}{Architecture Definition Document including}
\framesubtitle{Transition Architecture if required}
	\begin{itemize}
		\item The Architecture Definition Document is completed in this phase.
		\item Includes detailed descriptions of the Target Architecture.
		\item If a phased approach is needed, Transition Architectures will be defined.
		\item Transition Architecture describes significant conditions between the Baseline and Target Architecture.
		\item Used to define the stages toward the Target Architecture.
	\end{itemize}
\end{frame}

\begin{frame}{Contents of the Transition Architecture}
	\begin{itemize}
		\item Definition of transition states
		\item Business Architecture for each transition state
		\item Data Architecture for each transition state
		\item Application Architecture for each transition state
		\item Technology Architecture for each transition state
	\end{itemize}
\end{frame}

\section{Implementation Governance Model}

\begin{frame}{Implementation Governance Model}
	\begin{itemize}
		\item Once the architecture is defined, planning is required to manage the Transition Architecture during implementation.
		\item In organizations with established architecture functions, a governance framework may already be in place.
		\item However, specific processes, roles, responsibilities, and measurements may need to be defined per project.
		\item The Implementation Governance Model ensures that projects in the implementation stage enter appropriate Architecture Governance (Phase G).
	\end{itemize}
\end{frame}

\begin{frame}{\Large{Typical Contents of the Implementation Governance Model}}
	\begin{itemize}
		\item Governance processes
		\item Governance organizational structure
		\item Governance roles and responsibilities
		\item Governance checkpoints and success/failure criteria
	\end{itemize}
\end{frame}

	\begin{frame}
		\frametitle{Summary}
		\begin{enumerate}
			\item Phase F addresses migration planning, which focuses on moving from the Baseline to the Target Architectures.
			\item It includes creating the finalized Architecture Definition Document, Architecture Roadmap, and the detailed Implementation and Migration Plan.
			\item After completion of this phase, the preparation for implementation has been completed.
		\end{enumerate}
	\end{frame}

\end{document}
