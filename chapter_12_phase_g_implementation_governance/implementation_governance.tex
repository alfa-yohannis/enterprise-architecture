\documentclass{beamer}
\usepackage[utf8]{inputenc}

\usetheme{Madrid}
\usecolortheme{seahorse}

\title{Fase G: Pengawasan Implementasi dalam TOGAF's Architecture Development Method}
\author{Alfa Yohannis}
\date{13 May 2023}

\begin{document}
	
	\frame{\titlepage}
	
	\begin{frame}
		\frametitle{Pengantar}
		\begin{itemize}
			\item TOGAF (The Open Group Architecture Framework) adalah kerangka kerja yang membantu dalam pengembangan dan pengelolaan arsitektur perusahaan.
			\item Salah satu fase penting dalam TOGAF, fase G, adalah Pengawasan Implementasi.
			\item Pengawasan Implementasi bertujuan untuk memastikan bahwa arsitektur yang dihasilkan juga diimplementasikan dengan baik dan sesuai dengan rencana.
		\end{itemize}
	\end{frame}
	
	\begin{frame}
		\frametitle{Tujuan}
		\begin{itemize}
			\item Memastikan bahwa implementasi arsitektur berlangsung sesuai dengan rencana dan spesifikasi yang telah ditetapkan.
			\item Memantau dan mengendalikan perubahan pada arsitektur yang mungkin terjadi selama implementasi.
			\item Mengevaluasi kualitas dan keberhasilan implementasi arsitektur.
		\end{itemize}
	\end{frame}
	
	\begin{frame}
		\frametitle{Input}
		\begin{itemize}
			\item Rencana Implementasi Arsitektur (Architecture Implementation Plan)
			\item Rencana Pengelolaan Perubahan (Change Management Plan)
			\item Hasil dari fase-fase sebelumnya dalam TOGAF's Architecture Development Method
		\end{itemize}
	\end{frame}
	
	\begin{frame}
		\frametitle{Langkah-langkah}
		\begin{enumerate}
			\item Memastikan ketersediaan dan implementasi rencana pengelolaan perubahan.
			\item Memantau dan mengendalikan implementasi arsitektur.
			\item Mengevaluasi kualitas dan keberhasilan implementasi arsitektur.
			\item Melaporkan hasil evaluasi kepada pemangku kepentingan.
			\item Mengidentifikasi dan menangani masalah yang muncul selama implementasi.
		\end{enumerate}
	\end{frame}
	
	\begin{frame}
		\frametitle{Output}
		\begin{itemize}
			\item Laporan Implementasi Arsitektur (Architecture Implementation Report)
			\item Catatan perubahan pada arsitektur yang diimplementasikan
			\item Evaluasi kualitas dan keberhasilan implementasi
			\item Rencana aksi untuk mengatasi masalah yang teridentifikasi
		\end{itemize}
	\end{frame}
	
	\begin{frame}
		\frametitle{Contoh}
		\begin{itemize}
			\item Memastikan bahwa semua aplikasi dan infrastruktur yang diperlukan untuk implementasi arsitektur tersedia dan siap digunakan.
			\item Melakukan pemantauan progres implementasi dan memastikan bahwa pekerjaan dilakukan sesuai dengan jadwal yang telah ditetapkan.
			\item Mengevaluasi kualitas dan keberhasilan implementasi berdasarkan kriteria yang telah ditetapkan, seperti kepatuhan terhadap standar arsitektur, performa sistem, dan kepuasan pengguna.
			\item Melaporkan hasil evaluasi kepada pemangku kepentingan, termasuk manajemen senior dan tim proyek.
			\item Mengidentifikasi dan menangani masalah yang muncul selama implementasi, misalnya, konflik antara arsitektur yang diusulkan dan sistem yang ada.
		\end{itemize}
	\end{frame}

\begin{frame}
\frametitle{Ringkasan}
\begin{itemize}
\item Pengawasan Implementasi merupakan fase penting dalam TOGAF's Architecture Development Method untuk memastikan bahwa arsitektur yang dihasilkan juga diimplementasikan dengan baik dan sesuai dengan rencana.
\item Tujuan dari Pengawasan Implementasi adalah memastikan implementasi yang sesuai, memantau perubahan, dan mengevaluasi kualitas serta keberhasilan implementasi.
\item Langkah-langkah dalam Pengawasan Implementasi meliputi memastikan ketersediaan dan implementasi rencana pengelolaan perubahan, memantau dan mengendalikan implementasi, mengevaluasi kualitas dan keberhasilan implementasi, melaporkan hasil evaluasi kepada pemangku kepentingan, serta mengidentifikasi dan menangani masalah yang muncul.
\item Output dari fase ini meliputi Laporan Implementasi Arsitektur, catatan perubahan, evaluasi kualitas dan keberhasilan implementasi, serta rencana aksi untuk mengatasi masalah yang teridentifikasi.
\end{itemize}
\end{frame}

\end{document}