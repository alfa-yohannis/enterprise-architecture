\documentclass{beamer}

\usepackage[utf8]{inputenc}
\usetheme{Madrid}
\usecolortheme{seahorse}

\title{Fase C: Arsitektur Aplikasi dalam Metode Pengembangan Arsitektur TOGAF}
\author{Alfa Yohannis}
\date{\today}

\begin{document}

\frame{\titlepage}

\begin{frame}
\frametitle{Tujuan}
\begin{itemize}
\item Mendefinisikan kerangka kerja untuk aplikasi yang diperlukan untuk mendukung bisnis
\item Membuat interkoneksi antara aplikasi
\end{itemize}
\end{frame}

\begin{frame}
\frametitle{Input}
\begin{itemize}
\item Arsitektur Bisnis
\item Pernyataan Kebutuhan Arsitektur
\end{itemize}
\end{frame}

\begin{frame}
\frametitle{Langkah-langkah}
\begin{enumerate}
\item Mengembangkan Baseline Arsitektur Aplikasi
\item Mengembangkan Arsitektur Sasaran Aplikasi
\item Melakukan Gap Analysis
\item Mendefinisikan peta jalan aplikasi
\end{enumerate}
\end{frame}

\begin{frame}
\frametitle{Output}
\begin{itemize}
\item Baseline Arsitektur Aplikasi
\item Arsitektur Sasaran Aplikasi
\item Peta jalan aplikasi
\end{itemize}
\end{frame}

\begin{frame}
\frametitle{Contoh}
\begin{itemize}
\item Contoh Baseline Arsitektur Aplikasi: Diagram arsitektur aplikasi saat ini
\item Contoh Arsitektur Sasaran Aplikasi: Diagram arsitektur aplikasi masa depan
\end{itemize}
\end{frame}

\begin{frame}
\frametitle{Ringkasan}
Fase Arsitektur Aplikasi memungkinkan kita untuk mendefinisikan dan memvalidasi arsitektur aplikasi yang diperlukan untuk mendukung arsitektur bisnis. Fase ini penting untuk memastikan bahwa aplikasi dan interkoneksinya sesuai dengan kebutuhan bisnis.
\end{frame}

\end{document}
