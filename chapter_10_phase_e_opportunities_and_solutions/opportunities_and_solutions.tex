\documentclass{beamer}

\usepackage[utf8]{inputenc}
\usepackage{babel}
\usetheme{Madrid}
\usecolortheme{seahorse}

\title{Fase E: Peluang dan Solusi dalam Metode Pengembangan Arsitektur TOGAF}
\author{Alfa Yohannis}
\date{\today}

\begin{document}

\frame{\titlepage}

\begin{frame}
\frametitle{Tujuan}
\begin{itemize}
\item Mengidentifikasi peluang dan solusi yang dapat mencapai arsitektur yang diinginkan
\item Membuat rencana implementasi yang rinci
\end{itemize}
\end{frame}

\begin{frame}
\frametitle{Input}
\begin{itemize}
\item Arsitektur visi
\item Pernyataan arsitektur
\item Pernyataan kebutuhan
\end{itemize}
\end{frame}

\begin{frame}
\frametitle{Langkah-langkah}
\begin{enumerate}
\item Menentukan peluang
\item Menyusun solusi
\item Membuat rencana implementasi
\end{enumerate}
\end{frame}

\begin{frame}
\frametitle{Output}
\begin{itemize}
\item Daftar peluang dan solusi
\item Rencana implementasi arsitektur
\end{itemize}
\end{frame}

\begin{frame}
\frametitle{Contoh}
\begin{itemize}
\item Contoh peluang: Migrasi ke cloud
\item Contoh solusi: Implementasi layanan berbasis cloud
\end{itemize}
\end{frame}

\begin{frame}
\frametitle{Ringkasan}
Fase Peluang dan Solusi memungkinkan kita untuk mengidentifikasi dan menyusun solusi arsitektur yang efektif dan efisien. Ini membantu dalam merencanakan implementasi arsitektur secara rinci.
\end{frame}

\end{document}
