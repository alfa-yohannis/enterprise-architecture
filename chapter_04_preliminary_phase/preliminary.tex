\documentclass{beamer}

\usepackage[utf8]{inputenc}
\usetheme{Madrid}
\usecolortheme{seahorse}

\title{Fase Awal (Preliminary) dalam Metode Pengembangan Arsitektur TOGAF}
\author{Alfa Yohannis}
\date{\today}

\begin{document}

\frame{\titlepage}

\begin{frame}
\frametitle{Tujuan}
\begin{itemize}
\item Mempersiapkan organisasi untuk mendefinisikan dan menerapkan arsitektur
\item Mendefinisikan kerangka kerja arsitektur yang akan digunakan sepanjang proyek
\end{itemize}
\end{frame}

\begin{frame}
\frametitle{Input}
\begin{itemize}
\item Tujuan bisnis dan strategi organisasi
\item Arsitektur organisasi saat ini
\end{itemize}
\end{frame}

\begin{frame}
\frametitle{Langkah-langkah}
\begin{enumerate}
\item Mengidentifikasi stakeholder
\item Menyiapkan tim arsitektur
\item Mendefinisikan kerangka kerja arsitektur
\item Membuat pernyataan kebijakan arsitektur
\end{enumerate}
\end{frame}

\begin{frame}
\frametitle{Output}
\begin{itemize}
\item Organisasi dan tim arsitektur yang dipersiapkan
\item Kerangka kerja arsitektur yang didefinisikan
\item Pernyataan kebijakan arsitektur
\end{itemize}
\end{frame}

\begin{frame}
\frametitle{Contoh}
\begin{itemize}
\item Contoh kerangka kerja arsitektur: Diagram atau deskripsi kerangka kerja yang akan digunakan dalam proyek
\item Contoh pernyataan kebijakan arsitektur: Dokumen yang menetapkan kebijakan dan prosedur untuk arsitektur
\end{itemize}
\end{frame}

\begin{frame}
\frametitle{Ringkasan}
Fase Awal dalam Metode Pengembangan Arsitektur TOGAF adalah tentang mempersiapkan organisasi dan mendefinisikan kerangka kerja untuk proyek arsitektur. Fase ini sangat penting untuk memastikan bahwa tim, proses, dan alat yang tepat ada di tempat sebelum proyek dimulai.
\end{frame}

\end{document}
