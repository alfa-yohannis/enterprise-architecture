\documentclass[aspectratio=169]{beamer}

\usepackage[utf8]{inputenc}
\usetheme{Madrid}
\usecolortheme{seahorse}

\title{Fase Awal (Preliminary) dalam Metode Pengembangan Arsitektur TOGAF}
\author{Alfa Yohannis}
\date{\today}

\begin{document}
	
	\frame{\titlepage}
	
	\begin{frame}
		\frametitle{Sasaran}
		\begin{enumerate}
			\item \textbf{Tentukan} Kemampuan Arsitektur yang diinginkan oleh organisasi:
			\begin{enumerate}
				\item Tinjau \textbf{konteks} organisasi untuk melakukan arsitektur perusahaan
				\item Identifikasi \textbf{cakupan} bagian-bagian organisasi perusahaan yang terpengaruh oleh Kemampuan Arsitektur
				\item Identifikasi \textbf{kerangka kerja}, \textbf{metode}, dan \textbf{proses} \textbf{yang telah ditetapkan} yang \textbf{berhubungan} dengan Kemampuan Arsitektur
				\item Tetapkan \textbf{target} Kematangan Kemampuan
			\end{enumerate}
			\item \textbf{Tetapkan} Kemampuan Arsitektur:
			\begin{enumerate}
				\item Tentukan dan tetapkan \textbf{Model Organisasi} (orang, jabatan, peran, wewenang, departemen terkait, dsb.) untuk Arsitektur Perusahaan
				\item Tentukan dan tetapkan proses dan sumber daya terperinci untuk \textbf{tata kelola arsitektur}
				\item Pilih dan implementasikan \textbf{alat-alat yang mendukung} Kemampuan Arsitektur (misal, Porters' 5 forces, SWOT, stakeholder analysis, dsb.)
				\item Tentukan \textbf{prinsip-prinsip arsitektur}
			\end{enumerate}
		\end{enumerate}
		
	\end{frame}
	
	\begin{frame}
		\frametitle{Input}
		\begin{enumerate}
			\item Kerangka Kerja TOGAF
			\item Kerangka Kerja Arsitektur Lainnya
			\item Rencana bisnis, strategi bisnis, strategi TI
			\item Prinsip-prinsip bisnis, tujuan bisnis, dan pendorong bisnis
			\item Kerangka kerja tata Kelola dan hukum, termasuk strategi Tata Kelola Arsitektur
			\item Model Organisasi yang sudah ada untuk Arsitektur Perusahaan
			\item Kerangka Kerja Arsitektur yang sudah ada, jika ada
		\end{enumerate}
	\end{frame}
	
	\begin{frame}
		\frametitle{Langkah-langkah}
		\begin{enumerate}
			\item Menyusun ruang lingkup organisasi perusahaan yang berhubungan/terpengaruh
			\item Mengkonfirmasi dukungan untuk peningkatan kemampuan arsitektur dan kerangka kerja tata kelola
			\item Mendefinisikan dan membentuk tim dan organisasi arsitektur perusahaan
			\item Mengidentifikasi dan menetapkan prinsip-prinsip arsitektur
			\item Menyesuaikan kerangka kerja TOGAF dan, jika ada, kerangka kerja Arsitektur lain yang dipilih
			\item Mengimplementasikan alat-alat arsitektur
		\end{enumerate}
	\end{frame}
	
	\begin{frame}
		\frametitle{Output}
		\begin{enumerate}
			\item Model Organisasi untuk Arsitektur Perusahaan
			\item Kerangka Kerja Arsitektur yang di-sesuaikan/adaptasi, termasuk prinsip-prinsip arsitektur
			\item Repositori Arsitektur Awal
			\item Penegasan ulang atau referensi terhadap prinsip-prinsip bisnis, tujuan bisnis, dan pendorong bisnis
			\item Permintaan untuk pengerjaan arsitektur (dokumen resmi)
			\item Kerangka Kerja Tata Kelola Arsitektur
		\end{enumerate}
	\end{frame}
	
	\begin{frame}
		\frametitle{Mendefinisikan dan Membentuk Tim dan Organisasi Arsitektur Perusahaan}
		\begin{itemize}
			\item Menentukan kapabilitas perusahaan dan bisnis yang sudah ada
			\item Melakukan penilaian kematangan arsitektur/perubahan bisnis
			\item Mengidentifikasi kesenjangan di area kerja yang sudah ada
			\item Mengalokasikan peran dan tanggung jawab kunci untuk manajemen dan pengawasan kemampuan arsitektur perusahaan
			\item Menulis permintaan perubahan untuk proyek-proyek yang sudah ada (dokumen resmi)
			\item Menentukan batasan-batasan dalam pekerjaan arsitektur perusahaan
			\item Meninjau dan menyetujui dengan sponsor dan dewan
			\item Menilai persyaratan anggaran
		\end{itemize}
	\end{frame}

	\begin{frame}
		\frametitle{Prinsip-prinsip Arsitektur (Contoh)}
		\begin{enumerate}
			\item \textbf{Self-Serve} (Layanan Mandiri):
			\item \textbf{Pernyataan}: Pelanggan harus dapat melayani diri sendiri.
			\item \textbf{Alasan}: Menerapkan prinsip ini akan meningkatkan kepuasan pelanggan, mengurangi biaya administratif, dan potensial meningkatkan pendapatan.
			\item \textbf{Implikasi}: Terdapat implikasi untuk meningkatkan kemudahan penggunaan dan meminimalkan kebutuhan pelatihan; misalnya, anggota harus dapat memperbarui detail kontak mereka, dll., dan dapat membeli produk keanggotaan tambahan secara online.
			\item Contoh prinsip-prinsip yang lain: pakai software open source, taat terhadap hukum, ramah lingkungan
		\end{enumerate}
	\end{frame}
	
	\begin{frame}
		\frametitle{Ringkasan}
		Fase Awal dalam Metode Pengembangan Arsitektur TOGAF adalah tentang mempersiapkan organisasi dan mendefinisikan kerangka kerja untuk proyek arsitektur. Fase ini sangat penting untuk memastikan bahwa tim, proses, dan alat yang tepat ada di tempat sebelum proyek dimulai.
	\end{frame}
	
\end{document}
