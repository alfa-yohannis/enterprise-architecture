\documentclass[aspectratio=169, table]{beamer}

%\usepackage[beamertheme=./praditatheme]{Pradita}
\usepackage[utf8]{inputenc}

\usetheme{Pradita}

\subtitle{IT130204 - Information System \&\\Technology Architecture}

\title{Session-04:\\\LARGE{TOGAF ADM Preliminary Phase\\}}
\date[Serial]{\scriptsize {PRU/SPMI/FR-BM-18/0222}}
\author[Pradita]{{\textbf{Alfa Yohannis}}}

\begin{document}

    \frame{\titlepage}



    \begin{frame}
        \frametitle{Goals}
        % \framesubtitle{\hspace{1cm}}
        \begin{enumerate}
            \item \textbf{Specify} The Architectural Capabilities desired by the organization:
            \begin{enumerate}
                \item Review the organizational \textbf{context} for conducting enterprise architecture
                \item Identify \textbf{scope} parts of the enterprise organization affected by Architectural Capabilities
                \item Identify the \textbf{framework}, \textbf{method}, and \textbf{process} \textbf{defined} that \textbf{relate} to the Architectural Capability
                \item Set \textbf{target} Capability Maturity
            \end{enumerate}
        \end{enumerate}
    \end{frame}

    \begin{frame}
        \frametitle{Goals (2)}
        % \framesubtitle{\hspace{1cm}}
        \begin{enumerate}
            \setcounter{enumi}{1}
            \item \textbf{Define} the current and required capabilities:
            \begin{enumerate}
                \item Define the \textbf{Organizational Model} (people, titles, roles, authority, related departments, etc.) for the Enterprise Architecture
                \item Define and assign detailed processes and resources for \textbf{architectural governance}
                \item Select and implement \textbf{tools that support} Architecture Capabilities (e.g., Porters' 5 forces, SWOT, stakeholder analysis, etc.)
                \item Define \textbf{architectural principles}
            \end{enumerate}
        \end{enumerate}
    \end{frame}


    \begin{frame}
        \frametitle{Input}
        % \framesubtitle{\hspace{1cm}}
        \begin{enumerate}
            \item TOGAF Framework
            \item Other Architectural Frameworks
            \item Business plan, business strategy, IT strategy
            \item Business principles, business goals, and business drivers
            \item Governance and legal frameworks, including Architectural Governance strategies
            \item Existing Organizational Models for Enterprise Architecture
            \item Existing Architectural Frameworks, if any
        \end{enumerate}
    \end{frame}

    \begin{frame}
        \frametitle{Steps}
        % \framesubtitle{\hspace{1cm}}
        \begin{enumerate}
            \item Compile the scope of company organizations that are related/affected
            \item Confirm support for enhanced architecture and governance framework capabilities
            \item Define and form an enterprise architecture team and organization
            \item Identify and define architectural principles
            \item Customizes the TOGAF framework and, if applicable, other selected Architecture frameworks
            \item Implement architectural tools
        \end{enumerate}
    \end{frame}

    \begin{frame}
        \frametitle{Output}
        % \framesubtitle{\hspace{1cm}}
        \begin{enumerate}
            \item Organizational Models for Enterprise Architecture
            \item Customized/adapted Architectural Framework, including architectural principles
            \item Early Architecture Repository
            \item Reaffirmation or reference to business principles, business goals, and business drivers
            \item Request for architectural work (official document)
            \item Architectural Governance Framework
        \end{enumerate}
    \end{frame}


    \begin{frame}
        \frametitle{Defining and Forming Teams and}
        \framesubtitle{Enterprise Architecture Organization}
        \begin{itemize}
            \item Determine the capabilities of existing companies and businesses
            \item Conduct architectural maturity/business change assessments
            \item Identify gaps in existing work areas
            \item Allocate key roles and responsibilities for management and oversight of enterprise architecture capabilities
        \end{itemize}
    \end{frame}

    \begin{frame}
        \frametitle{Defining and Forming Teams and}
        \framesubtitle{Enterprise Architecture Organization (2)}
        \begin{itemize}
            \item Write change requests for existing projects (official documents)
            \item Defining boundaries in enterprise architecture work
            \item Review and approve with sponsor and board
            \item Assess budget requirements
        \end{itemize}
    \end{frame}


\begin{frame}
    \frametitle{Principles of Architecture (Examples)}
    \framesubtitle{\hspace{1cm}}
    \vspace{20pt}
    \begin{enumerate}
        \item \textbf{Self-Serve} principle
        \item \textbf{Statement}: Customers must be able to serve themselves.
        \item \textbf{Reason}: Applying these principles will increase customer satisfaction, reduce administrative costs, and potentially increase revenue.
        \item \textbf{Implication}: There are implications for increasing ease of use and minimizing training requirements; for example, members should be able to update their contact details, etc., and be able to purchase additional membership products online.
        \item Examples of other principles: use open source software, obey the law, be environmentally friendly
    \end{enumerate}
\end{frame}

\begin{frame}
    \frametitle{Summary}
    % \framesubtitle{\hspace{1cm}}
        \begin{enumerate}
        \item The Initial Phase in the TOGAF Architecture Development Method is about preparing the organization and defining the framework for the architectural project.
        \item This phase is critical to ensuring that the right teams, processes, and tools are in place before the project begins.
    \end{enumerate}
\end{frame}

\end{document}