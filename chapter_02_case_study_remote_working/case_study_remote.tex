\documentclass[aspectratio=169, table]{beamer}

%\usepackage[beamertheme=./praditatheme]{Pradita}
\usepackage[utf8]{inputenc}

\usetheme{Pradita}

\subtitle{MTI102-Information System \&\\Technology Architecture}

\title{Studi Kasus:\\Kerja Jarak Jauh}
\date[Serial]{\scriptsize {PRU/SPMI/FR-BM-18/0222}}
\author[Pradita]{\small {\textbf{Alfa Yohannis}}}

\begin{document}

	\frame{\titlepage}

	\begin{frame}
		\frametitle{Latar Belakang: WFO vs Remote Working}
		\framesubtitle{\hspace{1cm}}
		\begin{enumerate}
			\item Covid-19 sebagai pemicu utama
			\item Kontroversi seputar \textbf{WFO} vs \textbf{Remote Working}
			\item Ada yang \textbf{pro}, ada yang \textbf{kontra} Remote Working
			\item Masalah ini sampai sekarang masih relevan:
			\begin{enumerate}
				\item Perusahaan masih menerapkan \textbf{hybrid working}
				\item \textbf{Isu-isu umum} seperti kemacetan, lingkungan (polusi, \textit{climate change}), preferensi sebagian pekerja, dsb.
			\end{enumerate}
			\item It is a real-world problem
		\end{enumerate}
	\end{frame}

	\begin{frame}
		\frametitle{Penelitian Awal}
%		\framesubtitle{\hspace{1cm}}
		\begin{enumerate}
			\item Penelitian awal di MTI Universitas Pradita
			\item Tugas makalah beberapa mahasiswa
			\item Survei dari  91 responden (target 100 orang)
			\item Bisa menjadi gambaran akan masalah remote working
			\item Jadi landasan untuk penerapan \textbf{Enterprise Architecture} di perusahaan untuk mengatasi masalah \textbf{remote working}
		\end{enumerate}
	\end{frame}

	\begin{frame}
		\frametitle{Hasil: Keuntungan Remote Working}
		\framesubtitle{\hspace{1cm}}
		\begin{enumerate}
			\item \textbf{Hemat biaya}. Mengurangi biaya transportasi bagi karyawan. Bagi perusahaan, lebih hemat biaya operasional.
			\item \textbf{Hemat waktu}. Waktu yang biasanya digunakan untuk commutting digunakan untuk yang lainnya.
			\item \textbf{Work-life balance}. Waktu yang sisa digunakan untuk keluarga, hobi, dsb. sehingga secara mental lebih sehat.
			\item \textbf{Lebih produktif}. Waktu sisa digunakan untuk mengerjakan tugas yang lain. Selain itu, waktu dan tempat lebih bisa dipilih agar hasil kerja lebih optimal dan mengurangi distraksi yang biasanya dijumpai di kantor.
		\end{enumerate}
	\end{frame}

	\begin{frame}
		\frametitle{Hasil: Tantangan dalam Remote Working}
		\framesubtitle{\hspace{1cm}}
		\begin{enumerate}
			\item \textbf{Natur Pekerjaan}. Tidak semua pekerjaan secara natur dapat dikerjakan secara jarah jauh, misal karyawan pabrik, penelitian di laboratorium, dsb.
			\item \textbf{Masalah teknis}. Koneksi internet lambat, masalah teknis pada hardware/software, infrastruktur, aplikasi saat ini tidak memadai.
			\item \textbf{Butuh investasi/pengeluaran}. Butuh dana belanja paket data, koneksi internet, hardware/software.
			\item \textbf{Masalah Etiket}. Kontak atau penugasan di luar jam kerja, email yang tidak sopan, dsb.
			\item \textbf{Gangguan dari lingkungan}. Kondisi di rumah kurang kondusif untuk bekerja.

		\end{enumerate}
	\end{frame}

	\begin{frame}
		\frametitle{Hasil: Tantangan dalam Remote Working (2)}
		\framesubtitle{\hspace{1cm}}
		\begin{enumerate}

			\item \textbf{Masalah motivasi}. Karyawan tidak termotivasi jika tidak diawasi.
			\item \textbf{Masalah trust}. Atasan tidak percaya bawahan dapat bekerja mandiri.
			\item \textbf{Masalah komunikasi}. Masalah  kebiasaan/kualitas komunikasi yang berbeda dari komunikasi secara langsung.
			\item \textbf{Masalah kesulitan bersosialisasi, kerja sama}. Kesulitan membangun team work, bersosialisasi, dan networking.
		\end{enumerate}
	\end{frame}

	\begin{frame}
		\frametitle{Solusi dari Tantangan}
%		\framesubtitle{\hspace{1cm}}
		\begin{enumerate}
			\item \textbf{Solusi teknis}. Peningkatan koneksi internet, pengadaan hardware dan software, sewa/beli/bikin sendiri, perawatan hardware/software.
			\item \textbf{Sumber daya manusia}.
			\begin{enumerate}
				\item Peningkatan kemampuan non-teknis, misal etiket, kemandirian, motivasi, komunikasi, kepercayaan
				\item Peningkatan kemampuan teknis, misal pelatihan penggunaan software untuk pengguna, tim IT untuk trouble shooting.
			\end{enumerate}
			\item \textbf{Proses bisnis, tata kelola}. Perubahan proses bisnis disesuai dengan remote working. Aturan penilaian kinerja berdasarkan target capaian, bukan berdasarkan kehadiran/jam kerja.
		\end{enumerate}
	\end{frame}

	\begin{frame}
		\frametitle{Solusi dari Tantangan (2)}
%		\framesubtitle{\hspace{1cm}}
		\begin{enumerate}
			\item \textbf{Pendanaan}. Rekalkulasi dan realokasi dana.
			\item \textbf{Sosial}. Mengadakan acara-acara sosial secara berkala untuk membangun kerja sama, jaringan, dan keakraban.
			\item \textbf{Prasarana dan support}. Penyediaan tempat atau bantuan untuk remote working.
		\end{enumerate}
	\end{frame}

	\begin{frame}
		\frametitle{Syarat Kelulusan}
%		\framesubtitle{\hspace{1cm}}
		\begin{enumerate}
			\item Tugas (proyek) Akhir
			\item Publikasi 1 kali di jurnal internasional terindeks Scopus
			\item Publikasi 2 kali di jurnal nasional terindeks SINTA (platform indeks publikasi nasional)
			\item Publikasi 2 kali di prosiding konferensi internasional terindeks Scopus (biasanya IEEE atau ACM conferences)
		\end{enumerate}
	\end{frame}

	\begin{frame}
		\frametitle{Tugas 1 Trisemester}
%		\framesubtitle{\hspace{1cm}}
		\begin{enumerate}
			\item Buatlah \textbf{makalah} yang berkaitan dengan \textbf{penggunaan Enterprise Architecture untuk penerapan Kerja Jarak Jauh}.
			\item Template untuk penulisan sudah disediakan. Maksimal 9 halaman.
			\item Gunakan hasil kuesioner dan makalah untuk memahami masalah.
			\item Masing-masing mahasiswa akan membahas \textbf{penerapan Kerja Jarak Jauh} dari salah satu fase dari \textbf{Architecture Development Method (ADM) TOGAF}.
			\item Makalah bersifat kualitatif yang mana argumen dibangun di atas data hasil dari kuesioner dan hasil kajian dari literatur terkait.
			\item Target kualitas tulisan seperti makalah contoh yang dibagikan.
		\end{enumerate}
	\end{frame}

	\begin{frame}
		\frametitle{Tugas 1 Trisemester (2)}
%		\framesubtitle{\hspace{1cm}}
		\begin{enumerate}
			\item Selain membuat makalah, \textbf{mahasiswa juga diharapkan melakukan presentasi} terkait dengan topik yang ditugaskan sebagai bahan diskusi di kelas.
			\item Adapun jadwal dan pembagian nya bisa dilihat pada spreadsheet:  \url{https://docs.google.com/spreadsheets/d/1PHF8v-jM2REgikHIZETcKoFQDtSNbOiWrY1a0XNPJrI/edit?usp=sharing}
		\end{enumerate}
	\end{frame}

	\begin{frame}
		\frametitle{Nilai}
%		\framesubtitle{\hspace{1cm}}
		\begin{enumerate}
			\item Nilai individu = Keaktifan 10\% + Presentasi 20\% + Makalah 70\%
			\item Nilai kelompok = nilai rerata kelompok
			\item Nilai akhir individu = (nilai individu + nilai rerata kelompok) / 2
		\end{enumerate}
	\end{frame}

	\begin{frame}
		\frametitle{Topik-topik Penelitian Lainnya}
		\framesubtitle{\hspace{1cm}}
		\begin{enumerate}
			\item Sebagian besar \textbf{Enterprise Architecture} fokus kepada bagaimana IT/IS dapat menunjang bisnis dan ini ditandai dengan penyebutan secara eksplisit komponen IT/IS di dalam kerangka kerja tersebut.
			\item Lalu bagaimana dengan aspek lainnya, seperti finance, human resource, legal, dsb.?
			\item Misalnya: bagaimana \textbf{TOGAF} mengatasi aspek keuangan dari \textbf{Enterprise Architecture}? Apakah ada kekurangan atau kelebihan? dsb.
			\item Keuangan bisa diganti dengan aspek-aspek lain dari perusahaan, seperti SDM, legal.
		\end{enumerate}
	\end{frame}

\end{document}
