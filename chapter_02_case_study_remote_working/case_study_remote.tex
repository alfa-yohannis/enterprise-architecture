\documentclass[aspectratio=169, table]{beamer}

%\usepackage[beamertheme=./praditatheme]{Pradita}
\usepackage[utf8]{inputenc}

\usetheme{Pradita}

\subtitle{MTI102-Information Systems \&\\Technology Architecture}

\title{Case Study:\\Remote Work}
\date[Serial]{\scriptsize {PRU/SPMI/FR-BM-18/0222}}
\author[Pradita]{\small {\textbf{Alfa Yohannis}}}

\begin{document}

    \frame{\titlepage}

    \begin{frame}
        \frametitle{Background: WFO vs Remote Working}
        \framesubtitle{\hspace{1cm}}
        \begin{enumerate}
            \item Covid-19 as the main trigger
            \item Controversy surrounding \textbf{WFO} vs \textbf{Remote Working}
            \item Some are \textbf{pro}, some are \textbf{cons} Remote Working
            \item This problem is still relevant today:
            \begin{enumerate}
                \item Companies are still implementing \textbf{hybrid working}
                \item \textbf{General issues} such as traffic jams, the environment (pollution, \textit{climate change}), preferences of some workers, etc.
            \end{enumerate}
            \item It is a real-world problem
        \end{enumerate}
    \end{frame}



    \begin{frame}
        \frametitle{Preliminary Research}
        % \framesubtitle{\hspace{1cm}}
        \begin{enumerate}
            \item Initial research at MTI Pradita University
            \item Some students' paper assignments
            \item Survey from 91 respondents (target 100 people)
            \item Data is avaiable at: \url{https://zenodo.org/records/8116799}
            \item Could be an illustration of remote working problems
            \item Useful to the basis for implementing \textbf{Enterprise Architecture} in companies to overcome \textbf{remote working} problems
            \item Published paper: \url{https://ieeexplore.ieee.org/abstract/document/10327330}
        \end{enumerate}
    \end{frame}

    \begin{frame}
        \frametitle{Result: Benefits of Remote Working}
        \framesubtitle{\hspace{1cm}}
        \begin{enumerate}
            \item \textbf{Cost-effective}. Reduce transportation costs for employees. For companies, it saves operational costs.
            \item \textbf{Save time}. The time usually used for commuting is used for other things.
            \item \textbf{Work-life balance}. The remaining time is used for family, hobbies, etc. so you are mentally healthier.
            \item \textbf{More productive}. The remaining time is used to do other tasks. Apart from that, you can choose more time and place so that work results are more optimal and reduce distractions that are usually found in the office.
        \end{enumerate}
    \end{frame}

    \begin{frame}
        \frametitle{Results: Challenges in Remote Working}
        \framesubtitle{\hspace{1cm}}
        \begin{enumerate}
            \item \textbf{Nature of Work}. Not all jobs can naturally be done remotely, for example factory employees, research in laboratories, etc.
            \item \textbf{Technical issues}. Slow internet connection, technical problems with hardware/software, infrastructure, current applications are inadequate.
            \item \textbf{Requires investment/expenses}. Need funds to purchase data packages, internet connection, hardware/software.
            \item \textbf{Etiquette Issues}. Contacts or assignments outside of working hours, rude emails, etc.
            \item \textbf{Interference from the environment}. Conditions at home are not conducive to work.

        \end{enumerate}
    \end{frame}

    \begin{frame}
        \frametitle{Results: Challenges in Remote Working (2)}
        \framesubtitle{\hspace{1cm}}
        \begin{enumerate}

            \item \textbf{Motivational issues}. Employees are not motivated if they are not supervised.
            \item \textbf{Trust issues}. Superiors do not believe that subordinates can work independently.
            \item \textbf{Communication problems}. Problems with communication habits/quality that are different from direct communication.
            \item \textbf{Problems of difficulty socializing, working together}. Difficulty building team work, socializing and networking.
        \end{enumerate}
    \end{frame}

    \begin{frame}
        \frametitle{Solutions to the Challenges}
        % \framesubtitle{\hspace{1cm}}
        \begin{enumerate}
            \item \textbf{Technical solutions}. Increasing internet connection, procurement of hardware and software, rent/buy/build it yourself, maintenance of hardware/software.
            \item \textbf{Human resources}.
            \begin{enumerate}
                \item Increased non-technical abilities, for example etiquette, independence, motivation, communication, confidence
                \item Increasing technical capabilities, for example software usage training for users, IT team for trouble shooting.
            \end{enumerate}
            \item \textbf{Business processes, governance}. Changes to business processes to suit remote working. Performance appraisal rules are based on achievement targets, not based on attendance/working hours.
        \end{enumerate}
    \end{frame}

    \begin{frame}
        \frametitle{Solution to the Challenges (2)}
        % \framesubtitle{\hspace{1cm}}
        \begin{enumerate}
            \item \textbf{Funding}. Recalculation and reallocation of funds.
            \item \textbf{Social}. Hold social events regularly to build cooperation, networking and friendship.
            \item \textbf{Infrastructure and support}. Providing space or assistance for remote working.
        \end{enumerate}
    \end{frame}


\section{Class Activities and Assignments}
\begin{frame}
	\centering
	\Huge Class Activities and Assignments
\end{frame}

\begin{frame}{Class Activities and Assignments}
	Based on the findings from the previous session's assignment, determine what \textbf{capabilities} need to be added to your organization to improve its business quality. Next, find at least 30 references, including academic papers, reports, white papers, etc. Describe the capabilities you have identified in the paper.\\ 
	\vspace{10pt}
	For example, if the chosen capability is \textbf{hybrid learning}, the following \textit{slides} show what needs to be prepared to support this capability.
\end{frame}


\section{Examples of Some Capabilities or Solutions for Transitioning to a Hybrid Model for Classroom and Lecture Activities}

\begin{frame}{Learning Platforms and Technology}
	\begin{itemize}
		\item \textbf{Online Learning Platforms:} Use platforms such as Moodle, Blackboard, or Canvas to deliver lecture materials, assignments, and assessments. Ensure these platforms are easily accessible to students across various devices.
		\item \textbf{Virtual Classroom Systems:} Implement virtual classroom systems like Zoom, Microsoft Teams, or Google Meet for live lectures, group discussions, and faculty consultations.
		\item \textbf{Interactive Learning Tools:} Utilize tools such as online quizzes, polls, and digital whiteboards to enhance student engagement.
	\end{itemize}
\end{frame}

\begin{frame}{Teaching and Curriculum}
	\begin{itemize}
		\item \textbf{Hybrid Curriculum Design:} Design a curriculum that supports a hybrid format combining face-to-face and online lectures.
		\item \textbf{Asynchronous Learning Modules:} Create asynchronous modules allowing students to access materials and complete tasks on their own schedule.
		\item \textbf{Periodic Evaluations:} Implement a periodic evaluation system that includes online quizzes, individual assignments, and group projects.
	\end{itemize}
\end{frame}

\begin{frame}{Interaction and Engagement}
	\begin{itemize}
		\item \textbf{Q\&A and Discussion Sessions:} Hold regular online Q\&A and discussion sessions to maintain interaction between faculty and students.
		\item \textbf{Virtual Office Hours:} Provide virtual office hours where students can consult directly with faculty via video call or chat.
		\item \textbf{Discussion Forums:} Use discussion forums within the learning platform to allow students to ask questions and engage in online discussions.
	\end{itemize}
\end{frame}

\begin{frame}{Class Management}
	\begin{itemize}
		\item \textbf{Class Enrollment and Scheduling:} Manage class schedules and enrollment online through an academic management system.
		\item \textbf{Attendance Records:} Use digital attendance systems to monitor student participation in online lectures, including integration with learning platforms for automatic attendance tracking.
		\item \textbf{Material Access:} Ensure all lecture materials are accessible to students via the learning platform.
	\end{itemize}
\end{frame}

\begin{frame}{Assessment and Feedback}
	\begin{itemize}
		\item \textbf{Online Exams:} Use online exam tools to administer secure and well-managed tests.
		\item \textbf{Ongoing Feedback:} Provide timely and constructive feedback through the learning platform or email.
		\item \textbf{Authentic Assessment:} Implement authentic assessments relevant to real-world situations, such as case-based projects or virtual field studies.
	\end{itemize}
\end{frame}

\begin{frame}{Technology Support and Training}
	\begin{itemize}
		\item \textbf{Faculty Training:} Provide training to faculty on the use of online learning technology and effective teaching strategies for hybrid formats.
		\item \textbf{IT Support:} Offer responsive IT support to address technical issues faced by students and faculty during online sessions.
		\item \textbf{Guides and Tutorials:} Create clear guides and tutorials for students on how to use the learning platform.
	\end{itemize}
\end{frame}

\begin{frame}{Student Wellbeing}
	\begin{itemize}
		\item \textbf{Work-Life Balance:} Encourage students to maintain a balance between their studies and personal life by providing flexibility in schedules.
		\item \textbf{Mental Health Support:} Provide online mental health support services, including counseling.
		\item \textbf{Virtual Social Activities:} Organize virtual social activities to maintain student connections.
	\end{itemize}
\end{frame}

\begin{frame}{Data and Infrastructure}
	\begin{itemize}
		\item \textbf{Data Management:} Use cloud-based data management systems to securely store and manage academic and administrative data.
		\item \textbf{Data Security:} Implement strict data security protocols to protect sensitive information.
		\item \textbf{Infrastructure Connectivity:} Ensure IT infrastructure supports stable and fast connectivity for online learning.
	\end{itemize}
\end{frame}

\begin{frame}{Business Processes Related to Learning}
	\begin{itemize}
		\item \textbf{Process Automation:} Automate administrative processes such as course registration, grading, and academic reporting.
		\item \textbf{System Integration:} Integrate online learning systems with academic and financial management systems.
		\item \textbf{New Student Onboarding Process:} Design an online onboarding process for new students, including orientation and training on using the learning platform.
	\end{itemize}
\end{frame}



%    \begin{frame}
%        \frametitle{Graduation Requirements}
%        % \framesubtitle{\hspace{1cm}}
%        \begin{enumerate}
%            \item Final Assignment (project).
%            \item Publication once in an international journal indexed by Scopus
%            \item Publication 2 times in national journal indexed by SINTA (national publication index platform)
%            \item Publication 2 times in international conference proceedings indexed by Scopus (usually IEEE or ACM conferences)
%        \end{enumerate}
%    \end{frame}

%    \begin{frame}
%        \frametitle{1st Trimester Assignment}
%        % \framesubtitle{\hspace{1cm}}
%        \begin{enumerate}
%            \item Write a \textbf{paper} related to \textbf{the recent trends in Enterprise Architecture specific to a domain, e.g., health, education, goverment, etc.}.
%            \item Choose a domain that you are familiar with, probably related to your job.
%            \item Choose one national (Sinta 1-4) or international journal or academic conferences (indexed by Scopus).
%            \item Write your findings using the template provided by the journal or conference.
%        \end{enumerate}
%    \end{frame}

%    \begin{frame}
%        \frametitle{Assignment 1 Trimester (2)}
%        % \framesubtitle{\hspace{1cm}}
%        \begin{enumerate}
%            \item Apart from writing papers, \textbf{students are also expected to make presentations} related to the assigned topic as discussion material in class.
%        \end{enumerate}
%    \end{frame}

%    \begin{frame}
%        \frametitle{Value}
%        % \framesubtitle{\hspace{1cm}}
%        \begin{enumerate}
%            \item Paper 80\%
%            \item Presentation \%20
%        \end{enumerate}
%    \end{frame}
%
%    \begin{frame}
%        \frametitle{Other Research Topics}
%        \framesubtitle{\hspace{1cm}}
%        \begin{enumerate}
%%            \item Most \textbf{Enterprise Architecture} focuses on how IT/IS can support the business and this is characterized by explicit mention of IT/IS components in the framework.
%%            \item Then what about other aspects, such as finance, human resources, legal, etc.?
%            \item For example: how does \textbf{TOGAF} address the \textbf{financial} aspects of \textbf{Enterprise Architecture}? Are there any disadvantages or advantages? etc.
%            \item The \textbf{financial} aspect can be replaced with other aspects of the company, such as HR, legal.
%        \end{enumerate}
%    \end{frame}

\end{document}