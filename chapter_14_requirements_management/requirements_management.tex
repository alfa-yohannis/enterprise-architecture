\documentclass{beamer}

\usetheme{Madrid}
\usecolortheme{seahorse}

\title{Manajemen Persyaratan dalam Metode Pengembangan Arsitektur TOGAF}
\author{Alfa Yohannis}
\date{\today}

\begin{document}

\frame{\titlepage}

\begin{frame}
\frametitle{Tujuan Manajemen Persyaratan}
\begin{itemize}
\item Mengidentifikasi, memahami, dan merekam persyaratan bisnis.
\item Mengatur dan memprioritaskan daftar persyaratan.
\item Menyediakan input untuk perencanaan dan pengembangan arsitektur.
\end{itemize}
\end{frame}

\begin{frame}
\frametitle{Input Fase Manajemen Persyaratan}
\begin{itemize}
\item Output dari fase-fase ADM lainnya.
\item Stakeholder Map.
\item Daftar persyaratan bisnis dan teknis.
\end{itemize}
\end{frame}

\begin{frame}
\frametitle{Langkah-langkah dalam Fase Manajemen Persyaratan}
\begin{itemize}
\item Mengumpulkan persyaratan dari stakeholder.
\item Menganalisis dan mengklarifikasi persyaratan.
\item Mengkategorikan dan mengorganisir persyaratan.
\item Mengidentifikasi konflik dan duplikasi dalam persyaratan.
\item Membuat dan memelihara Repository Persyaratan.
\end{itemize}
\end{frame}

\begin{frame}
\frametitle{Output Fase Manajemen Persyaratan}
\begin{itemize}
\item Repository Persyaratan yang diperbarui.
\item Daftar persyaratan yang telah diprioritaskan.
\item Laporan status dan pembaruan tentang manajemen persyaratan.
\end{itemize}
\end{frame}

\begin{frame}
\frametitle{Contoh Fase Manajemen Persyaratan}
\begin{itemize}
\item Contoh penggunaan fase Manajemen Persyaratan adalah ketika perusahaan teknologi memutuskan untuk membangun sistem manajemen proyek baru.
\item Pengumpulan persyaratan dapat melibatkan wawancara dengan manajer proyek, pengguna sistem saat ini, dan pihak lain yang berkepentingan.
\item Hasilnya adalah daftar fitur dan fungsionalitas yang harus ada dalam sistem baru.
\end{itemize}
\end{frame}

\begin{frame}
\frametitle{Ringkasan Fase Manajemen Persyaratan}
\begin{itemize}
\item Fase Manajemen Persyaratan adalah tentang memahami apa yang dibutuhkan oleh bisnis dan bagaimana teknologi dapat mendukung kebutuhan tersebut.
\item Fase ini melibatkan pengumpulan, analisis, dan pengorganisasian persyaratan.
\item Output utama adalah daftar persyaratan yang telah diprioritaskan yang akan digunakan untuk mengarahkan perencanaan dan pengembangan arsitektur.
\end{itemize}
\end{frame}

\end{document}
