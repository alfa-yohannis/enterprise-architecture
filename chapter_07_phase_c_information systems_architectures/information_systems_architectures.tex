\documentclass{beamer}

\usepackage[utf8]{inputenc}
\usetheme{Madrid}
\usecolortheme{seahorse}

\title{Fase C: Arsitektur Sistem Informasi dalam Metode Pengembangan Arsitektur TOGAF}
\author{Prof. Alfa Yohannis}
\date{\today}

\begin{document}

\frame{\titlepage}

\begin{frame}
\frametitle{Tujuan}
\begin{itemize}
\item Mendefinisikan arsitektur data dan aplikasi yang diperlukan untuk mendukung bisnis
\item Mengintegrasikan data dan aplikasi dari berbagai sumber dalam organisasi
\end{itemize}
\end{frame}

\begin{frame}
\frametitle{Input}
\begin{itemize}
\item Arsitektur Bisnis
\item Pernyataan Kebutuhan Arsitektur
\end{itemize}
\end{frame}

\begin{frame}
\frametitle{Langkah-langkah}
\begin{enumerate}
\item Mengembangkan Baseline Arsitektur Data dan Aplikasi
\item Mengembangkan Arsitektur Sasaran Data dan Aplikasi
\item Melakukan Gap Analysis
\item Mendefinisikan peta jalan data dan aplikasi
\end{enumerate}
\end{frame}

\begin{frame}
\frametitle{Output}
\begin{itemize}
\item Baseline Arsitektur Data dan Aplikasi
\item Arsitektur Sasaran Data dan Aplikasi
\item Peta jalan data dan aplikasi
\end{itemize}
\end{frame}

\begin{frame}
\frametitle{Contoh}
\begin{itemize}
\item Contoh Baseline Arsitektur Data: Diagram arsitektur data saat ini
\item Contoh Baseline Arsitektur Aplikasi: Diagram arsitektur aplikasi saat ini
\item Contoh Arsitektur Sasaran Data: Diagram arsitektur data masa depan
\item Contoh Arsitektur Sasaran Aplikasi: Diagram arsitektur aplikasi masa depan
\end{itemize}
\end{frame}

\begin{frame}
\frametitle{Ringkasan}
Fase Arsitektur Sistem Informasi mencakup pembuatan Arsitektur Data dan Aplikasi. Fase ini memungkinkan kita untuk mendefinisikan dan memvalidasi struktur data dan aplikasi yang diperlukan untuk mendukung arsitektur bisnis. Fase ini penting untuk memastikan bahwa data, aplikasi dan integrasinya sesuai dengan kebutuhan bisnis.
\end{frame}

\end{document}
