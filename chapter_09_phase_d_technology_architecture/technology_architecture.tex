\documentclass{beamer}

\usepackage[utf8]{inputenc}
\usetheme{Madrid}
\usecolortheme{seahorse}

\title{Fase D: Arsitektur Teknologi dalam Metode Pengembangan Arsitektur TOGAF}
\author{Alfa Yohannis}
\date{\today}

\begin{document}

\frame{\titlepage}

\begin{frame}
\frametitle{Tujuan}
\begin{itemize}
\item Mendefinisikan arsitektur teknologi yang diperlukan untuk mendukung arsitektur bisnis dan arsitektur data dan aplikasi
\item Menjamin bahwa arsitektur ini responsif terhadap perubahan teknologi
\end{itemize}
\end{frame}

\begin{frame}
\frametitle{Input}
\begin{itemize}
\item Pernyataan arsitektur
\item Pernyataan kebutuhan
\item Arsitektur bisnis, arsitektur data, dan arsitektur aplikasi
\end{itemize}
\end{frame}

\begin{frame}
\frametitle{Langkah-langkah}
\begin{enumerate}
\item Mengumpulkan informasi teknologi
\item Mendefinisikan arsitektur teknologi
\item Memvalidasi dan menyetujui arsitektur teknologi
\end{enumerate}
\end{frame}

\begin{frame}
\frametitle{Output}
\begin{itemize}
\item Arsitektur teknologi yang didefinisikan
\item Pernyataan yang divalidasi dan disetujui
\end{itemize}
\end{frame}

\begin{frame}
\frametitle{Contoh}
\begin{itemize}
\item Contoh arsitektur teknologi: Implementasi cloud
\item Contoh validasi: Review dan persetujuan oleh pemangku kepentingan
\end{itemize}
\end{frame}

\begin{frame}
\frametitle{Ringkasan}
Fase Arsitektur Teknologi memungkinkan kita untuk mendefinisikan dan memvalidasi arsitektur teknologi yang diperlukan untuk mendukung arsitektur bisnis dan arsitektur data dan aplikasi. Fase ini penting untuk menjamin bahwa arsitektur ini dapat menanggapi perubahan teknologi.
\end{frame}

\end{document}
