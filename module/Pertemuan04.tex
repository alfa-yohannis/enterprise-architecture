\chapter{Fase Awal (\textit{Preliminary})}

\section{Tujuan}
Tujuan dari Fase Awal menentukan spesifikasi kapabilitas arsitektur yang diinginkan (\textit{Desired}) oleh organisasi serta mendefinisikan kapabilitas saat ini (\textit{Current}) dan apa saja yang diperlukan (\textit{Required}) untuk mencapai kapabilitas arsitektur yang dikehendaki. Pada tahap ini, kapabilitas arsitektur dan kebutuhan-kebutuhannya masih bersifat \textit{high-level} dan akan diperinci di fase-fase berikutnya. Tujuan ini dapat dibagi menjadi dua bagian utama:

\subsection{Spesifikasi Kapabilitas Arsitektur yang Diinginkan}
\begin{enumerate}
	\item Meninjau konteks organisasi untuk melakukan arsitektur enterprise.
	\item Mengidentifikasi cakupan bagian dari organisasi enterprise yang terpengaruh oleh Kapabilitas Arsitektur.
	\item Mengidentifikasi kerangka kerja, metode, dan proses yang terkait dengan Kapabilitas Arsitektur.
	\item Menetapkan target Kematangan Kapabilitas.
\end{enumerate}

\subsubsection{Contoh:}

\begin{enumerate}
	\item \textbf{Meninjau konteks organisasi untuk melakukan arsitektur enterprise}
	
	Sebagai langkah awal, universitas perlu meninjau strategi pendidikan yang telah ada dan melakukan evaluasi terhadap model pembelajaran saat ini. Misalnya, universitas mengevaluasi penggunaan ruang kelas fisik, platform e-learning yang sudah tersedia (seperti Learning Management System/LMS), serta kesiapan dosen dan mahasiswa dalam menggunakan teknologi digital. Hal ini mencakup peninjauan fasilitas IT yang ada, kesiapan jaringan internet di kampus, dan kapabilitas sumber daya manusia dalam mendukung pembelajaran hybrid.
	
	\item \textbf{Mengidentifikasi cakupan bagian dari organisasi enterprise yang terpengaruh oleh Kapabilitas Arsitektur}
	
	Mengidentifikasi bagian-bagian dari universitas yang akan terdampak oleh penerapan hybrid teaching and learning. Ini mencakup program studi, dosen, mahasiswa, tim IT, staf administrasi, serta bagian pengelola fasilitas. Contoh cakupan yang terpengaruh adalah perubahan pada metode pengajaran dosen, pelatihan penggunaan LMS untuk dosen dan mahasiswa, serta penyesuaian fasilitas kampus seperti ruang kelas yang dilengkapi dengan perangkat video conference.
	
	\item \textbf{Mengidentifikasi kerangka kerja, metode, dan proses yang terkait dengan Kapabilitas Arsitektur}
	
	Menentukan kerangka kerja dan metode yang akan digunakan untuk mengimplementasikan pembelajaran hybrid. Sebagai contoh, universitas dapat memilih kerangka kerja TOGAF untuk mengelola arsitektur enterprise, ITIL untuk mengelola layanan IT, dan metode ADDIE (Analysis, Design, Development, Implementation, and Evaluation) untuk mengembangkan kurikulum hybrid. Proses-proses yang terkait dapat mencakup pembuatan pedoman penggunaan LMS, kebijakan kehadiran online, serta proses penilaian yang mencakup tugas daring dan luring.
	
	\item \textbf{Menetapkan target Kematangan Kapabilitas}
	
	Menetapkan target kematangan kapabilitas untuk mengukur kesiapan universitas dalam menerapkan pembelajaran hybrid. Sebagai contoh, target kematangan dapat mencakup tingkat integrasi LMS dengan sistem akademik, tingkat adopsi teknologi oleh dosen dan mahasiswa, serta kesiapan infrastruktur IT. Pada tahap awal, target kematangan bisa berupa 50\% dosen menggunakan LMS untuk memberikan materi kuliah, 70\% mahasiswa mampu mengakses dan menggunakan platform hybrid, serta adanya integrasi penuh antara sistem informasi akademik dengan LMS.
\end{enumerate}

\subsection{Mendefinisikan Kapabilitas Saat Ini dan Kebutuhan Apa Saja yang Diperlukan}
\begin{enumerate}
	\item Mendefinisikan Model Organisasi (orang, jabatan, peran, wewenang, departemen terkait, dll.) untuk Arsitektur Enterprise.
	\item Mendefinisikan dan menetapkan proses dan sumber daya secara rinci untuk tata kelola arsitektur.
	\item Memilih dan menerapkan alat yang mendukung Kapabilitas Arsitektur (contoh: Porter’s 5 forces, SWOT, analisis pemangku kepentingan, dll.).
	\item Mendefinisikan prinsip-prinsip arsitektur.
\end{enumerate}

\subsubsection{Contoh:}

\begin{enumerate}
	\item \textbf{Mendefinisikan Model Organisasi (orang, jabatan, peran, wewenang, departemen terkait, dll.) untuk Arsitektur Enterprise}
	
	Dalam konteks hybrid teaching and learning, model organisasi yang perlu didefinisikan melibatkan berbagai peran dan jabatan, seperti:
	\begin{itemize}
		\item \textbf{Dosen dan Staf Pengajar}: Bertanggung jawab dalam menyiapkan materi yang dapat diakses baik secara daring maupun luring.
		\item \textbf{Koordinator Program Studi}: Memastikan bahwa kurikulum dapat diadaptasi ke format hybrid.
		\item \textbf{Tim IT}: Mengelola infrastruktur IT, seperti LMS dan platform video conference, serta memastikan dukungan teknis bagi dosen dan mahasiswa.
		\item \textbf{Departemen Pengembangan Sumber Daya Manusia}: Mengatur pelatihan bagi dosen dan staf pengajar terkait penggunaan teknologi dalam pembelajaran hybrid.
	\end{itemize}
	
	Setiap jabatan perlu memiliki peran dan tanggung jawab yang jelas, seperti dosen yang bertanggung jawab mengunggah materi di LMS setiap minggu atau Tim IT yang melakukan pengecekan infrastruktur daring sebelum setiap semester dimulai.
	
	\item \textbf{Mendefinisikan dan menetapkan proses dan sumber daya secara rinci untuk tata kelola arsitektur}
	
	Proses tata kelola arsitektur untuk pembelajaran hybrid dapat meliputi:
	\begin{itemize}
		\item Prosedur peninjauan kurikulum yang dilakukan oleh koordinator program studi dan dosen untuk memastikan kesesuaian dengan metode pembelajaran hybrid.
		\item Proses eskalasi teknis bagi masalah yang dihadapi oleh dosen atau mahasiswa dalam menggunakan LMS.
		\item Alur kerja pengelolaan konten digital, seperti pengaturan hak akses, upload materi kuliah, dan pengaturan integrasi dengan platform pihak ketiga seperti Zoom atau Google Meet.
	\end{itemize}
	
	Sumber daya yang diperlukan bisa mencakup alat bantu seperti software manajemen kursus, perangkat keras untuk ruang kelas hybrid (kamera, mikrofon, proyektor), serta platform digital untuk asesmen dan evaluasi.
	
	\item \textbf{Memilih dan menerapkan alat yang mendukung Kapabilitas Arsitektur}
	
	Alat yang dipilih harus mendukung proses hybrid teaching and learning. Beberapa contoh alat yang dapat diterapkan adalah:
	\begin{itemize}
		\item \textbf{Learning Management System (LMS)}: Platform seperti Moodle atau Google Classroom untuk mengelola materi kuliah, tugas, dan interaksi antara dosen dan mahasiswa.
		\item \textbf{Alat Video Conference}: Zoom atau Microsoft Teams untuk mendukung kuliah sinkron secara daring.
		\item \textbf{Analisis SWOT dan Pemetaan Pemangku Kepentingan}: Digunakan untuk memahami kekuatan, kelemahan, peluang, dan ancaman dalam penerapan hybrid teaching, serta untuk mengidentifikasi pemangku kepentingan utama yang terlibat.
		\item \textbf{Perangkat Asesmen Daring}: Google Forms atau platform khusus asesmen untuk mengelola ujian, kuis, dan penilaian mahasiswa.
	\end{itemize}
	
	Implementasi alat ini harus sesuai dengan kebutuhan kapabilitas arsitektur, seperti integrasi antara LMS dengan sistem akademik universitas untuk kemudahan pengelolaan nilai dan absensi.
	
	\item \textbf{Mendefinisikan prinsip-prinsip arsitektur}
	
	Prinsip-prinsip arsitektur yang didefinisikan untuk hybrid teaching and learning dapat mencakup:
	\begin{itemize}
		\item \textbf{Prinsip Keterbukaan Akses}: Mahasiswa harus dapat mengakses materi kuliah dan aktivitas pembelajaran dari mana saja dan kapan saja. Hal ini memerlukan desain platform yang user-friendly serta dukungan untuk perangkat mobile.
		\item \textbf{Prinsip Interaksi Sinkron dan Asinkron}: Pembelajaran harus mendukung interaksi secara langsung (live sessions) maupun tidak langsung (materi rekaman dan diskusi forum).
		\item \textbf{Prinsip Kualitas Materi}: Materi pembelajaran harus dapat memenuhi standar akademik, terstruktur, dan mudah dipahami baik dalam format daring maupun luring.
		\item \textbf{Prinsip Keamanan Data}: Menjamin privasi dan keamanan data mahasiswa, dosen, dan staf, terutama ketika menggunakan platform daring dan menyimpan data akademik.
	\end{itemize}
	
	Prinsip-prinsip ini akan menjadi landasan bagi keputusan strategis dan operasional dalam penerapan hybrid teaching and learning, memastikan bahwa setiap langkah yang diambil selaras dengan tujuan arsitektur keseluruhan.
\end{enumerate}



\section{Input}
Input untuk Fase Awal mencakup beberapa dokumen dan kerangka kerja penting:

\begin{itemize}
	\item Kerangka Kerja TOGAF
	\item Kerangka Arsitektur Lainnya
	\item Rencana bisnis, strategi bisnis, dan strategi IT
	\item Prinsip bisnis, tujuan bisnis, dan pendorong bisnis
	\item Kerangka kerja tata kelola dan hukum, termasuk strategi Tata Kelola Arsitektur
	\item Model Organisasi yang ada untuk Arsitektur Enterprise
	\item Kerangka Kerja Arsitektur yang sudah ada, jika ada
\end{itemize}

\subsubsection{Contoh Input yang Digunakan untuk Mendukung Hybrid Teaching and Learning}

\begin{itemize}
	\item \textbf{Kerangka Kerja TOGAF}
	
	TOGAF (The Open Group Architecture Framework) dapat digunakan sebagai panduan untuk mengembangkan kapabilitas hybrid teaching and learning di universitas. Misalnya, universitas dapat menggunakan tahapan dalam TOGAF Architecture Development Method (ADM) untuk mendefinisikan visi arsitektur hybrid, mengidentifikasi kapabilitas yang diperlukan, serta merencanakan implementasi dan pengelolaannya. Kerangka kerja ini membantu dalam merumuskan strategi, merancang model arsitektur, serta memastikan bahwa semua pemangku kepentingan memiliki pemahaman yang sama tentang arah pengembangan kapabilitas hybrid.
	
	\item \textbf{Kerangka Arsitektur Lainnya}
	
	Selain TOGAF, universitas dapat memanfaatkan kerangka kerja lain yang sesuai dengan kebutuhan, seperti Zachman Framework atau Gartner's Enterprise Architecture Framework. Misalnya, Zachman Framework dapat digunakan untuk memetakan komponen-komponen hybrid teaching and learning dalam aspek `What, How, Where, Who, When,` dan `Why`. Ini memungkinkan universitas untuk memahami hubungan antara proses pembelajaran, teknologi yang digunakan, serta peran pemangku kepentingan secara lebih komprehensif.
	
	\item \textbf{Rencana bisnis, strategi bisnis, dan strategi IT}
	
	Rencana bisnis universitas harus sejalan dengan strategi hybrid teaching and learning. Sebagai contoh, jika visi universitas adalah untuk meningkatkan aksesibilitas pendidikan bagi mahasiswa yang berada di luar kota, maka strategi hybrid harus mendukung hal ini dengan menyediakan platform yang dapat diakses dari berbagai lokasi dan perangkat. Selain itu, strategi IT harus mencakup investasi pada infrastruktur seperti cloud computing, sistem informasi akademik, serta platform e-learning yang terintegrasi.
	
	\item \textbf{Prinsip bisnis, tujuan bisnis, dan pendorong bisnis}
	
	Prinsip bisnis universitas yang mendukung hybrid teaching and learning dapat mencakup prinsip `Student-Centric`, di mana setiap keputusan tentang pembelajaran diambil dengan mempertimbangkan pengalaman dan kenyamanan mahasiswa. Tujuan bisnis seperti peningkatan jumlah mahasiswa dari wilayah terpencil atau peningkatan retensi mahasiswa juga menjadi pendorong utama bagi kapabilitas hybrid. Dengan prinsip-prinsip ini, universitas dapat memastikan bahwa setiap perubahan arsitektur sejalan dengan misi dan nilai bisnis.
	
	\item \textbf{Kerangka kerja tata kelola dan hukum, termasuk strategi Tata Kelola Arsitektur}
	
	Kerangka kerja tata kelola seperti COBIT (Control Objectives for Information and Related Technologies) dapat digunakan untuk memastikan bahwa semua inisiatif dalam hybrid teaching and learning dikelola dengan baik, serta memenuhi regulasi dan hukum yang berlaku. Sebagai contoh, universitas harus mematuhi undang-undang perlindungan data pribadi dalam pengelolaan data akademik mahasiswa. Strategi tata kelola arsitektur juga harus mencakup penanganan risiko teknologi serta pengelolaan perubahan dalam kapabilitas hybrid secara berkelanjutan.
	
	\item \textbf{Model Organisasi yang ada untuk Arsitektur Enterprise}
	
	Model organisasi yang sudah ada dapat digunakan sebagai acuan untuk mengembangkan kapabilitas hybrid. Misalnya, jika universitas sudah memiliki struktur departemen IT dan tim akademik, maka dapat ditambahkan peran baru seperti `Hybrid Learning Coordinator` yang bertugas untuk memastikan implementasi dan pengawasan pembelajaran hybrid. Perubahan-perubahan ini perlu dikomunikasikan dengan baik ke seluruh pemangku kepentingan agar ada pemahaman yang jelas tentang tanggung jawab baru.
	
	\item \textbf{Kerangka Kerja Arsitektur yang sudah ada, jika ada}
	
	Jika universitas telah memiliki kerangka kerja arsitektur sebelumnya, misalnya untuk mendukung pembelajaran daring, maka kerangka kerja tersebut dapat disesuaikan dan diperluas untuk mendukung pembelajaran hybrid. Hal ini dapat mencakup penambahan komponen baru pada arsitektur teknologi, seperti integrasi antara ruang kelas fisik dan virtual, serta pengelolaan data interaksi mahasiswa yang mencakup kehadiran dan partisipasi baik secara daring maupun luring.
\end{itemize}

\section{Langkah-Langkah}
Langkah-langkah yang terlibat dalam Fase Awal adalah sebagai berikut:

\begin{enumerate}
	\item Menyusun cakupan organisasi perusahaan yang terkait/terpengaruh.
	\item Memastikan dukungan untuk kapabilitas kerangka kerja arsitektur dan tata kelola yang ditingkatkan.
	\item Mendefinisikan dan membentuk tim arsitektur enterprise dan organisasi.
	\item Mengidentifikasi dan mendefinisikan prinsip-prinsip arsitektur.
	\item Menyesuaikan kerangka kerja TOGAF, \textit{if applicable}, dengan kerangka arsitektur lain yang digunakan.
	\item Menerapkan alat-alat arsitektur.
\end{enumerate}

\subsubsection{Contoh Langkah-langkah dalam Mengembangkan Kapabilitas Hybrid Teaching and Learning}

\begin{enumerate}
	\item \textbf{Menyusun cakupan organisasi perusahaan yang terkait/terpengaruh}
	
	Dalam konteks hybrid teaching and learning di universitas, cakupan organisasi yang terkait meliputi berbagai departemen, fakultas, serta unit pendukung yang memiliki peran penting dalam implementasi pembelajaran hybrid. Misalnya:
	\begin{itemize}
		\item \textbf{Departemen Teknologi Informasi}: Bertanggung jawab terhadap pengelolaan Learning Management System (LMS), infrastruktur jaringan, serta integrasi platform daring seperti Zoom atau Microsoft Teams.
		\item \textbf{Fakultas dan Program Studi}: Terlibat dalam penyusunan kurikulum hybrid, pelatihan dosen, serta pemantauan kualitas pembelajaran.
		\item \textbf{Lembaga Pengembangan Sumber Daya Manusia}: Bertanggung jawab memberikan pelatihan dan workshop bagi dosen terkait metode pengajaran hybrid.
		\item \textbf{Bagian Administrasi dan Keuangan}: Mengatur pengadaan peralatan yang mendukung, seperti perangkat keras untuk ruang kelas hybrid.
	\end{itemize}
	
	Dengan menetapkan cakupan organisasi yang terlibat, universitas dapat memastikan koordinasi dan kolaborasi yang lebih baik antar-departemen dalam penerapan kapabilitas hybrid.
	
	\item \textbf{Memastikan dukungan untuk kapabilitas kerangka kerja arsitektur dan tata kelola yang ditingkatkan}
	
	Universitas harus memastikan bahwa setiap pemangku kepentingan mendukung kapabilitas kerangka kerja arsitektur yang diperbarui. Misalnya, dukungan ini dapat berbentuk alokasi anggaran khusus untuk implementasi sistem pembelajaran hybrid, penyusunan SOP (Standar Operasional Prosedur) baru yang mengakomodasi format hybrid, serta kesediaan manajemen untuk mengadopsi prinsip-prinsip tata kelola arsitektur dalam setiap keputusan strategis terkait pendidikan.
	
	Komunikasi dengan pemangku kepentingan juga menjadi kunci, termasuk dengan dekan, rektorat, dosen, serta mahasiswa, agar setiap pihak memahami manfaat dan peran mereka dalam mendukung kapabilitas hybrid.
	
	\item \textbf{Mendefinisikan dan membentuk tim arsitektur enterprise dan organisasi}
	
	Pembentukan tim arsitektur enterprise khusus untuk hybrid teaching and learning dapat terdiri dari:
	\begin{itemize}
		\item \textbf{Chief Information Officer (CIO)}: Memimpin pengembangan arsitektur IT yang mendukung kapabilitas hybrid.
		\item \textbf{Hybrid Learning Coordinator}: Menjembatani antara kebutuhan akademik dan teknis dalam penerapan pembelajaran hybrid.
		\item \textbf{IT Support Specialist}: Menyediakan dukungan teknis harian, termasuk troubleshooting dan pelatihan penggunaan teknologi.
		\item \textbf{Academic Development Specialist}: Mengawasi penyusunan materi pembelajaran yang dapat diakses baik secara daring maupun luring.
	\end{itemize}
	
	Tim ini bertanggung jawab merencanakan, mengimplementasikan, serta mengevaluasi kapabilitas hybrid untuk memastikan penerapan yang efektif dan efisien di seluruh universitas.
	
	\item \textbf{Mengidentifikasi dan mendefinisikan prinsip-prinsip arsitektur}
	
	Prinsip-prinsip arsitektur yang didefinisikan harus mencakup aspek-aspek penting dari hybrid teaching and learning, seperti:
	\begin{itemize}
		\item \textbf{Konektivitas yang Konsisten}: Menjamin akses internet dan jaringan di seluruh kampus serta memungkinkan mahasiswa untuk bergabung dari lokasi mana pun.
		\item \textbf{Fleksibilitas dalam Pengajaran dan Pembelajaran}: Memastikan bahwa materi kuliah dapat diakses secara sinkron dan asinkron, memberikan fleksibilitas waktu kepada mahasiswa.
		\item \textbf{Keamanan dan Privasi}: Melindungi data pribadi dosen dan mahasiswa, terutama dalam penggunaan platform pembelajaran daring.
	\end{itemize}
	
	Prinsip-prinsip ini akan menjadi panduan bagi setiap keputusan yang diambil dalam proses implementasi kapabilitas hybrid di universitas.
	
	\item \textbf{Menyesuaikan kerangka kerja TOGAF, \textit{if applicable}, dengan kerangka arsitektur lain yang digunakan}
	
	Menyesuaikan kerangka kerja TOGAF untuk mendukung hybrid teaching and learning dapat melibatkan penambahan tahap khusus pada ADM (Architecture Development Method) untuk menangani integrasi antara platform pembelajaran daring dengan sistem informasi akademik yang ada di universitas. Misalnya, universitas dapat menambahkan `Phase G: Implementation Governance` khusus untuk memastikan bahwa setiap perubahan sistem dilakukan sesuai rencana dan standar kualitas yang telah ditetapkan.
	
	Jika universitas juga menggunakan kerangka kerja lain seperti ITIL (Information Technology Infrastructure Library), maka penyesuaian dapat dilakukan pada aspek pengelolaan layanan IT yang mendukung kapabilitas hybrid.
	
	\item \textbf{Menerapkan alat-alat arsitektur}
	
	Alat-alat arsitektur yang diterapkan harus mampu mendukung seluruh proses dan kapabilitas hybrid. Contoh alat yang dapat diterapkan di universitas untuk kapabilitas ini antara lain:
	\begin{itemize}
		\item \textbf{Enterprise Architecture Tools}: Seperti Sparx Enterprise Architect untuk merancang arsitektur hybrid secara komprehensif, mencakup integrasi platform, model data, serta diagram proses.
		\item \textbf{Project Management Tools}: Seperti Jira atau Trello untuk mengelola tugas-tugas terkait implementasi kapabilitas hybrid.
		\item \textbf{Change Management Tools}: Seperti ServiceNow untuk menangani permintaan perubahan yang berkaitan dengan sistem atau teknologi hybrid yang digunakan.
	\end{itemize}
	
	Penerapan alat-alat ini akan membantu universitas dalam merencanakan, melacak, serta mengevaluasi setiap tahap implementasi kapabilitas hybrid, memastikan bahwa semua berjalan sesuai dengan visi dan misi yang telah ditentukan.
\end{enumerate}

\section{Output}
Output dari Fase Awal mencakup item-item berikut:

\begin{itemize}
	\item Model Organisasi untuk Arsitektur Enterprise
	\item Kerangka Kerja Arsitektur yang disesuaikan/diadaptasi, termasuk prinsip-prinsip arsitektur
	\item Repositori Arsitektur Awal
	\item Penegasan kembali atau referensi pada prinsip bisnis, tujuan bisnis, dan pendorong bisnis
	\item Permintaan untuk pekerjaan arsitektur (dokumen resmi)
	\item Kerangka Kerja Tata Kelola Arsitektur
\end{itemize}

\subsubsection{Contoh Output:}

\begin{itemize}
	\item \textbf{Model Organisasi untuk Arsitektur Enterprise}
	
	Dalam konteks `hybrid teaching and learning`, model organisasi untuk arsitektur enterprise harus mencakup peran dan tanggung jawab yang jelas di seluruh departemen dan unit terkait di universitas. Misalnya:
	\begin{itemize}
		\item Pembentukan unit `Hybrid Learning Office` yang berfungsi sebagai pusat koordinasi untuk seluruh kegiatan yang berkaitan dengan pembelajaran hybrid.
		\item Posisi `Hybrid Learning Coordinator` yang memiliki wewenang untuk mengawasi integrasi antara sistem pembelajaran daring dan luring.
		\item Peran `IT Support Specialist` yang bertugas khusus untuk memastikan infrastruktur teknologi selalu siap mendukung kegiatan belajar-mengajar baik secara daring maupun luring.
		\item `Digital Content Specialist` yang membantu dosen dalam mengembangkan materi ajar interaktif, seperti video pembelajaran, kuis daring, serta modul-modul lainnya.
	\end{itemize}
	
	Dengan model organisasi ini, universitas dapat dengan lebih mudah memastikan setiap elemen kapabilitas hybrid dapat dijalankan secara efisien dan terkoordinasi.
	
	\item \textbf{Kerangka Kerja Arsitektur yang disesuaikan/diadaptasi, termasuk prinsip-prinsip arsitektur}
	
	Kerangka kerja arsitektur yang disesuaikan untuk `hybrid teaching and learning` dapat mencakup prinsip-prinsip berikut:
	\begin{itemize}
		\item \textbf{Prinsip Aksesibilitas}: Materi pembelajaran harus dapat diakses oleh semua mahasiswa, baik yang mengikuti kelas secara fisik di kampus maupun secara daring dari lokasi lain.
		\item \textbf{Prinsip Fleksibilitas Pengajaran}: Dosen diberikan kebebasan untuk merancang metode pembelajaran yang paling sesuai, baik menggunakan pendekatan sinkron maupun asinkron.
		\item \textbf{Prinsip Integrasi Sistem}: Semua sistem yang digunakan (Learning Management System, sistem informasi akademik, serta platform video conference) harus dapat saling terhubung dan berbagi data secara efisien.
	\end{itemize}
	
	Penyesuaian prinsip-prinsip ini akan membantu universitas dalam menciptakan lingkungan pembelajaran hybrid yang kohesif dan terintegrasi.
	
	\item \textbf{Repositori Arsitektur Awal}
	
	Repositori arsitektur awal dalam implementasi `hybrid teaching and learning` dapat berupa:
	\begin{itemize}
		\item \textbf{Dokumentasi Infrastruktur Teknologi}: Berisi informasi tentang server, jaringan, serta perangkat keras dan lunak yang digunakan untuk mendukung pembelajaran hybrid.
		\item \textbf{Blueprint Integrasi Sistem}: Menjelaskan bagaimana berbagai sistem di universitas (LMS, sistem informasi akademik, dan lain-lain) saling terhubung.
		\item \textbf{Model Data Pengguna}: Meliputi informasi yang diperlukan untuk mengelola akses pengguna, baik dosen maupun mahasiswa, pada sistem pembelajaran daring dan luring.
	\end{itemize}
	
	Repositori ini menjadi referensi awal yang dapat dikembangkan seiring dengan berkembangnya kapabilitas hybrid di universitas.
	
	\item \textbf{Penegasan kembali atau referensi pada prinsip bisnis, tujuan bisnis, dan pendorong bisnis}
	
	Implementasi `hybrid teaching and learning` harus sesuai dengan prinsip bisnis universitas, tujuan bisnis, serta pendorong bisnis. Misalnya:
	\begin{itemize}
		\item \textbf{Prinsip Bisnis}: Universitas berkomitmen untuk memberikan akses pendidikan yang setara bagi semua mahasiswa, terlepas dari lokasi mereka.
		\item \textbf{Tujuan Bisnis}: Meningkatkan jumlah pendaftaran mahasiswa internasional melalui penyediaan program studi yang dapat diikuti secara hybrid.
		\item \textbf{Pendorong Bisnis}: Meningkatkan fleksibilitas dalam kurikulum untuk menarik lebih banyak mahasiswa dewasa yang ingin belajar sambil bekerja.
	\end{itemize}
	
	Penegasan ini memastikan bahwa kapabilitas hybrid tidak hanya diterapkan dari sisi teknologi, tetapi juga sejalan dengan misi dan visi strategis universitas.
	
	\item \textbf{Permintaan untuk pekerjaan arsitektur (dokumen resmi)}
	
	Contoh dapat dilihat di Subbab \ref{Request for Architectural Work}.
	
	\item \textbf{Kerangka Kerja Tata Kelola Arsitektur}
	
	Kerangka kerja tata kelola arsitektur untuk `hybrid teaching and learning` dapat mencakup elemen-elemen berikut:
	\begin{itemize}
		\item \textbf{Struktur Kepemimpinan}: Membentuk tim tata kelola yang terdiri dari pimpinan IT, pimpinan akademik, serta perwakilan fakultas untuk mengevaluasi penerapan kapabilitas hybrid secara berkala.
		\item \textbf{Prosedur Evaluasi}: Menetapkan prosedur evaluasi dan audit rutin untuk mengukur efektivitas sistem hybrid, baik dari aspek teknis maupun akademis.
		\item \textbf{Mekanisme Eskalasi}: Menyusun mekanisme eskalasi untuk menyelesaikan permasalahan yang muncul selama penerapan kapabilitas hybrid.
		\item \textbf{Dokumentasi dan Pelaporan}: Menyusun dokumentasi hasil evaluasi serta laporan perkembangan kapabilitas hybrid untuk dilaporkan kepada pemangku kepentingan utama.
	\end{itemize}
	
	Kerangka kerja tata kelola ini akan memastikan bahwa kapabilitas hybrid dapat diimplementasikan dengan standar kualitas yang tinggi dan dapat terus dikembangkan sesuai kebutuhan.
\end{itemize}

\section{Mendefinisikan dan Membentuk Tim dan Organisasi Arsitektur Enterprise}
Langkah ini melibatkan pembentukan tim dan mendefinisikan struktur organisasi arsitektur enterprise. Tugas-tugas yang termasuk:

\begin{itemize}
	\item Menentukan kapabilitas perusahaan dan bisnis yang ada.
	\item Melakukan penilaian kematangan arsitektur/perubahan bisnis.
	\item Mengidentifikasi kesenjangan di area kerja yang ada.
	\item Menetapkan peran dan tanggung jawab utama untuk pengelolaan dan pengawasan kapabilitas arsitektur enterprise.
	\item Menulis permintaan perubahan untuk proyek yang ada (dokumen resmi).
	\item Mendefinisikan batasan dalam pekerjaan arsitektur enterprise.
	\item Meninjau dan menyetujui dengan sponsor dan dewan.
	\item Menilai kebutuhan anggaran.
\end{itemize}

\subsubsection{Contoh:}

Langkah ini melibatkan pembentukan tim dan mendefinisikan struktur organisasi arsitektur enterprise yang akan mengelola kapabilitas `hybrid teaching and learning`. Tugas-tugas yang termasuk dalam langkah ini dapat dijelaskan dengan contoh-contoh nyata berikut:

\begin{itemize}
	\item \textbf{Menentukan kapabilitas perusahaan dan bisnis yang ada}
	
	Dalam konteks `hybrid teaching and learning`, universitas perlu meninjau kapabilitas yang dimiliki saat ini, misalnya:
	\begin{itemize}
		\item Apakah universitas sudah memiliki Learning Management System (LMS) yang memadai untuk mendukung pembelajaran daring?
		\item Sejauh mana infrastruktur IT yang ada dapat menampung lonjakan penggunaan platform daring selama jam perkuliahan?
		\item Apakah sumber daya manusia (dosen dan staf) telah memiliki kompetensi yang memadai untuk mengelola pembelajaran hybrid?
	\end{itemize}
	
	Langkah ini akan membantu universitas dalam mengidentifikasi kekuatan dan kelemahan yang ada sebelum memulai pengembangan kapabilitas lebih lanjut.
	
	\item \textbf{Melakukan penilaian kematangan arsitektur/perubahan bisnis}
	
	Penilaian kematangan arsitektur dan perubahan bisnis dilakukan dengan cara menilai kesiapan universitas dalam mengadopsi dan menerapkan model `hybrid teaching and learning`. Contoh nyata dari penilaian ini meliputi:
	\begin{itemize}
		\item Menilai sejauh mana program studi yang ada dapat dikonversi menjadi format hybrid tanpa mengurangi kualitas pembelajaran.
		\item Melakukan survei kesiapan mahasiswa dan dosen dalam menggunakan platform daring sebagai bagian dari kegiatan belajar-mengajar.
		\item Menentukan apakah kebijakan akademik dan administratif mendukung pelaksanaan kegiatan pembelajaran hybrid.
	\end{itemize}
	
	Hasil dari penilaian ini akan menjadi panduan dalam menetapkan prioritas pengembangan kapabilitas.
	
	\item \textbf{Mengidentifikasi kesenjangan di area kerja yang ada}
	
	Identifikasi kesenjangan dilakukan dengan cara membandingkan kapabilitas yang ada dengan kapabilitas ideal yang ingin dicapai. Contoh kesenjangan yang mungkin ditemukan adalah:
	\begin{itemize}
		\item Kurangnya pelatihan bagi dosen untuk membuat materi pembelajaran interaktif yang sesuai dengan platform daring.
		\item Sistem informasi akademik yang belum terintegrasi dengan LMS, sehingga menyulitkan proses pengelolaan nilai dan absensi mahasiswa.
		\item Infrastruktur jaringan yang tidak stabil sehingga menyulitkan akses mahasiswa terhadap materi ajar daring.
	\end{itemize}
	
	Identifikasi kesenjangan ini penting agar universitas dapat menentukan langkah-langkah strategis untuk mengatasi hambatan-hambatan yang ada.
	
	\item \textbf{Menetapkan peran dan tanggung jawab utama untuk pengelolaan dan pengawasan kapabilitas arsitektur enterprise}
	
	Dalam penerapan `hybrid teaching and learning`, menetapkan peran dan tanggung jawab yang jelas sangatlah penting. Contoh penetapan peran meliputi:
	\begin{itemize}
		\item \textbf{Hybrid Learning Coordinator}: Bertanggung jawab dalam memastikan semua kegiatan pembelajaran hybrid berjalan lancar, mengatasi kendala yang muncul, dan menjadi penghubung antara dosen dan tim IT.
		\item \textbf{Instructional Designer}: Berperan untuk membantu dosen dalam merancang materi ajar yang sesuai dengan metode hybrid, baik untuk pembelajaran sinkron maupun asinkron.
		\item \textbf{Technical Support Specialist}: Menyediakan dukungan teknis untuk dosen dan mahasiswa, terutama dalam penggunaan LMS, sistem video conference, serta perangkat keras dan lunak lainnya.
	\end{itemize}
	
	Penetapan peran ini memastikan adanya pembagian tanggung jawab yang jelas, sehingga kapabilitas hybrid dapat dikelola dengan lebih efisien.
	
	\item \textbf{Menulis permintaan perubahan untuk proyek yang ada (dokumen resmi)}
	
	Jika kapabilitas `hybrid teaching and learning` akan mempengaruhi proyek-proyek yang sedang berjalan di universitas, diperlukan permintaan perubahan resmi. Contoh dokumen yang dapat disusun:
	\begin{itemize}
		\item Permintaan perubahan untuk menyelaraskan proyek pengembangan kurikulum dengan strategi hybrid.
		\item Permintaan perubahan untuk pengembangan infrastruktur IT yang mendukung platform daring dan luring secara terintegrasi.
		\item Permintaan perubahan untuk menyediakan anggaran tambahan guna pelatihan dosen dalam memanfaatkan platform hybrid.
	\end{itemize}
	
	Dokumen ini harus disusun secara rinci dan disetujui oleh pihak-pihak yang terkait, seperti manajemen fakultas dan divisi IT.
	
	\item \textbf{Mendefinisikan batasan dalam pekerjaan arsitektur enterprise}
	
	Dalam kapabilitas `hybrid teaching and learning`, mendefinisikan batasan kerja penting untuk menghindari konflik tanggung jawab dan memastikan fokus pada tujuan utama. Contoh batasan dapat meliputi:
	\begin{itemize}
		\item Batasan pada pengembangan konten hanya untuk mata kuliah yang secara formal terdaftar dalam kurikulum hybrid.
		\item Batasan tanggung jawab IT Support Specialist hanya pada permasalahan teknis dan tidak mencakup pelatihan materi ajar.
		\item Batasan akses pada sistem informasi bagi mahasiswa hanya pada jam-jam tertentu untuk memastikan stabilitas server.
	\end{itemize}
	
	Batasan-batasan ini perlu didefinisikan dengan jelas agar setiap pihak yang terlibat memahami lingkup pekerjaan masing-masing.
	
	\item \textbf{Meninjau dan menyetujui dengan sponsor dan dewan}
	
	Sebelum kapabilitas `hybrid teaching and learning` diimplementasikan, perlu ada persetujuan resmi dari sponsor (pihak yang mendanai proyek) dan dewan universitas. Proses ini dapat meliputi:
	\begin{itemize}
		\item Presentasi strategi dan rencana implementasi kapabilitas hybrid kepada sponsor dan dewan.
		\item Diskusi mengenai alokasi sumber daya (dana, tenaga kerja, infrastruktur) yang diperlukan.
		\item Persetujuan atas rencana kerja dan penetapan milestone yang harus dicapai.
	\end{itemize}
	
	Persetujuan ini akan memastikan bahwa seluruh pemangku kepentingan memiliki pemahaman yang sama dan mendukung penuh pelaksanaan kapabilitas hybrid di universitas.
	
	\item \textbf{Menilai kebutuhan anggaran}
	
	Penilaian kebutuhan anggaran dilakukan dengan memperkirakan biaya yang dibutuhkan untuk penerapan kapabilitas `hybrid teaching and learning`. Contoh perhitungan anggaran dapat mencakup:
	\begin{itemize}
		\item Biaya pengembangan infrastruktur IT (server, jaringan, lisensi perangkat lunak).
		\item Biaya pelatihan dan pengembangan kompetensi dosen.
		\item Biaya pemeliharaan dan dukungan teknis selama periode implementasi awal.
		\item Biaya pengembangan materi ajar digital, seperti video pembelajaran, modul interaktif, dan lain-lain.
	\end{itemize}
	
	Estimasi anggaran ini akan menjadi acuan bagi manajemen universitas dalam menyusun rencana pembiayaan yang realistis dan sesuai kebutuhan.
\end{itemize}

\section{Contoh Prinsip-Prinsip Arsitektur}
Berikut adalah beberapa contoh prinsip arsitektur:

\begin{itemize}
	\item \textbf{Prinsip Pembelajaran Fleksibel}
	\begin{itemize}
		\item \textbf{Pernyataan}: Mahasiswa harus memiliki fleksibilitas untuk memilih format pembelajaran yang sesuai dengan kebutuhan dan preferensi mereka, baik secara daring maupun luring.
		\item \textbf{Alasan}: Penerapan prinsip ini akan meningkatkan kenyamanan dan motivasi belajar mahasiswa, memungkinkan mereka untuk mengikuti perkuliahan tanpa kendala lokasi atau waktu, serta memperluas akses pendidikan bagi mahasiswa yang memiliki keterbatasan fisik atau komitmen waktu lainnya.
		\item \textbf{Implikasi}: Implementasi pembelajaran fleksibel memerlukan investasi pada infrastruktur teknologi, seperti platform LMS (Learning Management System) yang andal, fasilitas ruang kelas hybrid, serta pelatihan bagi dosen untuk merancang dan menyampaikan materi pembelajaran yang efektif dalam format daring maupun luring.
	\end{itemize}
	
	\item \textbf{Prinsip Akses Sumber Belajar yang Terintegrasi}
	\begin{itemize}
		\item \textbf{Pernyataan}: Mahasiswa dan dosen harus memiliki akses yang terintegrasi ke seluruh sumber belajar, termasuk materi kuliah, rekaman video, forum diskusi, dan hasil evaluasi akademik.
		\item \textbf{Alasan}: Akses yang terintegrasi akan mendukung proses belajar mengajar yang lebih efisien dan transparan, mengurangi kebingungan dalam mencari sumber materi, serta memudahkan interaksi antara mahasiswa dan dosen di berbagai platform.
		\item \textbf{Implikasi}: Diperlukan pengembangan dan integrasi sistem informasi kampus yang menyatukan berbagai sumber belajar ke dalam satu portal. Selain itu, kebijakan akses yang aman dan terstruktur perlu diterapkan untuk melindungi data akademik dan pribadi mahasiswa.
	\end{itemize}
	
	\item \textbf{Prinsip Interaksi Aktif}
	\begin{itemize}
		\item \textbf{Pernyataan}: Proses pembelajaran harus mendorong interaksi aktif antara mahasiswa dan dosen, serta antar mahasiswa itu sendiri, baik secara daring maupun luring.
		\item \textbf{Alasan}: Interaksi aktif meningkatkan keterlibatan mahasiswa, memperdalam pemahaman konsep, dan memfasilitasi kolaborasi yang lebih baik dalam menyelesaikan tugas dan proyek.
		\item \textbf{Implikasi}: Penggunaan teknologi interaktif, seperti alat kolaborasi daring (misalnya, Google Workspace atau Microsoft Teams), serta penerapan teknik pembelajaran berbasis proyek dan diskusi kelompok harus didorong untuk menciptakan lingkungan belajar yang dinamis dan partisipatif.
	\end{itemize}
\end{itemize}

\section{Contoh \textit{Request for Architectural Work}: Penambahan Kapabilitas Hybrid Working di Universitas ABC}
\label{Request for Architectural Work}

\subsection*{1. Sponsor Organisasi}
\begin{itemize}
	\item \textbf{Nama Sponsor}: Dr. Andi Setiawan
	\item \textbf{Posisi}: Wakil Rektor Bidang Teknologi dan Inovasi
	\item \textbf{Kontak}: andi.setiawan@abcuniversity.ac.id, +62-812-3456-7890
\end{itemize}

\subsection*{2. Pernyataan Misi Organisasi}
“Menyediakan lingkungan pendidikan yang inklusif, inovatif, dan terhubung secara global untuk mengembangkan generasi profesional dan pemimpin masa depan.”

\subsection*{3. Tujuan Bisnis (dan Perubahan)}
\begin{itemize}
	\item \textbf{Tujuan 1}: Meningkatkan fleksibilitas kerja untuk mendukung pengaturan hybrid working bagi dosen dan staf administrasi.
	\item \textbf{Tujuan 2}: Meningkatkan produktivitas dan kepuasan pegawai melalui peningkatan kapabilitas kerja jarak jauh.
	\item \textbf{Tujuan 3}: Mengoptimalkan pemanfaatan fasilitas kampus dengan mengurangi kebutuhan ruang kerja tetap.
	\item \textbf{Perubahan}: Bertransisi dari model kerja di kantor menjadi model hybrid di mana pegawai dapat memilih untuk bekerja di kampus atau dari rumah.
\end{itemize}

\subsection*{4. Rencana Strategis Bisnis}
\begin{itemize}
	\item Menerapkan lingkungan kerja hybrid yang seimbang antara kerja jarak jauh dan di kampus.
	\item Membekali dosen dan staf administrasi dengan alat digital yang diperlukan untuk memastikan kolaborasi dan komunikasi yang lancar.
	\item Mengembangkan panduan dan kebijakan untuk pengaturan kerja hybrid guna menjaga efisiensi operasional.
\end{itemize}

\subsection*{5. Batas Waktu}
\begin{itemize}
	\item \textbf{Tanggal Mulai Proyek}: 1 November 2024
	\item \textbf{Tanggal Penyelesaian Proyek}: 31 Maret 2025
\end{itemize}

\subsection*{6. Perubahan pada Lingkungan Bisnis}
\begin{itemize}
	\item Meningkatnya permintaan untuk pengaturan kerja jarak jauh dan fleksibel seiring perubahan budaya kerja global pasca pandemi COVID-19.
	\item Kebutuhan untuk menyesuaikan harapan pegawai universitas terkait keseimbangan kerja dan kehidupan.
\end{itemize}

\subsection*{7. Batasan Organisasi}
\begin{itemize}
	\item Implementasi pengaturan kerja hybrid tidak boleh mengganggu operasional akademik dan administrasi yang penting.
	\item Kepatuhan terhadap kebijakan universitas terkait keamanan data dan kerahasiaan saat bekerja jarak jauh.
\end{itemize}

\subsection*{8. Informasi Anggaran, Batasan Finansial}
\begin{itemize}
	\item \textbf{Anggaran yang Dialokasikan}: IDR 2.000.000.000
	\item Anggaran mencakup peningkatan infrastruktur TI, lisensi perangkat lunak, alat kolaborasi digital, serta pelatihan bagi pegawai.
\end{itemize}

\subsection*{9. Batasan Eksternal, Batasan Bisnis}
\begin{itemize}
	\item Kepatuhan terhadap regulasi pemerintah terkait kerja jarak jauh dan manajemen data digital.
	\item Menjamin interoperabilitas antara sistem baru dan infrastruktur TI yang ada.
\end{itemize}

\subsection*{10. Deskripsi Sistem Bisnis Saat Ini}
\begin{itemize}
	\item Operasional universitas sebagian besar dilakukan secara tatap muka, dengan kapabilitas kerja jarak jauh yang terbatas dan hanya digunakan saat pandemi.
	\item Alat kolaborasi yang ada mencakup layanan video conference dasar dan komunikasi email.
\end{itemize}

\subsection*{11. Deskripsi Arsitektur/Sistem TI Saat Ini}
\begin{itemize}
	\item Infrastruktur TI saat ini adalah campuran antara server on-premise dan layanan cloud yang terbatas untuk penyimpanan dan komunikasi.
	\item Sistem yang ada belum mendukung alat kolaborasi canggih seperti ruang kerja virtual bersama atau alat manajemen proyek yang terintegrasi.
\end{itemize}

\subsection*{12. Deskripsi Organisasi yang Mengembangkan Arsitektur}
\begin{itemize}
	\item \textbf{Nama Organisasi}: Departemen TI Universitas ABC
	\item \textbf{Ketua Tim}: Bapak Joko Santoso, Kepala Departemen TI
	\item \textbf{Tanggung Jawab}: Tim akan mengawasi implementasi kapabilitas kerja hybrid, termasuk integrasi sistem, pelatihan staf, dan pengembangan kebijakan.
\end{itemize}

\subsection*{13. Deskripsi Sumber Daya yang Tersedia pada Organisasi yang Mengembangkan Arsitektur}
\begin{itemize}
	\item \textbf{Sumber Daya Manusia}: Tim profesional TI yang berdedikasi, mengkhususkan diri dalam administrasi sistem, rekayasa jaringan, dan pengembangan aplikasi.
	\item \textbf{Sumber Daya Teknis}: Akses ke pusat data on-premise dan infrastruktur cloud yang ada untuk pengujian dan pengembangan.
	\item \textbf{Sumber Daya Finansial}: IDR 2.000.000.000 yang dialokasikan khusus untuk proyek ini.
\end{itemize}


\section{Aktivitas Kelas dan Tugas}
 Definisikan kapabilitas arsitektur Anda berdasarkan informasi yang dikumpulkan dalam fase Preliminary. Deskripsikan keadaan saat ini (As-Is) dan keadaan target (To-Be) untuk memperjelas kapabilitas tersebut dalam bentuk dokumen \textit{Request for Architectural Work}.