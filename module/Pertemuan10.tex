\chapter{Fase E: Peluang dan Solusi}

\section{Tujuan}
Tujuan dari fase ini adalah:
\begin{enumerate}
	\item Menghasilkan versi lengkap awal dari Peta Jalan Arsitektur, berdasarkan analisis kesenjangan dan komponen kandidat Peta Jalan Arsitektur dari Fase B, C, dan D.
	\item Menentukan apakah pendekatan bertahap diperlukan, dan jika ya, mengidentifikasi Arsitektur Transisi yang akan memberikan nilai bisnis yang berkelanjutan.
\end{enumerate}

\section{Input}
Fase ini membutuhkan masukan sebagai berikut:
\begin{itemize}
	\item Informasi produk
	\item Permintaan untuk Pekerjaan Arsitektur
	\item Penilaian Kapabilitas
	\item Rencana Komunikasi
	\item Metodologi Perencanaan
	\item Model Organisasi untuk Arsitektur Perusahaan
	\item Model dan Kerangka Tata Kelola
	\item Kerangka Arsitektur yang Disesuaikan
	\item Pernyataan Pekerjaan Arsitektur
	\item Visi Arsitektur
	\item Repositori Arsitektur
	\item Draf Dokumen Definisi Arsitektur
	\item Draf Spesifikasi Persyaratan Arsitektur
	\item Permintaan Perubahan untuk program dan proyek yang ada
	\item Komponen kandidat Peta Jalan Arsitektur dari Fase B, C, dan D
\end{itemize}

\subsection*{Contoh:}

\begin{itemize}
	\item \textbf{Informasi Produk}: Informasi ini mencakup rincian mengenai produk teknologi yang akan mendukung pembelajaran hybrid, termasuk spesifikasi dan kemampuan produk. Misalnya, informasi produk untuk Learning Management System (LMS) seperti Moodle, mencakup fitur manajemen kelas, integrasi alat evaluasi, dan keamanan data. Contoh lainnya adalah informasi produk kamera PTZ (Pan-Tilt-Zoom) yang digunakan di ruang kelas, yang mencakup resolusi, kompatibilitas perangkat lunak, dan jangkauan.
	
	\item \textbf{Permintaan untuk Pekerjaan Arsitektur}: Dokumen ini menjelaskan kebutuhan spesifik untuk merancang atau meningkatkan arsitektur teknologi. Misalnya, permintaan untuk merancang jaringan kampus yang memungkinkan akses cepat ke LMS dari seluruh kampus dan area luar. Contoh lainnya adalah permintaan untuk memperbarui sistem keamanan data agar memenuhi kebijakan privasi mahasiswa dan staf dalam pembelajaran hybrid.
	
	\item \textbf{Penilaian Kapabilitas}: Penilaian ini mengevaluasi kemampuan universitas dalam mendukung teknologi untuk pembelajaran hybrid. Misalnya, penilaian terhadap infrastruktur jaringan untuk mendukung video streaming tanpa gangguan. Contoh lainnya adalah penilaian kemampuan tim TI universitas untuk menyediakan dukungan teknis bagi dosen dan mahasiswa selama pembelajaran hybrid.
	
	\item \textbf{Rencana Komunikasi}: Rencana ini menetapkan bagaimana informasi akan disampaikan kepada pemangku kepentingan terkait implementasi teknologi hybrid. Misalnya, rencana untuk menginformasikan dosen dan mahasiswa tentang perubahan yang memungkinkan akses langsung ke sesi video konferensi melalui LMS. Contoh lainnya adalah rencana komunikasi dengan staf administrasi tentang prosedur baru dalam manajemen data mahasiswa pada platform hybrid.
	
	\item \textbf{Metodologi Perencanaan}: Metodologi ini mencakup pendekatan perencanaan yang digunakan untuk implementasi arsitektur hybrid learning. Misalnya, metodologi Agile untuk memungkinkan pengembangan dan penyesuaian LMS secara bertahap sesuai kebutuhan pengguna. Contoh lainnya adalah penggunaan metodologi Waterfall untuk proyek instalasi perangkat keras jaringan, di mana setiap fase harus diselesaikan sebelum melanjutkan ke fase berikutnya.
	
	\item \textbf{Model Organisasi untuk Arsitektur Perusahaan}: Model ini mendefinisikan struktur organisasi untuk mendukung penerapan arsitektur hybrid learning. Misalnya, model organisasi yang melibatkan tim TI, fakultas, dan manajemen kampus dalam proses penerapan LMS. Contoh lainnya adalah pembentukan tim khusus yang menangani koordinasi proyek integrasi ruang kelas fisik dengan alat digital.
	
	\item \textbf{Model dan Kerangka Tata Kelola}: Model ini menetapkan pedoman tata kelola untuk mengelola arsitektur teknologi. Misalnya, kerangka tata kelola yang memastikan kepatuhan terhadap regulasi keamanan data mahasiswa dan staf. Contoh lainnya adalah model tata kelola yang mengatur hak akses ke data di LMS untuk mencegah akses tidak sah ke informasi akademik.
	
	\item \textbf{Kerangka Arsitektur yang Disesuaikan}: Kerangka ini menyesuaikan metodologi arsitektur untuk memenuhi kebutuhan khusus universitas dalam mendukung hybrid learning. Misalnya, kerangka yang disesuaikan untuk integrasi LMS dengan sistem administrasi kampus. Contoh lainnya adalah kerangka yang menggabungkan standar pedagogi dan teknis dalam implementasi ruang kelas hybrid.
	
	\item \textbf{Pernyataan Pekerjaan Arsitektur}: Pernyataan ini mendokumentasikan ruang lingkup dan tujuan dari pekerjaan arsitektur teknologi. Misalnya, pernyataan untuk proyek integrasi LMS dengan sistem video konferensi, yang menjelaskan cakupan fitur-fitur yang akan dikembangkan. Contoh lainnya adalah pernyataan pekerjaan untuk peningkatan jaringan kampus yang mendukung konektivitas dan akses ke LMS dari seluruh area kampus.
	
	\item \textbf{Visi Arsitektur}: Visi ini menggambarkan tujuan jangka panjang dari arsitektur hybrid learning di universitas. Misalnya, visi yang mencakup integrasi antara LMS, video konferensi, dan ruang kelas digital untuk akses seamless. Contoh lainnya adalah visi yang memprioritaskan keamanan data dan aksesibilitas, memungkinkan mahasiswa dan dosen untuk mengakses materi dari perangkat apa pun dengan tingkat keamanan yang tinggi.
	
	\item \textbf{Repositori Arsitektur}: Repositori ini menyimpan dokumentasi dan perubahan arsitektur teknologi. Misalnya, repositori yang menyimpan desain dan spesifikasi perangkat keras untuk ruang kelas hybrid. Contoh lainnya adalah repositori yang mencatat pembaruan dan perubahan pada LMS, termasuk integrasi baru dan konfigurasi keamanan.
	
	\item \textbf{Draf Dokumen Definisi Arsitektur}: Dokumen ini mendeskripsikan rincian arsitektur teknologi saat ini dan target yang dirancang untuk mendukung hybrid learning. Misalnya, draf yang mencakup deskripsi infrastruktur jaringan, perangkat audio-visual, dan LMS. Contoh lainnya adalah draf yang mendeskripsikan arsitektur target yang menggabungkan LMS dengan aplikasi administrasi akademik.
	
	\item \textbf{Draf Spesifikasi Persyaratan Arsitektur}: Dokumen ini mencantumkan persyaratan teknis dan operasional untuk arsitektur hybrid learning. Misalnya, persyaratan agar LMS mampu menampung hingga 10.000 pengguna aktif pada waktu yang sama. Contoh lainnya adalah persyaratan agar data mahasiswa yang disimpan di server dienkripsi dan diakses hanya oleh pihak berwenang.
	
	\item \textbf{Permintaan Perubahan untuk Program dan Proyek yang Ada}: Permintaan ini mengajukan modifikasi pada proyek yang sedang berjalan untuk memenuhi kebutuhan hybrid learning. Misalnya, permintaan perubahan untuk menambahkan fitur kelas asinkron pada LMS. Contoh lainnya adalah permintaan peningkatan bandwidth jaringan kampus agar dapat menangani lebih banyak koneksi video konferensi secara bersamaan.
	
	\item \textbf{Komponen Kandidat Peta Jalan Arsitektur dari Fase B, C, dan D}: Komponen ini adalah elemen kandidat untuk dimasukkan dalam peta jalan arsitektur yang direncanakan. Misalnya, komponen peningkatan jaringan untuk mendukung akses hybrid learning. Contoh lainnya adalah komponen untuk mengintegrasikan sistem kehadiran dengan LMS agar data kehadiran dari kelas daring dan tatap muka dapat terhubung secara otomatis.
\end{itemize}


\section{Langkah-langkah}
\begin{enumerate}
	\item Menentukan/mengonfirmasi atribut perubahan utama perusahaan
	\item Menentukan kendala bisnis untuk implementasi
	\item Meninjau dan mengonsolidasi hasil Analisis Kesenjangan dari Fase B hingga D
	\item Meninjau persyaratan terintegrasi di berbagai fungsi bisnis terkait
	\item Mengonsolidasi dan merekonsiliasi persyaratan interoperabilitas
	\item Memperbaiki dan memvalidasi ketergantungan
	\item Mengonfirmasi kesiapan dan risiko untuk transformasi bisnis
	\item Merumuskan Strategi Implementasi dan Migrasi
	\item Mengidentifikasi dan mengelompokkan paket pekerjaan utama
	\item Mengidentifikasi Arsitektur Transisi
	\item Membuat Peta Jalan Arsitektur \& Rencana Implementasi dan Migrasi
\end{enumerate}

\subsection*{Contoh:}

\begin{itemize}
	\item \textbf{Menentukan/mengonfirmasi atribut perubahan utama perusahaan}: Langkah ini mencakup penilaian atribut utama yang perlu diubah dalam rangka mendukung pembelajaran hybrid, seperti budaya, kebijakan, dan kemampuan teknologi. Misalnya, perubahan pada kebijakan akses jaringan untuk memastikan mahasiswa dan dosen dapat mengakses platform LMS dari luar kampus. Contoh lainnya adalah perubahan budaya yang mengutamakan digitalisasi, dengan pelatihan bagi dosen untuk mengoptimalkan penggunaan alat-alat digital dalam pembelajaran.
	
	\item \textbf{Menentukan kendala bisnis untuk implementasi}: Identifikasi kendala bisnis yang dapat mempengaruhi urutan implementasi atau keputusan terkait teknologi hybrid. Misalnya, anggaran terbatas untuk peningkatan jaringan yang mengharuskan universitas memprioritaskan peningkatan pada gedung-gedung utama terlebih dahulu. Contoh lainnya adalah waktu yang terbatas pada semester berjalan, sehingga implementasi dilakukan saat libur semester agar tidak mengganggu proses pembelajaran.
	
	\item \textbf{Meninjau dan mengonsolidasi hasil Analisis Kesenjangan dari Fase B hingga D}: Mengonsolidasi hasil analisis kesenjangan yang mengidentifikasi perbedaan antara infrastruktur yang ada dengan kebutuhan untuk mendukung pembelajaran hybrid. Misalnya, analisis kesenjangan dapat menunjukkan bahwa kapasitas server LMS perlu ditingkatkan untuk menangani jumlah pengguna yang meningkat selama periode ujian. Contoh lain adalah identifikasi kebutuhan tambahan kamera dan mikrofon berkualitas tinggi di ruang kelas untuk memperbaiki kualitas pembelajaran daring.
	
	\item \textbf{Meninjau persyaratan terintegrasi di berbagai fungsi bisnis terkait}: Menilai kebutuhan teknologi yang terintegrasi di berbagai fungsi kampus seperti akademik, administrasi, dan perpustakaan untuk memastikan solusi pembelajaran hybrid terpenuhi. Misalnya, memastikan data mahasiswa di LMS dapat terintegrasi dengan sistem perpustakaan digital. Contoh lainnya adalah memastikan bahwa kalender akademik di portal kampus selaras dengan jadwal kelas daring dan luring di LMS.
	
	\item \textbf{Mengonsolidasi dan merekonsiliasi persyaratan interoperabilitas}: Mengumpulkan persyaratan interoperabilitas antara berbagai sistem yang digunakan untuk hybrid learning agar integrasi berjalan lancar. Misalnya, integrasi antara platform LMS dengan aplikasi video konferensi seperti Zoom, yang memerlukan standar interoperabilitas agar mahasiswa dapat mengakses sesi daring langsung dari LMS. Contoh lainnya adalah sinkronisasi data kehadiran mahasiswa di kelas daring dengan sistem administrasi akademik.
	
	\item \textbf{Memperbaiki dan memvalidasi ketergantungan}: Memastikan bahwa semua ketergantungan antar sistem dan perangkat dalam arsitektur hybrid learning telah diidentifikasi dan divalidasi. Misalnya, ketergantungan antara server LMS dan koneksi jaringan kampus yang harus mendukung streaming berkualitas tinggi. Contoh lain adalah ketergantungan perangkat keras seperti kamera PTZ (pan-tilt-zoom) dan perangkat lunak video konferensi yang digunakan dalam ruang kelas hybrid.
	
	\item \textbf{Mengonfirmasi kesiapan dan risiko untuk transformasi bisnis}: Mengidentifikasi kesiapan universitas untuk mengadopsi pembelajaran hybrid serta menilai risiko terkait implementasi ini. Misalnya, menilai kesiapan staf TI untuk menangani dukungan teknis yang meningkat selama transisi ke pembelajaran hybrid. Contoh lainnya adalah mengidentifikasi risiko keamanan data karena akses remote yang lebih luas ke sistem kampus.
	
	\item \textbf{Merumuskan Strategi Implementasi dan Migrasi}: Menyusun strategi implementasi, baik melalui pendekatan Greenfield, Revolusioner, atau Evolusioner, untuk mendukung penerapan hybrid learning. Misalnya, strategi Evolusioner dengan memulai pengenalan kelas hybrid di beberapa program studi terlebih dahulu sebagai tahap uji coba. Contoh lainnya adalah strategi Revolusioner untuk menerapkan sistem manajemen kelas secara serentak di seluruh fakultas pada awal semester baru.
	
	\item \textbf{Mengidentifikasi dan mengelompokkan paket pekerjaan utama}: Mengelompokkan kegiatan implementasi menjadi paket pekerjaan yang masing-masing mencakup rangkaian tugas terkait. Misalnya, paket pekerjaan untuk instalasi perangkat audio-visual di ruang kelas, yang mencakup pemasangan mikrofon, speaker, dan kamera. Contoh lainnya adalah paket pekerjaan untuk integrasi LMS dengan sistem administrasi, termasuk penyiapan API dan uji coba integrasi data mahasiswa.
	
	\item \textbf{Mengidentifikasi Arsitektur Transisi}: Menyusun arsitektur transisi yang diperlukan untuk mencapai arsitektur target secara bertahap. Misalnya, arsitektur transisi yang meliputi pengaturan ruang kelas dengan perlengkapan minimum sebelum pembaruan ke konfigurasi penuh. Contoh lainnya adalah menyediakan akses sementara ke LMS melalui aplikasi mobile sebelum peluncuran aplikasi yang terintegrasi penuh.
	
	\item \textbf{Membuat Peta Jalan Arsitektur \& Rencana Implementasi dan Migrasi}: Menyusun peta jalan yang mencakup urutan implementasi serta rencana migrasi ke arsitektur hybrid learning target. Misalnya, peta jalan untuk peningkatan infrastruktur TI yang dimulai dengan peningkatan kapasitas jaringan sebelum integrasi LMS dan aplikasi video konferensi. Contoh lain adalah rencana migrasi yang menjadwalkan integrasi bertahap data mahasiswa dari sistem lama ke LMS baru.
\end{itemize}


\section{Keluaran}
Keluaran dari fase ini mencakup:
\begin{itemize}
	\item Pernyataan Pekerjaan Arsitektur, diperbarui jika diperlukan
	\item Visi Arsitektur, diperbarui jika diperlukan
	\item Draf Dokumen Definisi Arsitektur, diperbarui jika diperlukan, termasuk:
	\begin{itemize}
		\item Arsitektur Transisi, jika ada
	\end{itemize}
	\item Draf Spesifikasi Persyaratan Arsitektur, termasuk:
	\begin{itemize}
		\item Penilaian Kesenjangan, Solusi, dan Kajian Saling Ketergantungan
	\end{itemize}
	\item Penilaian Kapabilitas, termasuk:
	\begin{itemize}
		\item Penilaian Kapabilitas Bisnis
		\item Penilaian Kapabilitas TI
	\end{itemize}
	\item Peta Jalan Arsitektur, termasuk:
	\begin{itemize}
		\item Portofolio paket pekerjaan
		\item Identifikasi Arsitektur Transisi, jika ada
		\item Rekomendasi implementasi
	\end{itemize}
	\item Rencana Implementasi dan Migrasi (garis besar)
\end{itemize}

\subsection*{Contoh:}

\begin{itemize}
	\item \textbf{Pernyataan Pekerjaan Arsitektur, diperbarui jika diperlukan}: Pernyataan ini mendokumentasikan ruang lingkup dan tujuan dari pekerjaan arsitektur yang direncanakan atau sedang berlangsung dalam mendukung pembelajaran hybrid. Misalnya, pernyataan ini dapat mencakup deskripsi integrasi Learning Management System (LMS) dengan sistem video konferensi untuk mendukung kelas daring dan luring secara bersamaan. Contoh lain adalah pernyataan yang diperbarui untuk proyek peningkatan infrastruktur jaringan kampus agar dapat mendukung akses cepat ke platform e-learning dari berbagai lokasi di kampus.
	
	\item \textbf{Visi Arsitektur, diperbarui jika diperlukan}: Visi arsitektur menggambarkan tujuan jangka panjang dari arsitektur teknologi yang dirancang untuk mendukung hybrid learning. Sebagai contoh, visi arsitektur untuk hybrid learning di universitas dapat mencakup akses seamless antara aplikasi LMS, video konferensi, dan ruang kelas digital. Contoh lain adalah visi yang berfokus pada pengoptimalan pengalaman pengguna, di mana mahasiswa dan dosen dapat mengakses konten dan interaksi kelas dari perangkat apa pun dengan keamanan dan stabilitas yang terjamin.
	
	\item \textbf{Draf Dokumen Definisi Arsitektur, diperbarui jika diperlukan, termasuk Arsitektur Transisi jika ada}: Dokumen ini merinci arsitektur teknologi saat ini dan target untuk mendukung pembelajaran hybrid, serta langkah-langkah transisi menuju arsitektur target. Misalnya, pada pengembangan hybrid learning, arsitektur transisi dapat berupa integrasi bertahap aplikasi LMS dan video konferensi, sehingga dosen dan mahasiswa dapat beradaptasi. Contoh lainnya adalah transisi sistem akademik yang mencakup migrasi data ke cloud untuk memudahkan akses remote bagi seluruh pengguna.
	
	\item \textbf{Draf Spesifikasi Persyaratan Arsitektur, termasuk Penilaian Kesenjangan, Solusi, dan Kajian Saling Ketergantungan}: Spesifikasi ini mencantumkan persyaratan teknis dan solusi untuk memenuhi kebutuhan hybrid learning. Misalnya, proyek peningkatan infrastruktur kelas hybrid mungkin mengidentifikasi kesenjangan dalam sistem audio-visual ruang kelas yang mengakibatkan kualitas audio yang tidak optimal bagi peserta daring, sehingga solusi mikrofon tambahan atau speaker menjadi pilihan. Contoh lain adalah proyek integrasi LMS dengan sistem administrasi yang mengurangi kesenjangan interoperabilitas sehingga data mahasiswa dan kelas bisa sinkron secara otomatis.
	
	\item \textbf{Penilaian Kapabilitas, termasuk Penilaian Kapabilitas Bisnis dan TI}: Penilaian ini mengukur kesiapan universitas dalam mendukung pembelajaran hybrid dari perspektif bisnis dan teknologi. Misalnya, dalam proyek hybrid learning, penilaian kapabilitas bisnis dapat mengidentifikasi kebutuhan tambahan staf TI untuk mendukung kelas daring dan luring secara bersamaan. Contoh lainnya, dalam penilaian kapabilitas TI, dapat mencakup pengukuran kapasitas server LMS untuk mendukung streaming video beresolusi tinggi dan mengakomodasi ribuan pengguna aktif secara bersamaan.
	
	\item \textbf{Peta Jalan Arsitektur, termasuk Portofolio Paket Pekerjaan, Identifikasi Arsitektur Transisi jika ada, dan Rekomendasi Implementasi}: Peta jalan ini menyediakan langkah-langkah dan jadwal implementasi untuk mencapai arsitektur hybrid learning yang diinginkan. Sebagai contoh, dalam implementasi ruang kelas pintar, peta jalan ini mungkin mencakup paket pekerjaan yang meliputi instalasi perangkat audio-visual, integrasi perangkat lunak LMS, dan pengujian. Contoh lainnya adalah peta jalan untuk platform LMS terintegrasi yang menyarankan pembaruan perangkat keras server dan aplikasi untuk meningkatkan skalabilitas dan performa.
	
	\item \textbf{Rencana Implementasi dan Migrasi (garis besar)}: Rencana ini mengatur langkah-langkah untuk mengimplementasikan arsitektur hybrid learning secara bertahap. Misalnya, pada migrasi LMS ke cloud, rencana ini akan mencakup tahapan migrasi data, konfigurasi keamanan, dan pengujian fungsionalitas untuk memastikan layanan berjalan optimal. Contoh lainnya adalah pada proyek peningkatan jaringan kampus, rencana ini mencakup jadwal peningkatan perangkat keras seperti router dan switch untuk mendukung konektivitas tanpa batas di seluruh area kampus.
\end{itemize}


\section{Faktor Implementasi dan Matriks Penilaian Deduksi}

\begin{table}[th!]
	\centering
	\begin{tabular}{|p{0.05\textwidth}|p{0.18\textwidth}|p{0.3\textwidth}|p{0.36\textwidth}|}
		\hline
		\textbf{\#} & \textbf{Faktor Implementasi} & \textbf{Deskripsi} & \textbf{Deduksi} \\
		\hline
		1 & Perubahan Teknologi Pembelajaran & Universitas berencana mengganti sistem manajemen pembelajaran (Learning Management System, LMS) lama dengan LMS baru yang mendukung integrasi video konferensi dan modul interaktif. & Transisi ini membutuhkan pelatihan bagi dosen dan mahasiswa untuk menguasai platform baru, serta alokasi waktu migrasi data dari LMS lama ke LMS baru. Langkah ini perlu diprioritaskan agar transisi berjalan lancar sebelum semester baru dimulai. \\
		\hline
		2 & Konsolidasi Layanan IT & Menggabungkan beberapa layanan IT, seperti sistem administrasi akademik dan perpustakaan digital, ke dalam satu portal terpusat untuk memudahkan akses bagi mahasiswa dan dosen. & Konsolidasi ini memerlukan penyesuaian infrastruktur IT dan integrasi data dari sistem yang berbeda. Hal ini dapat memengaruhi urutan migrasi dan membutuhkan koordinasi antar-departemen, khususnya antara tim IT dan akademik. \\
		\hline
		3 & Pengadaan Perangkat Kelas Pintar untuk Pembelajaran Hybrid & Pengenalan perangkat audio-visual di ruang kelas untuk mendukung pembelajaran hybrid, seperti kamera otomatis dan mikrofon berkualitas tinggi yang dapat menangkap interaksi di ruang kelas. & Memerlukan pemasangan dan konfigurasi perangkat di ruang kelas serta pelatihan bagi dosen tentang penggunaan perangkat ini. Pengadaan dan pemasangan perangkat harus dijadwalkan sebelum semester baru dimulai untuk memastikan ketersediaan teknologi dalam pembelajaran hybrid. \\
		\hline
		4 & Peningkatan Kapasitas Jaringan Wi-Fi di Kampus & Menyediakan akses Wi-Fi yang stabil dan cepat di seluruh area kampus untuk mendukung akses LMS dan video konferensi bagi mahasiswa dan dosen. & Memerlukan peningkatan router dan titik akses (access points) di berbagai gedung. Proyek ini bergantung pada anggaran yang tersedia dan harus diselesaikan sebelum peningkatan LMS dan pengenalan kelas hybrid. \\
		\hline
		5 & Penambahan Saluran Layanan Mahasiswa Digital & Menyediakan layanan digital baru, seperti konseling akademik virtual dan pusat bantuan daring, untuk mendukung mahasiswa yang berpartisipasi dalam pembelajaran hybrid. & Memerlukan pelatihan bagi staf terkait, pengaturan perangkat lunak layanan, dan penjadwalan sumber daya untuk mendukung layanan ini. Hal ini dapat memengaruhi ketersediaan staf di unit lain dan harus diperhitungkan dalam rencana implementasi secara keseluruhan. \\
		\hline
	\end{tabular}
	\caption{Faktor Implementasi untuk Pembelajaran Hybrid di Universitas}
\end{table}

Sebuah \textbf{Faktor Implementasi} mengacu pada kondisi, batasan, atau aspek yang dapat secara signifikan memengaruhi perencanaan dan pelaksanaan rencana implementasi dan migrasi arsitektur. Faktor-faktor ini berdampak pada strategi, jadwal, dan manajemen risiko dalam proses migrasi dari keadaan saat ini ke arsitektur target, serta membantu dalam mengidentifikasi tindakan atau batasan yang perlu dipertimbangkan dalam tahap perencanaan. Dengan mendokumentasikan faktor-faktor ini, efeknya pada proyek dapat dipahami dengan lebih baik, menjadikannya elemen penting dalam perencanaan implementasi dan migrasi yang realistis dan efektif.

\textbf{Matriks Penilaian dan Deduksi Faktor Implementasi} adalah alat yang digunakan untuk mendokumentasikan dan menganalisis faktor-faktor ini. Matriks ini mencakup rincian tentang setiap faktor, deskripsinya, alasan di baliknya, dan deduksi atau dampak spesifik yang dimiliki terhadap rencana migrasi, membantu membentuk pendekatan manajemen risiko yang komprehensif. Matriks ini mendukung identifikasi tindakan yang diperlukan, seperti alokasi sumber daya, penentuan prioritas, atau perubahan strategi berdasarkan penilaian faktor-faktor tersebut.

\section{Matriks Konsolidasi Kesenjangan, Solusi, dan Ketergantungan}


\begin{table}[th!]
	\centering
	\begin{tabular}{|p{.02\textwidth}|p{.1\textwidth}|p{.33\textwidth}|p{.33\textwidth}|p{.08\textwidth}|}
		\hline
		\textbf{\#} & \textbf{Arsitektur} & \textbf{Gap} & \textbf{Solusi Potensial} & \textbf{Membutuhkan} \\
		\hline
		1 & Bisnis & Integrasi layanan mahasiswa untuk akses informasi akademik yang terpusat & 
		Pengembangan portal mahasiswa terpadu dengan akses ke informasi akademik, keuangan, dan administrasi & 
		 \\
		\hline
		2 & Aplikasi & LMS tidak memiliki fitur interaktif untuk pembelajaran daring & 
		Integrasi LMS dengan alat video konferensi dan forum diskusi real-time & 
		 \\
		\hline
		3 & Data & Kurangnya sistem pelaporan yang komprehensif untuk analisis data mahasiswa & 
		Implementasi sistem BI (Business Intelligence) dengan laporan yang dapat dikustomisasi & 
		 \\
		\hline
		4 & Teknologi & Keterbatasan kapasitas server dalam menangani akses serentak dari mahasiswa & 
		Upgrade server atau migrasi ke solusi cloud untuk kapasitas yang lebih besar & 
		 \\
		\hline
		5 & Bisnis & Kebutuhan untuk pelatihan dosen dalam penggunaan alat hybrid learning & 
		Program pelatihan berkala yang mencakup LMS, video konferensi, dan alat interaktif lainnya & 
		 \\
		\hline
		6 & Aplikasi & LMS tidak mendukung fitur pengumpulan dan pengarsipan tugas secara otomatis & 
		Penambahan modul manajemen tugas dan arsip dalam LMS & 
		 \\
		\hline
		7 & Data & Keterbatasan akses data real-time untuk evaluasi kinerja mahasiswa & 
		Integrasi sistem evaluasi kinerja dengan dashboard analitik & 
		 \\
		\hline
		8 & Teknologi & Tidak ada sistem autentikasi tunggal (Single Sign-On) untuk akses aplikasi & 
		Implementasi SSO untuk akses LMS, portal akademik, dan aplikasi lainnya & 
		 \\
		\hline
		9 & Bisnis & Keterbatasan komunikasi dan kolaborasi antara dosen dan mahasiswa & 
		Penerapan platform komunikasi terintegrasi seperti Microsoft Teams atau Slack & 
		 \\
		\hline
		10 & Teknologi & Kebutuhan akan jaringan Wi-Fi yang stabil dan cepat di seluruh area kampus & 
		Peningkatan kapasitas Wi-Fi di kampus dengan perangkat router baru & 
		 \\
		\hline
	\end{tabular}
	\caption{Tabel Gap dan Solusi Potensial dalam Arsitektur Hybrid Learning}
\end{table}



\textbf{Matriks Konsolidasi Kesenjangan, Solusi, dan Ketergantungan} adalah alat penting untuk mendokumentasikan, menganalisis, dan merencanakan tindakan berdasarkan identifikasi kesenjangan (*gaps*), solusi potensial, dan ketergantungan antar elemen dalam arsitektur yang diinginkan. Matriks ini mengonsolidasikan informasi mengenai komponen yang dibutuhkan dalam arsitektur target dan solusi untuk menjembatani perbedaan antara arsitektur saat ini (baseline) dan target.

\begin{enumerate}
	\item \textbf{Kesenjangan (Gaps)}: Elemen yang belum ada dalam arsitektur saat ini, tetapi diperlukan dalam arsitektur target. Kesenjangan dapat berupa fitur, kapabilitas, atau sumber daya tambahan yang dibutuhkan untuk mencapai tujuan arsitektur.
	
	\item \textbf{Solusi Potensial (Potential Solutions)}: Tindakan atau komponen yang dapat diimplementasikan untuk menjawab kesenjangan yang ada. Solusi ini berfungsi untuk memastikan bahwa semua elemen yang diperlukan dalam arsitektur target dapat dicapai.
	
	\item \textbf{Ketergantungan (Dependencies)}: Hubungan antara kesenjangan dan solusi yang menunjukkan urutan atau keterkaitan dalam implementasi. Misalnya, beberapa solusi mungkin harus diimplementasikan sebelum solusi lain dapat berfungsi dengan optimal.
\end{enumerate}

\section{Peta Jalan Arsitektur Hybrid Learning}
\label{sec:peta_jalan}

\begin{enumerate}
	\item \textbf{Kapasitas Server untuk Akses Serentak (Teknologi)}:
	\begin{itemize}
		\item \textbf{Gap:} Keterbatasan kapasitas server dalam menangani akses serentak.
		\item \textbf{Solusi:} Upgrade server atau migrasi ke solusi cloud untuk kapasitas yang lebih besar.
		\item \textbf{Implementasi:} Dilakukan sebagai prioritas utama untuk memastikan infrastruktur yang mendukung beban pengguna secara stabil sebelum aplikasi dan layanan data diterapkan.
	\end{itemize}
	
	\item \textbf{Jaringan Wi-Fi Kampus (Teknologi)}:
	\begin{itemize}
		\item \textbf{Gap:} Kebutuhan akan jaringan Wi-Fi yang stabil dan cepat di seluruh area kampus.
		\item \textbf{Solusi:} Peningkatan kapasitas Wi-Fi di kampus dengan perangkat router baru.
		\item \textbf{Implementasi:} Dilakukan setelah peningkatan server, mendukung akses nirkabel yang luas dan stabil untuk perangkat pengguna.
		\item \textbf{Alasan Urutan:} Wi-Fi stabil menjadi dasar penting bagi akses pengguna yang memerlukan mobilitas di area kampus.
	\end{itemize}
	
	\item \textbf{Sistem Autentikasi Tunggal (Single Sign-On) (Teknologi)}:
	\begin{itemize}
		\item \textbf{Gap:} Tidak ada sistem SSO untuk akses aplikasi kampus.
		\item \textbf{Solusi:} Implementasi SSO untuk akses LMS, portal akademik, dan aplikasi lainnya.
		\item \textbf{Implementasi:} Setelah jaringan Wi-Fi dan server siap, SSO memudahkan akses terintegrasi dan aman.
		\item \textbf{Alasan Urutan:} Infrastruktur stabil memungkinkan SSO memberikan pengalaman akses yang mulus dan aman.
	\end{itemize}
	
	\item \textbf{Integrasi Layanan Mahasiswa (Bisnis)}:
	\begin{itemize}
		\item \textbf{Gap:} Layanan mahasiswa belum terintegrasi untuk akses informasi akademik yang terpusat.
		\item \textbf{Solusi:} Mengembangkan portal mahasiswa terpadu yang memungkinkan akses ke informasi akademik, keuangan, dan administrasi secara terpusat.
		\item \textbf{Implementasi:} Setelah implementasi SSO untuk memudahkan akses terintegrasi.
		\item \textbf{Alasan Urutan:} Portal mahasiswa terpadu memerlukan autentikasi yang aman, yang disediakan oleh SSO, untuk memberikan akses yang cepat dan terintegrasi.
	\end{itemize}
	
	\item \textbf{Fitur Interaktif dalam LMS (Aplikasi)}:
	\begin{itemize}
		\item \textbf{Gap:} LMS tidak memiliki fitur interaktif untuk pembelajaran daring.
		\item \textbf{Solusi:} Integrasi LMS dengan alat video konferensi dan forum diskusi real-time.
		\item \textbf{Implementasi:} Setelah SSO dan portal mahasiswa terpadu tersedia untuk memastikan akses terintegrasi.
		\item \textbf{Alasan Urutan:} Fitur interaktif LMS lebih efektif jika pengguna sudah memiliki akses mudah melalui portal mahasiswa dan SSO.
	\end{itemize}
	
	\item \textbf{Komunikasi dan Kolaborasi Dosen-Mahasiswa (Bisnis)}:
	\begin{itemize}
		\item \textbf{Gap:} Terbatasnya komunikasi dan kolaborasi antara dosen dan mahasiswa.
		\item \textbf{Solusi:} Mengimplementasikan platform komunikasi terintegrasi seperti Microsoft Teams atau Slack.
		\item \textbf{Implementasi:} Setelah LMS dan portal mahasiswa tersedia untuk mendukung interaksi lebih dalam.
		\item \textbf{Alasan Urutan:} Kolaborasi lebih efektif setelah pengguna memiliki akses ke LMS dan portal mahasiswa untuk aktivitas pembelajaran utama.
	\end{itemize}
	
	\item \textbf{Pelaporan dan Analisis Data Mahasiswa (Data)}:
	\begin{itemize}
		\item \textbf{Gap:} Kurangnya sistem pelaporan komprehensif untuk analisis data mahasiswa.
		\item \textbf{Solusi:} Implementasi sistem Business Intelligence (BI) dengan laporan yang dapat dikustomisasi.
		\item \textbf{Implementasi:} Setelah fitur interaktif LMS tersedia untuk mendukung data yang dibutuhkan.
		\item \textbf{Alasan Urutan:} Sistem BI memerlukan data dari LMS dan aktivitas portal untuk memberikan analisis menyeluruh.
	\end{itemize}
	
	\item \textbf{Pengumpulan dan Pengarsipan Tugas (Aplikasi)}:
	\begin{itemize}
		\item \textbf{Gap:} LMS tidak mendukung pengumpulan dan pengarsipan tugas otomatis.
		\item \textbf{Solusi:} Menambahkan modul manajemen tugas dan arsip dalam LMS.
		\item \textbf{Implementasi:} Bersamaan dengan atau setelah fitur interaktif dalam LMS.
		\item \textbf{Alasan Urutan:} Modul ini mendukung pengalaman pembelajaran lebih lengkap dan membantu mengumpulkan data untuk BI.
	\end{itemize}
	
	\item \textbf{Pelatihan Dosen untuk Hybrid Learning (Bisnis)}:
	\begin{itemize}
		\item \textbf{Gap:} Kebutuhan untuk pelatihan dosen dalam penggunaan alat hybrid learning.
		\item \textbf{Solusi:} Melaksanakan program pelatihan berkala yang mencakup penggunaan LMS, video konferensi, dan alat interaktif lainnya.
		\item \textbf{Implementasi:} Setelah platform komunikasi dan fitur LMS telah tersedia.
		\item \textbf{Alasan Urutan:} Pelatihan akan lebih efektif jika dosen telah memiliki akses ke platform dan fitur yang akan mereka gunakan.
	\end{itemize}
	
	\item \textbf{Akses Data Real-Time untuk Evaluasi Kinerja (Data)}:
	\begin{itemize}
		\item \textbf{Gap:} Terbatasnya akses data real-time untuk evaluasi kinerja mahasiswa.
		\item \textbf{Solusi:} Integrasi sistem evaluasi kinerja dengan dashboard analitik.
		\item \textbf{Implementasi:} Dilakukan setelah sistem BI terintegrasi dengan LMS.
		\item \textbf{Alasan Urutan:} Data evaluasi real-time hanya dapat diakses jika BI telah mengumpulkan dan mengintegrasikan data dari LMS dan modul evaluasi kinerja.
	\end{itemize}
\end{enumerate}


\begin{table}[ht!]
	\centering
	\begin{tabular}{|p{.05\textwidth}|p{.3\textwidth}|p{.3\textwidth}|p{.2\textwidth}|}
		\hline
		\textbf{No.} & \textbf{Komponen} & \textbf{Kegiatan} & \textbf{Waktu Implementasi} \\
		\hline
		1 & Kapasitas Server untuk Akses Serentak (Teknologi) & 
		Upgrade server atau migrasi ke solusi cloud untuk kapasitas yang lebih besar & 
		Q1 - Prioritas awal untuk mendukung stabilitas akses \\
		\hline
		2 & Jaringan Wi-Fi Kampus (Teknologi) & 
		Peningkatan kapasitas Wi-Fi di seluruh kampus dengan perangkat router baru & 
		Q1 - Dilakukan setelah server ditingkatkan \\
		\hline
		3 & Sistem Autentikasi Tunggal (Single Sign-On) (Teknologi) & 
		Implementasi SSO untuk akses LMS, portal akademik, dan aplikasi lainnya & 
		Q2 - Setelah server dan Wi-Fi siap \\
		\hline
		4 & Integrasi Layanan Mahasiswa (Bisnis) & 
		Mengembangkan portal mahasiswa terpadu dengan akses ke informasi akademik, keuangan, dan administrasi & 
		Q2 - Setelah SSO siap untuk akses terpusat \\
		\hline
		5 & Fitur Interaktif dalam LMS (Aplikasi) & 
		Integrasi LMS dengan alat video konferensi dan forum diskusi real-time & 
		Q3 - Setelah SSO dan portal mahasiswa tersedia \\
		\hline
		6 & Komunikasi dan Kolaborasi Dosen-Mahasiswa (Bisnis) & 
		Implementasi platform komunikasi terintegrasi (misalnya, Microsoft Teams atau Slack) & 
		Q3 - Setelah LMS dan portal siap untuk mendukung interaksi \\
		\hline
		7 & Pelaporan dan Analisis Data Mahasiswa (Data) & 
		Implementasi sistem Business Intelligence (BI) dengan laporan yang dapat dikustomisasi & 
		Q4 - Setelah fitur interaktif LMS tersedia \\
		\hline
		8 & Pengumpulan dan Pengarsipan Tugas (Aplikasi) & 
		Menambahkan modul manajemen tugas dan arsip dalam LMS & 
		Q4 - Bersamaan atau setelah fitur interaktif LMS \\
		\hline
		9 & Pelatihan Dosen untuk Hybrid Learning (Bisnis) & 
		Program pelatihan untuk penggunaan LMS, video konferensi, dan alat interaktif & 
		Q4 - Setelah platform komunikasi dan fitur LMS siap \\
		\hline
		10 & Akses Data Real-Time untuk Evaluasi Kinerja (Data) & 
		Integrasi sistem evaluasi kinerja dengan dashboard analitik & 
		Q1 Tahun 2 - Setelah sistem BI tersedia \\
		\hline
	\end{tabular}
	\caption{Rencana Migrasi untuk Implementasi Arsitektur Hybrid Learning}
	\label{tab:rencana_migrasi_arsitektur_hybrid_learning}
\end{table}


\section{Rencana Migrasi Arsitektur Hybrid Learning}
\label{sec:rencana_migrasi}
Contoh rencana migrasi arsitektur untuk hybrid learning dapat dilihat pada Tabel \ref{tab:rencana_migrasi_arsitektur_hybrid_learning}.

\section{Ringkasan}
\begin{enumerate}
	\item Fase Peluang dan Solusi memungkinkan kita untuk mengidentifikasi dan mengembangkan solusi arsitektur yang efektif dan efisien.
	\item Ini membantu dalam merencanakan implementasi arsitektur secara terperinci.
\end{enumerate}

\section{Aktivitas Kelas dan Tugas}
Buatlah Peta Jalan (Subbab \ref{sec:peta_jalan}) dan Rencana Migrasi (Subbab \ref{sec:rencana_migrasi}) untuk mewujudkan arsitektur \textbf{As-Is} atau kapabilitas yang Anda pilih.