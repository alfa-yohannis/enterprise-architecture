\chapter{Fase F: Rencana Migrasi}

\section{Tujuan}
\begin{enumerate}
	\item Menyelesaikan Roadmap Arsitektur dan mendukung Rencana Implementasi dan Migrasi.
	\item Memastikan bahwa Rencana Implementasi dan Migrasi terkoordinasi dengan pendekatan perubahan organisasi.
	\item Memastikan bahwa nilai bisnis dan biaya dari paket kerja (\textit{work packages}) serta Arsitektur Transisi dipahami oleh para pemangku kepentingan utama.
\end{enumerate}

\section{Input}
\begin{enumerate}
	\item Permintaan untuk Pekerjaan Arsitektur
	\item Rencana Komunikasi
	\item Model Organisasi untuk Arsitektur Perusahaan
	\item Model dan Kerangka Tata Kelola
	\item Kerangka Arsitektur yang Disesuaikan
	\item Pernyataan Pekerjaan Arsitektur
	\item Visi Arsitektur
	\item Repositori Arsitektur
	\item Draf Spesifikasi Kebutuhan Arsitektur
	\item Draf Dokumen Definisi Arsitektur, termasuk:
	\begin{enumerate}
		\item Arsitektur Transisi, jika ada
	\end{enumerate}
	\item Permintaan Perubahan untuk program dan proyek yang ada
	\item Roadmap Arsitektur, termasuk:
	\begin{enumerate}
		\item Identifikasi paket kerja
		\item Identifikasi Arsitektur Transisi
		\item Matriks Faktor Implementasi
	\end{enumerate}
	\item Penilaian Kapabilitas
	\item Rencana Implementasi dan Migrasi (garis besar)
\end{enumerate}

\section{Langkah-langkah}
\begin{enumerate}
	\item Konfirmasi interaksi kerangka manajemen untuk Rencana Implementasi dan Migrasi.
	\item Berikan nilai bisnis untuk setiap paket kerja.
	\item Estimasi kebutuhan sumber daya, jadwal proyek, dan ketersediaan/penyediaan.
	\item Prioritaskan proyek migrasi melalui evaluasi biaya/manfaat dan validasi risiko.
	\item Konfirmasi Roadmap Arsitektur dan perbarui Dokumen Definisi Arsitektur.
	\item Selesaikan Rencana Implementasi dan Migrasi.
	\item Selesaikan siklus pengembangan arsitektur dan dokumentasikan pembelajaran.
\end{enumerate}

\subsection*{Contoh:}

\begin{enumerate}
	\item \textbf{Konfirmasi interaksi kerangka manajemen untuk Rencana Implementasi dan Migrasi:} 
	Memastikan bahwa rencana implementasi selaras dengan kerangka manajemen universitas untuk mendukung keberhasilan proyek. Ini termasuk koordinasi dengan departemen teknologi informasi, akademik, dan administrasi agar semua bagian universitas bergerak dalam arah yang sama selama proses migrasi ke lingkungan pembelajaran hybrid.
	
	\item \textbf{Berikan nilai bisnis untuk setiap paket kerja:} 
	Nilai bisnis dari setiap paket kerja dinilai berdasarkan dampaknya terhadap tujuan pendidikan. Misalnya, paket kerja yang mencakup pengembangan platform pembelajaran online yang terintegrasi memiliki nilai tinggi karena meningkatkan aksesibilitas dan fleksibilitas pembelajaran bagi mahasiswa. Sebaliknya, proyek peningkatan kapasitas jaringan kampus dapat dinilai tinggi jika mendukung kebutuhan akses internet yang stabil bagi dosen dan mahasiswa.
	
	\item \textbf{Estimasi kebutuhan sumber daya, jadwal proyek, dan ketersediaan/penyediaan:} 
	Estimasi ini mencakup anggaran dan alokasi sumber daya, seperti staf IT dan tim pengembangan kurikulum, dengan mempertimbangkan jumlah pekerjaan. Misalnya, implementasi Learning Management System (LMS) membutuhkan alokasi waktu untuk pelatihan dosen dan mahasiswa, sementara peningkatan jaringan memerlukan persediaan perangkat keras yang tersedia pada jadwal yang direncanakan.
	
	\item \textbf{Prioritaskan proyek migrasi melalui evaluasi biaya/manfaat dan validasi risiko:} 
	Prioritas diberikan pada proyek yang memberikan manfaat langsung untuk kebutuhan pembelajaran hybrid. Misalnya, peningkatan infrastruktur server agar mampu menangani kelas daring dan tatap muka mungkin memiliki prioritas tinggi, sementara proyek integrasi fitur analitik bisa ditunda hingga kebutuhan dasar terpenuhi.
	
	\item \textbf{Konfirmasi Roadmap Arsitektur dan perbarui Dokumen Definisi Arsitektur:} 
	Roadmap arsitektur diperiksa ulang untuk memastikan bahwa semua langkah menuju sistem hybrid telah diidentifikasi dengan baik. Dokumen ini diperbarui untuk mencakup perubahan pada desain platform yang ditemukan selama proses migrasi, seperti penambahan sistem kolaborasi virtual atau integrasi dengan platform konferensi video.
	
	\item \textbf{Selesaikan Rencana Implementasi dan Migrasi:} 
	Dokumen ini disusun untuk memasukkan daftar proyek lengkap, jadwal, dan rincian anggaran. Misalnya, tahapan implementasi LMS dirinci dengan jadwal pelatihan dosen dan sosialisasi kepada mahasiswa untuk mempermudah transisi ke lingkungan belajar hybrid.
	
	\item \textbf{Selesaikan siklus pengembangan arsitektur dan dokumentasikan pembelajaran:} 
	Sesi evaluasi diadakan untuk menilai proses transisi, mendokumentasikan tantangan, dan solusi yang berhasil diterapkan. Hasil evaluasi, seperti efektivitas penggunaan platform digital dalam pembelajaran, direkam untuk memperbaiki siklus migrasi di masa mendatang.
\end{enumerate}

\section{Output}
\begin{enumerate}
	\item Rencana Implementasi dan Migrasi (rinci)
	\item Dokumen Definisi Arsitektur final, termasuk:
	\begin{enumerate}
		\item Arsitektur Transisi yang disempurnakan, jika ada
	\end{enumerate}
	\item Spesifikasi Kebutuhan Arsitektur yang disempurnakan
	\item Roadmap Arsitektur yang disempurnakan
	\item Blok Bangunan Arsitektur yang dapat digunakan kembali (ABBs)
	\item Permintaan untuk pekerjaan arsitektur baru untuk siklus ADM berikutnya (jika ada)
	\item Model Tata Kelola Implementasi
	\item Permintaan Perubahan untuk Kapabilitas Arsitektur yang dihasilkan dari pembelajaran.
\end{enumerate}

\section{Rencana Implementasi dan Migrasi}

Rencana Implementasi dan Migrasi dikembangkan dalam Fase E dan F, menyediakan jadwal proyek untuk implementasi Arsitektur Target. Rencana ini mencakup proyek yang dapat dieksekusi yang dikelompokkan ke dalam portofolio dan program organisasi. 

Konten umum Rencana Implementasi dan Migrasi meliputi:
\begin{itemize}
	\item Strategi Implementasi dan Migrasi:
	\begin{itemize}
		\item Arah strategis implementasi
		\item Pendekatan urutan implementasi
	\end{itemize}
	\item Rincian proyek dan portofolio:
	\begin{itemize}
		\item Alokasi paket kerja ke proyek dan portofolio
		\item Kapabilitas yang disediakan oleh proyek
		\item Milestone dan jadwal
		\item Struktur rincian kerja
	\end{itemize}
	\item Dokumen proyek yang mungkin disertakan:
	\begin{itemize}
		\item Paket kerja yang disertakan
		\item Nilai bisnis
		\item Risiko, isu, asumsi, dan ketergantungan
		\item Persyaratan sumber daya dan biaya
		\item Manfaat migrasi (termasuk pemetaan ke kebutuhan bisnis)
		\item Perkiraan biaya opsi migrasi
	\end{itemize}
\end{itemize}

\subsection*{Contoh:}

\begin{itemize}
	\item \textbf{Strategi Implementasi dan Migrasi:}
	
	\begin{itemize}
		\item \textbf{Arah strategis implementasi:} Dalam konteks hybrid learning, arah strategis mungkin memprioritaskan penyediaan akses pembelajaran online yang terintegrasi untuk semua mata kuliah. Misalnya, universitas dapat fokus pada pengembangan sistem yang memungkinkan kelas diakses baik secara daring maupun luring secara bersamaan. Arah lain bisa berupa peningkatan layanan pembelajaran dengan memberikan akses mudah ke materi dan evaluasi melalui LMS yang terintegrasi.
		
		\item \textbf{Pendekatan urutan implementasi:} Untuk optimalisasi, implementasi bisa dimulai dengan penyediaan infrastruktur jaringan dan server sebelum beralih ke pengembangan fitur-fitur kolaborasi online. Contoh lain adalah memastikan sistem konferensi video siap pakai sebelum mengembangkan fitur penilaian otomatis di LMS.
	\end{itemize}
	
	\item \textbf{Rincian Proyek dan Portofolio:}
	
	\begin{itemize}
		\item \textbf{Alokasi paket kerja ke proyek dan portofolio:} Paket kerja seperti "Pengembangan Fitur Diskusi Online" dapat dialokasikan ke proyek LMS, sementara "Peningkatan Bandwidth Kampus" masuk ke portofolio infrastruktur teknologi. Contoh lainnya, “Pelatihan Dosen untuk Penggunaan LMS” dapat dialokasikan ke proyek adopsi pengguna.
		
		\item \textbf{Kapabilitas yang disediakan oleh proyek:} Proyek untuk menyediakan fitur video streaming pada LMS meningkatkan kapabilitas kelas hybrid, sehingga memungkinkan dosen mengajar langsung dari berbagai lokasi. Contoh lain adalah pengembangan fitur analitik di LMS yang memungkinkan evaluasi kinerja mahasiswa secara real-time.
		
		\item \textbf{Milestone dan jadwal:} Milestone seperti "Penyelesaian Pengujian Prototipe LMS" menandai kesiapan untuk uji coba bagi pengguna awal. Milestone lainnya, "Integrasi Sistem Pembelajaran Online dengan Perpustakaan Digital," memungkinkan mahasiswa mengakses materi pembelajaran lebih mudah.
		
		\item \textbf{Struktur rincian kerja (WBS):} Dalam proyek hybrid learning, WBS mungkin mencakup konfigurasi LMS, integrasi dengan aplikasi video konferensi, dan pelatihan dosen. Contoh lain, untuk proyek peningkatan Wi-Fi kampus, WBS bisa mencakup pemasangan router, pengaturan jaringan, dan uji coba akses mahasiswa.
	\end{itemize}
	
	\item \textbf{Dokumen Proyek yang Mungkin Disertakan:}
	
	\begin{itemize}
		\item \textbf{Paket kerja yang disertakan:} Setiap proyek perlu mendetailkan paket kerja yang disertakan, seperti "Pelatihan Penggunaan LMS (\textit{Learning Management System})" untuk pelatihan dosen dan mahasiswa. Contoh lain adalah paket kerja "Pembuatan Panduan LMS," yang memudahkan pengguna mengakses fitur-fitur LMS.
		
		\item \textbf{Nilai bisnis:} Mengukur nilai bisnis proyek hybrid learning penting untuk keberlanjutan universitas. Misalnya, LMS yang memungkinkan pembelajaran jarak jauh memperluas jangkauan universitas kepada lebih banyak mahasiswa. Contoh lain adalah peningkatan infrastruktur Wi-Fi, yang meningkatkan pengalaman belajar mahasiswa secara keseluruhan.
		
		\item \textbf{Risiko, isu, asumsi, dan ketergantungan:} Risiko seperti keterbatasan ketersediaan perangkat yang dapat menghambat proyek harus diantisipasi. Contoh lain adalah ketergantungan pada vendor LMS untuk menyediakan fitur kustom, yang bisa memperpanjang waktu implementasi.
		
		\item \textbf{Persyaratan sumber daya dan biaya:} Dalam pengembangan hybrid learning, biaya untuk lisensi software dan bandwidth mungkin diperlukan. Contoh lain adalah biaya tambahan untuk pelatihan staf TI dalam menangani aplikasi pembelajaran baru.
		
		\item \textbf{Manfaat migrasi (termasuk pemetaan ke kebutuhan bisnis):} Manfaat dari LMS mencakup fleksibilitas pembelajaran, sesuai dengan kebutuhan mahasiswa yang belajar jarak jauh. Contoh lainnya, penerapan platform analitik untuk evaluasi kinerja mahasiswa mendukung kebutuhan universitas untuk penilaian akademik yang akurat.
		
		\item \textbf{Perkiraan biaya opsi migrasi:} Misalnya, biaya migrasi LMS secara bertahap dibandingkan dengan migrasi langsung, yang dapat lebih terjangkau secara bertahap namun berdampak pada waktu peluncuran. Contoh lainnya adalah perbandingan antara peningkatan jaringan internal dan outsourcing yang memiliki perbedaan dalam kontrol dan biaya.
	\end{itemize}
\end{itemize}

\section{Dokumen Definisi Arsitektur, termasuk Arsitektur Transisi}

Dokumen Definisi Arsitektur diselesaikan pada fase ini. Dokumen ini mencakup deskripsi detail mengenai Arsitektur Target. Jika implementasi Arsitektur Target memerlukan pendekatan bertahap, satu atau lebih Arsitektur Transisi akan didefinisikan dalam dokumen ini. Arsitektur Transisi menggambarkan kondisi signifikan arsitektural antara Arsitektur Dasar dan Arsitektur Target, yang digunakan untuk mendefinisikan tahap-tahap yang diperlukan untuk mencapai Arsitektur Target.

Konten umum dalam Arsitektur Transisi meliputi:
\begin{itemize}
	\item Definisi status transisi
	\item Arsitektur Bisnis untuk setiap status transisi
	\item Arsitektur Data untuk setiap status transisi
	\item Arsitektur Aplikasi untuk setiap status transisi
	\item Arsitektur Teknologi untuk setiap status transisi
\end{itemize}

\subsection*{Contoh:}

\begin{itemize}
	\item \textbf{Definisi status transisi:} Status transisi merujuk pada tahapan signifikan antara Arsitektur Dasar dan Arsitektur Target, yang menguraikan langkah-langkah yang diperlukan untuk setiap tahapan. Misalnya, universitas dapat memulai dengan menerapkan platform penyimpanan konten pembelajaran secara online sebagai tahap awal, sebelum beralih ke infrastruktur cloud yang sepenuhnya terintegrasi. Contoh lainnya adalah memulai dengan menyediakan akses ke materi kursus secara online sebelum menambahkan fitur-fitur kolaborasi seperti ruang diskusi virtual dan evaluasi daring.
	
	\item \textbf{Arsitektur Bisnis untuk setiap status transisi:} Arsitektur Bisnis dalam status transisi menguraikan perubahan struktural atau prosedural untuk mendukung model hybrid. Sebagai contoh, dalam transisi ke sistem manajemen pembelajaran digital, struktur administrasi akademik mungkin perlu diubah agar memungkinkan pengelolaan kelas baik secara online maupun tatap muka. Contoh lainnya adalah mengintegrasikan peran tambahan bagi staf pendukung teknologi untuk memastikan dosen dan mahasiswa mendapat bantuan teknis selama pembelajaran daring.
	
	\item \textbf{Arsitektur Data untuk setiap status transisi:} Komponen ini mendefinisikan bagaimana data dan manajemennya berkembang dalam setiap transisi. Misalnya, transisi awal bisa berupa integrasi data mahasiswa dari sistem informasi akademik ke platform LMS. Contoh lainnya adalah membangun integrasi data antara sistem perpustakaan digital dan portal mahasiswa, memungkinkan akses data yang real-time untuk semua mahasiswa.
	
	\item \textbf{Arsitektur Aplikasi untuk setiap status transisi:} Arsitektur Aplikasi menguraikan bagaimana aplikasi berinteraksi dan terintegrasi sepanjang transisi ke Arsitektur Target. Misalnya, dalam pengenalan ERP secara bertahap, tahap awal dapat mengintegrasikan modul keuangan dan administrasi, sementara tahap berikutnya mencakup sistem penilaian dan kehadiran online. Contoh lain adalah migrasi dari beberapa aplikasi pembelajaran ke platform LMS terpadu, dimulai dengan modul pengumpulan tugas sebelum menambahkan modul diskusi dan penilaian otomatis.
	
	\item \textbf{Arsitektur Teknologi untuk setiap status transisi:} Arsitektur Teknologi mendeskripsikan perubahan dalam tumpukan teknologi untuk setiap tahap transisi. Sebagai contoh, selama migrasi ke cloud, tahap awal mungkin melibatkan infrastruktur hybrid (campuran on-premises dan cloud) sampai semua sistem dapat berbasis cloud sepenuhnya. Contoh lainnya adalah peningkatan infrastruktur jaringan kampus secara bertahap, dengan tahap awal berfokus pada peningkatan kapasitas Wi-Fi, diikuti oleh penerapan langkah-langkah keamanan untuk mendukung akses yang lebih aman.
\end{itemize}

\section{Model Tata Kelola Implementasi}

Setelah arsitektur didefinisikan, diperlukan perencanaan bagaimana Arsitektur Transisi akan dikelola selama implementasi. Dalam universitas yang memiliki fungsi arsitektur yang mapan, biasanya sudah ada kerangka tata kelola, namun proses, peran, tanggung jawab, dan pengukuran mungkin perlu didefinisikan untuk tiap proyek.

Model Tata Kelola Implementasi yang dihasilkan dari Fase F memastikan bahwa proyek yang memasuki tahap implementasi juga memasuki Tata Kelola Arsitektur yang sesuai untuk Fase G.

Konten umum dari Model Tata Kelola Implementasi meliputi:
\begin{itemize}
	\item Proses tata kelola
	\item Struktur organisasi tata kelola
	\item Peran dan tanggung jawab tata kelola
	\item Titik pemeriksaan tata kelola serta kriteria keberhasilan/kegagalan
\end{itemize}

\subsection*{Contoh:}

\begin{itemize}
	\item \textbf{Proses tata kelola:} Proses tata kelola mencakup langkah-langkah untuk memastikan implementasi berjalan sesuai rencana dan terukur. Sebagai contoh, universitas dapat memiliki prosedur untuk menilai apakah modul pembelajaran baru telah memenuhi standar kualitas dan kelayakan sebelum diimplementasikan dalam kelas. Contoh lainnya adalah proses audit berkala untuk meninjau kepatuhan proyek dengan kebijakan keamanan data dan integrasi sistem.
	
	\item \textbf{Struktur organisasi tata kelola:} Struktur ini menjelaskan peran tim yang terlibat dalam pengawasan implementasi arsitektur. Misalnya, tim IT dan akademik dapat bekerja sama di bawah satu komite khusus untuk mengawasi proses transisi. Contoh lainnya adalah pembentukan komite pengarah yang terdiri dari perwakilan fakultas, dosen, dan staf IT untuk memantau dan mengarahkan strategi implementasi hybrid learning.
	
	\item \textbf{Peran dan tanggung jawab tata kelola:} Peran dan tanggung jawab menguraikan tugas spesifik dalam proses tata kelola. Misalnya, seorang Manajer Proyek dapat bertanggung jawab atas pencapaian milestone proyek, sementara seorang Koordinator LMS dapat bertugas memastikan integrasi platform LMS berjalan sesuai target. Contoh lainnya adalah menugaskan seorang Pengawas Kepatuhan yang memastikan semua aplikasi memenuhi standar keamanan dan persyaratan aksesibilitas bagi mahasiswa.
	
	\item \textbf{Titik pemeriksaan tata kelola serta kriteria keberhasilan/kegagalan:} Titik pemeriksaan adalah tahap evaluasi di mana kemajuan proyek dinilai. Sebagai contoh, titik pemeriksaan awal dapat melibatkan penilaian terhadap efektivitas pelatihan LMS bagi dosen. Contoh lainnya, pada titik pemeriksaan akhir, evaluasi mungkin mencakup keberhasilan integrasi aplikasi e-learning dengan sistem administrasi, serta identifikasi bug yang perlu diperbaiki sebelum peluncuran penuh.
\end{itemize}

\section{Contoh Rencana Implementasi dan Migrasi untuk \textit{Hybrid Learning and Teaching}}
\label{sec:contoh_rencana_implementasi_dan_migrasi}

\subsection{Strategi Implementasi dan Migrasi}
\begin{itemize}
	\item \textbf{Arah Strategis Implementasi:} 
	Prioritas implementasi adalah mendukung infrastruktur dan sistem yang memungkinkan pembelajaran hybrid secara efisien. Fokus utama adalah menyediakan akses yang setara bagi mahasiswa dan dosen dalam lingkungan online maupun fisik melalui sistem Learning Management System (LMS) dan platform kolaborasi.
	
	\item \textbf{Pendekatan Urutan Implementasi:} 
	Implementasi dimulai dari peningkatan infrastruktur teknologi, seperti konektivitas Wi-Fi dan server kampus, kemudian dilanjutkan dengan pengembangan fitur-fitur pada LMS yang mendukung kolaborasi daring, diikuti oleh integrasi platform video konferensi untuk pembelajaran synchronous.
\end{itemize}

\subsection{Rincian Proyek dan Portofolio}
\begin{itemize}
	\item \textbf{Alokasi Paket Kerja ke Proyek dan Portofolio:} 
	Paket kerja dialokasikan sebagai berikut:
	\begin{itemize}
		\item \textit{Infrastruktur Teknologi}: mencakup pemasangan dan peningkatan perangkat Wi-Fi di kampus serta upgrade kapasitas server. \textbf{Anggaran: IDR 200 juta}
		\item \textit{Pengembangan LMS}: mencakup pengembangan modul interaktif, integrasi video konferensi, dan fitur diskusi daring. \textbf{Anggaran: IDR 150 juta}
		\item \textit{Pelatihan Dosen dan Mahasiswa}: mencakup pelatihan penggunaan LMS dan platform hybrid bagi dosen dan mahasiswa. \textbf{Anggaran: IDR 50 juta}
	\end{itemize}
	
	\item \textbf{Kapabilitas yang Disediakan oleh Proyek:} 
	Proyek ini meningkatkan kapabilitas universitas dalam mengelola pembelajaran hybrid secara efektif, termasuk kapabilitas akses materi secara daring, kolaborasi real-time, serta pelacakan kinerja akademik melalui platform digital.
	
	\item \textbf{Milestone dan Jadwal:} 
	\begin{itemize}
		\item \textit{Milestone 1:} Selesainya peningkatan infrastruktur jaringan (2 bulan).
		\item \textit{Milestone 2:} Penyelesaian pengembangan modul interaktif dan fitur kolaborasi pada LMS (4 bulan).
		\item \textit{Milestone 3:} Pelatihan bagi dosen dan mahasiswa untuk penggunaan LMS dan platform hybrid (5 bulan).
	\end{itemize}
	
	\item \textbf{Struktur Rincian Kerja (WBS):} 
	WBS untuk implementasi hybrid learning mencakup:
	\begin{itemize}
		\item \textit{Infrastruktur Teknologi}: instalasi perangkat Wi-Fi, pengaturan server, dan tes jaringan.
		\item \textit{Pengembangan LMS}: konfigurasi modul, integrasi video, dan pengujian sistem.
		\item \textit{Pelatihan Pengguna}: penyusunan materi pelatihan, pelaksanaan pelatihan, dan penilaian efektivitas pelatihan.
	\end{itemize}
\end{itemize}

\subsection{Dokumen Proyek yang Mungkin Disertakan}
\begin{enumerate}
	\item \textbf{Paket Kerja yang Disertakan:} 
	Mencakup pengembangan LMS, pemasangan infrastruktur jaringan, dan pelatihan pengguna. Setiap paket kerja memiliki rincian tugas, seperti perencanaan, implementasi, dan evaluasi.
	
	\item \textbf{Nilai Bisnis:} 
	Meningkatkan efisiensi operasional dan aksesibilitas pembelajaran melalui pembelajaran hybrid. Proyek ini memungkinkan universitas menjangkau lebih banyak mahasiswa dan meningkatkan fleksibilitas pembelajaran.
	
	\item \textbf{Risiko, Isu, Asumsi, dan Ketergantungan:} 
	\begin{itemize}
		\item Risiko: Potensi gangguan dalam pelaksanaan pelatihan karena kurangnya kesiapan dosen dan mahasiswa.
		\item Isu: Keterbatasan anggaran yang memengaruhi alokasi sumber daya.
		\item Asumsi: Infrastruktur dasar dan konektivitas internet yang memadai sudah tersedia di kampus.
		\item Ketergantungan: Ketergantungan pada vendor LMS untuk kustomisasi sistem sesuai kebutuhan universitas.
	\end{itemize}
	
	\item \textbf{Persyaratan Sumber Daya dan Biaya:} 
	Estimasi sumber daya mencakup tim IT, vendor perangkat keras, dan tenaga pelatih. Total anggaran sebesar IDR 500 juta, termasuk biaya berikut:
	\begin{itemize}
		\item Infrastruktur Teknologi (Wi-Fi dan server): \textbf{IDR 200 juta}
		\item Pengembangan LMS: \textbf{IDR 150 juta}
		\item Pelatihan Dosen dan Mahasiswa: \textbf{IDR 50 juta}
		\item Biaya Kontingensi dan Operasional Tambahan: \textbf{IDR 100 juta}
	\end{itemize}
	
	\item \textbf{Manfaat Migrasi (Termasuk Pemetaan ke Kebutuhan Bisnis):} 
	Manfaat dari implementasi ini adalah meningkatkan fleksibilitas pembelajaran dan menciptakan sistem hybrid yang mudah diakses, mendukung kebutuhan mahasiswa dan dosen untuk dapat belajar dan mengajar dari berbagai lokasi.
	
	\item \textbf{Perkiraan Biaya Opsi Migrasi:} 
	Perbandingan antara migrasi bertahap dan migrasi langsung:
	\begin{itemize}
		\item Migrasi Bertahap: \textbf{IDR 600 juta}, namun lebih rendah risiko dengan pemantauan di setiap tahap.
		\item Migrasi Langsung: \textbf{IDR 500 juta}, lebih cepat namun memerlukan kesiapan total infrastruktur.
	\end{itemize}
\end{enumerate}

\section{Contoh Dokumen Definisi Arsitektur Transisi untuk \textit{Hybrid Learning and Teaching} }
\label{sec:contoh_dokumen_definisi_arsitektur_transisi}

\begin{enumerate}
	\item \textbf{Definisi Status Transisi:} 
	Status transisi ini mendeskripsikan tahapan migrasi dari sistem pembelajaran konvensional ke sistem pembelajaran hybrid yang mendukung kelas tatap muka dan daring secara simultan. Tahap ini difokuskan pada infrastruktur, aplikasi, dan prosedur yang mendukung aksesibilitas bagi mahasiswa dan dosen di kampus dan luar kampus. Target akhir dari transisi ini adalah menyediakan platform pembelajaran yang mendukung interaksi real-time dan integrasi fitur penilaian otomatis.
	
	\item \textbf{Arsitektur Bisnis untuk Setiap Status Transisi:}
	\begin{itemize}
		\item \textit{Tahap 1 - Persiapan Infrastruktur:} 
		Melibatkan penguatan struktur organisasi untuk mendukung perubahan, termasuk pembentukan tim teknologi pembelajaran dan peningkatan kapasitas tim IT.
		\item \textit{Tahap 2 - Penerapan Sistem Hybrid:} 
		Universitas membentuk kebijakan baru untuk mendukung pembelajaran hybrid, seperti panduan penggunaan Learning Management System (LMS) dan pelatihan intensif bagi dosen.
		\item \textit{Tahap 3 - Optimalisasi Operasional:} 
		Pada tahap ini, prosedur administrasi disesuaikan untuk memastikan pelacakan kehadiran dan penilaian yang konsisten antara pembelajaran daring dan luring.
	\end{itemize}
	
	\item \textbf{Arsitektur Data untuk Setiap Status Transisi:}
	\begin{itemize}
		\item \textit{Tahap 1 - Integrasi Data Mahasiswa:} 
		Data mahasiswa dari Sistem Informasi Akademik (SIA) diintegrasikan dengan LMS agar informasi kehadiran dan penilaian dapat diakses secara terpusat.
		\item \textit{Tahap 2 - Implementasi Penyimpanan Cloud:} 
		Penyimpanan data dipindahkan ke cloud, memungkinkan akses data yang lebih fleksibel dan memperkuat penyimpanan data bagi materi pembelajaran yang dibagikan secara online.
		\item \textit{Tahap 3 - Integrasi Analitik Data:} 
		Ditambahkan sistem analitik untuk memantau kinerja mahasiswa dalam real-time, menyediakan data visual bagi dosen untuk evaluasi.
	\end{itemize}
	
	\item \textbf{Arsitektur Aplikasi untuk Setiap Status Transisi:}
	\begin{itemize}
		\item \textit{Tahap 1 - Pengembangan LMS:} 
		LMS diperbarui dengan fitur-fitur dasar seperti unggahan materi, modul diskusi, dan forum tanya jawab.
		\item \textit{Tahap 2 - Integrasi Video Konferensi:} 
		Ditambahkan modul video konferensi untuk mendukung kelas synchronous secara real-time, memungkinkan dosen dan mahasiswa terhubung secara langsung meskipun berada di lokasi berbeda.
		\item \textit{Tahap 3 - Pengembangan Fitur Penilaian Otomatis:} 
		Ditambahkan fitur penilaian otomatis untuk tes dan kuis daring, sehingga mempermudah dosen dalam melakukan evaluasi dan meningkatkan efisiensi.
	\end{itemize}
	
	\item \textbf{Arsitektur Teknologi untuk Setiap Status Transisi:}
	\begin{itemize}
		\item \textit{Tahap 1 - Peningkatan Infrastruktur Jaringan:} 
		Kapasitas Wi-Fi dan jaringan kampus diperkuat untuk mendukung akses internet yang stabil di seluruh area kampus.
		\item \textit{Tahap 2 - Penyediaan Server dan Penyimpanan Cloud:} 
		Universitas melakukan upgrade server dan beralih ke cloud untuk penyimpanan data yang lebih besar, memungkinkan penyimpanan materi pembelajaran, rekaman kelas, dan akses data yang lebih cepat.
		\item \textit{Tahap 3 - Peningkatan Keamanan Teknologi:} 
		Ditambahkan firewall dan sistem deteksi intrusi untuk memastikan keamanan data, terutama untuk melindungi informasi akademik dan pribadi mahasiswa yang terintegrasi dalam LMS.
	\end{itemize}
\end{enumerate}

\section{Contoh Dokumen Model Tata Kelola Implementasi \textit{Hybrid Learning and Teaching}}
\label{sec:contoh_dokumen_model_tata_kelola}
\begin{itemize}
	\item \textbf{Proses Tata Kelola:}
	Proses tata kelola ini dirancang untuk memastikan bahwa semua aspek implementasi sistem pembelajaran hybrid sesuai dengan standar yang ditetapkan. 
	\begin{itemize}
		\item \textit{Rapat Koordinasi Mingguan:} Mengadakan rapat mingguan dengan tim IT, pengajar, dan administrasi untuk membahas kemajuan proyek, kendala, dan tindakan korektif.
		\item \textit{Evaluasi dan Uji Coba Sistem:} Setiap tahap implementasi akan melalui evaluasi fungsional untuk memastikan integrasi LMS, video konferensi, dan platform hybrid berjalan baik. Proses uji coba ini melibatkan uji performa, uji pengguna, dan pengumpulan masukan dari dosen dan mahasiswa.
		\item \textit{Penerapan Kebijakan dan Standar Keamanan Data:} Menetapkan kebijakan untuk memastikan perlindungan data pengguna, termasuk pengelolaan akses data yang terbatas dan audit berkala.
	\end{itemize}
	
	\item \textbf{Struktur Organisasi Tata Kelola:}
	Struktur organisasi tata kelola mencakup tim lintas departemen yang memastikan implementasi sistem berjalan lancar.
	\begin{itemize}
		\item \textit{Komite Pengarah Proyek Hybrid Learning:} Terdiri dari perwakilan dari Departemen Teknologi Informasi, Akademik, dan Manajemen yang bertanggung jawab memberikan arahan strategis.
		\item \textit{Tim Implementasi LMS:} Terdiri dari spesialis IT, pengembang sistem, dan ahli konten untuk menangani pengembangan dan pengujian LMS secara teknis.
		\item \textit{Tim Dukungan Pengguna:} Bertugas mendampingi dosen dan mahasiswa dalam penggunaan sistem LMS dan video konferensi, termasuk bantuan teknis, pelatihan, dan sosialisasi kebijakan.
	\end{itemize}
	
	\item \textbf{Peran dan Tanggung Jawab Tata Kelola:}
	Peran dan tanggung jawab masing-masing anggota tata kelola diatur untuk memastikan implementasi yang efektif.
	\begin{itemize}
		\item \textit{Manajer Proyek Hybrid Learning:} Bertanggung jawab atas koordinasi keseluruhan proyek, pencapaian milestone, dan komunikasi antar tim.
		\item \textit{Koordinator LMS:} Bertugas memastikan sistem LMS berjalan sesuai spesifikasi dan mendukung seluruh fitur hybrid learning.
		\item \textit{Pengawas Keamanan Data:} Memastikan standar keamanan data diterapkan di seluruh sistem untuk melindungi informasi akademik dan pribadi mahasiswa.
		\item \textit{Pelatih Dosen dan Mahasiswa:} Bertanggung jawab memberikan pelatihan tentang penggunaan LMS dan aplikasi video konferensi serta menyediakan panduan praktis.
	\end{itemize}
	
	\item \textbf{Titik Pemeriksaan Tata Kelola serta Kriteria Keberhasilan/Kegagalan:}
	Titik pemeriksaan atau checkpoint ini dilakukan pada setiap tahap implementasi dengan kriteria evaluasi yang telah ditetapkan.
	\begin{itemize}
		\item \textit{Checkpoint 1 - Penyelesaian Infrastruktur Dasar:} Infrastruktur Wi-Fi dan server telah siap, dengan keberhasilan ditentukan oleh kestabilan jaringan di kampus (target: konektivitas >95\% uptime).
		\item \textit{Checkpoint 2 - Integrasi LMS dan Video Konferensi:} Sistem LMS dan aplikasi video konferensi telah terintegrasi, dengan keberhasilan diukur melalui tes pengguna yang mencakup >90\% kelancaran tanpa gangguan.
		\item \textit{Checkpoint 3 - Pelatihan Pengguna:} Dosen dan mahasiswa telah mengikuti pelatihan LMS, dengan keberhasilan ditentukan dari hasil survei kepuasan pengguna yang mencapai >85\%.
		\item \textit{Checkpoint 4 - Uji Fungsional Akhir:} Dilakukan pengujian akhir sebelum peluncuran, dengan keberhasilan diukur berdasarkan persentase fungsi utama yang beroperasi tanpa error (target: >95\%).
	\end{itemize}
\end{itemize}

\section{Ringkasan}
\begin{enumerate}
	\item Fase F berfokus pada perencanaan migrasi dari Arsitektur Dasar ke Arsitektur Target, mendukung pengembangan sistem hybrid learning di universitas.
	\item Menghasilkan Dokumen Definisi Arsitektur final, Roadmap Arsitektur, dan Rencana Implementasi dan Migrasi yang rinci.
	\item Setelah fase ini selesai, persiapan untuk implementasi arsitektur hybrid learning telah selesai.
\end{enumerate}




\section{Aktivitas Kelas dan Tugas}

Buatlah Rencana Implementasi dan Migrasi (Subbab \ref{sec:contoh_rencana_implementasi_dan_migrasi}), Definisi Arsitektur Transisi (jika diperlukan) (Subbab \ref{sec:contoh_dokumen_definisi_arsitektur_transisi}), dan Model Tata Kelola (Subbbab \ref{sec:contoh_dokumen_model_tata_kelola}) untuk mengimplementasikan Kapabilitas yang Anda pilih.