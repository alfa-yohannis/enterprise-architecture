\chapter{Arsitektur Teknologi}

\section{Tujuan}
Tujuan dari fase ini meliputi:
\begin{enumerate}
	\item Mengembangkan Target Arsitektur Teknologi untuk mendukung arsitektur aplikasi dan data, sejalan dengan Visi Arsitektur dan kebutuhan pemangku kepentingan.
	\item Mengidentifikasi komponen peta jalan arsitektur kandidat dengan menganalisis kesenjangan antara Arsitektur Teknologi Baseline dan Target.
\end{enumerate}

\section{Masukan}
Berikut adalah masukan untuk fase ini:
\begin{enumerate}
	\item Permintaan untuk Pekerjaan Arsitektur
	\item Penilaian Kapabilitas
	\item Rencana Komunikasi
	\item Model Organisasi untuk Arsitektur Perusahaan
	\item Kerangka Arsitektur yang Disesuaikan
	\item Prinsip Teknologi (jika ada)
	\item Pernyataan Pekerjaan Arsitektur
	\item Visi Arsitektur
	\item Draf Dokumen Definisi Arsitektur:
	\begin{itemize}
		\item Arsitektur Bisnis Baseline (rinci)
		\item Arsitektur Bisnis Target (rinci)
		\item Arsitektur Data Baseline (rinci)
		\item Arsitektur Data Target (rinci)
		\item Arsitektur Aplikasi Baseline (rinci)
		\item Arsitektur Aplikasi Target (rinci)
		\item Arsitektur Teknologi Baseline (tingkat tinggi)
		\item Arsitektur Teknologi Target (tingkat tinggi)
	\end{itemize}
	\item Repositori Arsitektur
	\item Draf Spesifikasi Kebutuhan Arsitektur:
	\begin{itemize}
		\item Hasil Analisis Kesenjangan
		\item Kebutuhan teknis yang relevan
	\end{itemize}
	\item Komponen Arsitektur Bisnis, Data, dan Aplikasi untuk Peta Jalan Arsitektur
\end{enumerate}

\section{Prinsip Teknologi}
Prinsip dasar teknologi yang membimbing fase ini adalah:
\begin{enumerate}
	\item \textbf{Perubahan Berdasarkan Kebutuhan}: Perubahan pada aplikasi dan teknologi hanya dilakukan sebagai respons terhadap kebutuhan bisnis.
	\item \textbf{Manajemen Perubahan yang Responsif}: Perubahan pada lingkungan informasi perusahaan diterapkan tepat waktu.
	\item \textbf{Kontrol Keberagaman Teknologi}: Keberagaman teknologi diminimalkan untuk mengontrol biaya pemeliharaan dan meningkatkan kompatibilitas sistem.
	\item \textbf{Interoperabilitas}: Perangkat lunak dan perangkat keras harus mematuhi standar yang mendukung interoperabilitas data, aplikasi, dan teknologi.
\end{enumerate}

\section{Langkah-langkah}
Langkah-langkah berikut dilakukan untuk mengembangkan Arsitektur Teknologi:
\begin{enumerate}
	\item Memilih model referensi, sudut pandang, dan alat.
	\item Mengembangkan Deskripsi Arsitektur Teknologi Baseline.
	\item Mengembangkan Deskripsi Arsitektur Teknologi Target.
	\item Melakukan Analisis Kesenjangan.
	\item Mendefinisikan komponen peta jalan kandidat.
	\item Menyelesaikan dampak di seluruh Lanskap Arsitektur.
	\item Melakukan tinjauan formal dengan pemangku kepentingan.
	\item Menyelesaikan Arsitektur Teknologi.
	\item Membuat Dokumen Definisi Arsitektur.
\end{enumerate}

\subsection*{Contoh:}

\begin{itemize}
	\item \textbf{Memilih Model Referensi, Sudut Pandang, dan Alat}: Tahap ini melibatkan pemilihan model referensi dan sudut pandang yang paling sesuai untuk merancang arsitektur teknologi yang mendukung hybrid learning. \emph{Contoh:} Memilih model referensi TOGAF untuk institusi pendidikan dan alat seperti Microsoft Visio atau Lucidchart untuk membuat diagram arsitektur, serta memilih Moodle atau Google Classroom sebagai LMS utama.
	
	\item \textbf{Mengembangkan Deskripsi Arsitektur Teknologi Baseline}: Deskripsi ini mencatat arsitektur teknologi yang ada untuk mendukung pembelajaran hybrid, termasuk perangkat keras, perangkat lunak, dan infrastruktur jaringan. \emph{Contoh:} Menyusun inventaris server yang digunakan untuk LMS dan aplikasi video konferensi serta menggambarkan perangkat keras di ruang kelas seperti proyektor dan speaker. Contoh lainnya adalah mencatat platform video konferensi yang digunakan, seperti Zoom atau Microsoft Teams.
	
	\item \textbf{Mengembangkan Deskripsi Arsitektur Teknologi Target}: Deskripsi target mendefinisikan kondisi ideal dari arsitektur teknologi yang diinginkan untuk mendukung hybrid learning secara optimal. \emph{Contoh:} Menentukan kebutuhan untuk integrasi penuh antara LMS dan aplikasi video konferensi untuk pengalaman yang lebih efisien bagi dosen dan mahasiswa, serta menambah bandwidth jaringan untuk mendukung kelas daring yang lebih lancar. Contoh lain adalah memperbarui perangkat keras di ruang kelas untuk mendukung streaming video resolusi tinggi.
	
	\item \textbf{Melakukan Analisis Kesenjangan}: Analisis ini membandingkan kondisi teknologi baseline dengan target untuk mengidentifikasi kekurangan yang perlu diatasi. \emph{Contoh:} Menemukan bahwa kapasitas server saat ini tidak memadai untuk mendukung akses serentak dari banyak pengguna dan bahwa integrasi sistem LMS dengan alat penilaian masih kurang memadai. Contoh lainnya adalah kebutuhan untuk meng-upgrade kamera di ruang kelas untuk mendukung sesi tatap muka dan daring secara bersamaan.
	
	\item \textbf{Mendefinisikan Komponen Peta Jalan Kandidat}: Langkah ini mencakup identifikasi komponen teknologi yang perlu ditingkatkan atau ditambahkan dalam jangka waktu tertentu. \emph{Contoh:} Menyusun peta jalan untuk meningkatkan kapasitas server dan memperbarui perangkat tambahan di ruang kelas. Contoh lain adalah menjadwalkan pelatihan bagi dosen tentang penggunaan aplikasi yang diintegrasikan dengan LMS.
	
	\item \textbf{Menyelesaikan Dampak di Seluruh Lanskap Arsitektur}: Mengevaluasi bagaimana perubahan arsitektur teknologi akan memengaruhi sistem lain di universitas. \emph{Contoh:} Menilai dampak peningkatan kapasitas server LMS terhadap jaringan administrasi universitas. Contoh lain adalah memastikan bahwa sistem pembayaran biaya kuliah tidak terganggu saat infrastruktur jaringan diperbarui.
	
	\item \textbf{Melakukan Tinjauan Formal dengan Pemangku Kepentingan}: Tinjauan ini melibatkan berbagai pemangku kepentingan untuk memastikan bahwa arsitektur yang dirancang memenuhi kebutuhan mereka. \emph{Contoh:} Melakukan pertemuan dengan dosen untuk mendapatkan umpan balik tentang antarmuka LMS, dan bertemu dengan mahasiswa untuk memahami kebutuhan mereka terkait akses ke materi kuliah. 
	
	\item \textbf{Menyelesaikan Arsitektur Teknologi}: Setelah tinjauan, arsitektur diselesaikan berdasarkan umpan balik yang diterima dan siap untuk diimplementasikan. \emph{Contoh:} Menetapkan spesifikasi akhir dari perangkat keras yang dibutuhkan di ruang kelas, dan mengonfirmasi penyediaan bandwidth yang cukup. Contoh lain adalah memfinalisasi integrasi LMS dengan sistem penilaian daring.
	
	\item \textbf{Membuat Dokumen Definisi Arsitektur}: Dokumen ini merangkum komponen arsitektur, termasuk baseline, target, analisis kesenjangan, dan peta jalan yang telah ditentukan. \emph{Contoh:} Menyusun dokumen yang memuat spesifikasi server, perangkat video konferensi, dan perangkat ruang kelas serta jadwal implementasi. Contoh lainnya adalah menyertakan diagram integrasi LMS dan sistem penilaian untuk referensi teknis.
\end{itemize}


\section{Keluaran}
Keluaran utama dari fase ini meliputi:
\begin{enumerate}
	\item Pernyataan Pekerjaan Arsitektur yang diperbarui, jika diperlukan.
	\item Prinsip teknologi yang divalidasi atau baru.
	\item Draf Dokumen Definisi Arsitektur dengan pembaruan.
	\item Draf Spesifikasi Kebutuhan Arsitektur yang diperbarui.
	\item Komponen Arsitektur Teknologi untuk Peta Jalan Arsitektur.
\end{enumerate}

\section{Katalog, Matriks, dan Diagram}
Fase Arsitektur Teknologi mencakup berbagai artefak:
\begin{itemize}
	\item \textbf{Katalog}: Katalog Standar Teknologi, Portofolio Teknologi.
	\item \textbf{Matriks}: Matriks Aplikasi/Teknologi.
	\item \textbf{Diagram}: Diagram Lingkungan dan Lokasi, Diagram Dekonstruksi Platform, Diagram Pemrosesan, Diagram Jaringan Komputasi/Perangkat Keras, Diagram Teknik Komunikasi.
\end{itemize}

\subsection*{Contoh:}

\begin{itemize}
	\item \textbf{Katalog}: Katalog dalam konteks hybrid learning di universitas mencakup standar teknologi dan portofolio teknologi yang mendukung pembelajaran jarak jauh dan tatap muka.
	\begin{itemize}
		\item \textbf{Katalog Standar Teknologi}: Katalog ini mendokumentasikan standar perangkat dan perangkat lunak yang direkomendasikan untuk dosen dan mahasiswa, seperti perangkat lunak video konferensi (misalnya, Zoom, Microsoft Teams) dan Learning Management System (LMS) seperti Moodle atau Google Classroom.
		\item \textbf{Portofolio Teknologi}: Portofolio teknologi mencakup daftar perangkat keras dan perangkat lunak yang digunakan universitas untuk mendukung kegiatan hybrid learning. Misalnya, portofolio dapat mencakup daftar komputer, proyektor, platform LMS, dan aplikasi penilaian online yang digunakan di seluruh fakultas.
	\end{itemize}
	
	\item \textbf{Matriks}: Matriks dalam arsitektur hybrid learning menunjukkan hubungan antara aplikasi yang digunakan dan teknologi pendukungnya.
	\begin{itemize}
		\item \textbf{Matriks Aplikasi/Teknologi}: Matriks ini memetakan aplikasi pembelajaran seperti sistem LMS dan aplikasi konferensi video dengan perangkat keras dan sistem operasi yang kompatibel. Misalnya, matriks menunjukkan aplikasi Zoom yang dapat digunakan di berbagai perangkat seperti komputer lab atau laptop pribadi dosen dan mahasiswa.
	\end{itemize}
	
	\item \textbf{Diagram}: Diagram membantu memvisualisasikan distribusi, proses, dan jaringan yang mendukung pembelajaran hybrid di universitas.
	\begin{itemize}
		\item \textbf{Diagram Lingkungan dan Lokasi}: Diagram ini menunjukkan lokasi fisik dan lingkungan dari perangkat dan teknologi yang digunakan untuk pembelajaran hybrid, seperti ruang kelas yang dilengkapi kamera dan mikrofon, serta akses internet di berbagai area kampus.
		\item \textbf{Diagram Dekonstruksi Platform}: Diagram ini menggambarkan komponen-komponen dari platform LMS atau platform video konferensi. Misalnya, diagram Moodle mencakup komponen untuk manajemen kelas, penyimpanan konten, dan akses pengguna.
		\item \textbf{Diagram Pemrosesan}: Diagram ini menunjukkan alur data dalam sistem pembelajaran, seperti bagaimana data interaksi mahasiswa di LMS diolah untuk menghasilkan laporan kehadiran atau performa.
		\item \textbf{Diagram Jaringan Komputasi/Perangkat Keras}: Diagram ini menggambarkan jaringan komputer dan perangkat keras yang mendukung pembelajaran hybrid, seperti koneksi antara server LMS dengan jaringan internet kampus, router, dan perangkat endpoint.
		\item \textbf{Diagram Teknik Komunikasi}: Diagram ini menunjukkan protokol dan saluran komunikasi yang digunakan dalam hybrid learning, misalnya penggunaan protokol HTTPS untuk akses aman ke LMS dan protokol WebRTC untuk streaming video di platform konferensi.
	\end{itemize}
\end{itemize}


\section{Dokumen Definisi Arsitektur yang Diperbarui}
Topik yang harus dibahas dalam Dokumen Definisi Arsitektur terkait Arsitektur Teknologi meliputi:
\begin{itemize}
	\item Arsitektur Teknologi Baseline, jika ada 
	\item Arsitektur Teknologi Target, mencakup:
	\begin{itemize}
		\item Komponen teknologi dan hubungannya dengan sistem informasi
		\item Platform teknologi dan dekomposisinya, menunjukkan kombinasi teknologi yang diperlukan untuk mewujudkan "stack" teknologi tertentu
		\item Lingkungan dan lokasi dengan pengelompokan teknologi yang diperlukan dalam lingkungan komputasi (misalnya, pengembangan, produksi)
		\item Beban pemrosesan yang diharapkan dan distribusi beban di seluruh komponen teknologi
		\item Komunikasi fisik (jaringan)
		\item Spesifikasi perangkat keras dan jaringan
	\end{itemize}
	\item Tampilan yang sesuai dengan sudut pandang yang dipilih untuk menjawab kekhawatiran utama pemangku kepentingan
\end{itemize}

\subsection*{Contoh:}

\begin{itemize}
	\item \textbf{Arsitektur Teknologi Baseline, jika ada}: Arsitektur teknologi baseline adalah gambaran teknologi saat ini yang digunakan untuk mendukung operasi utama dalam universitas. Misalnya, pada sistem e-learning, arsitektur baseline dapat mencakup server lokal yang mendukung aplikasi Learning Management System (LMS) seperti Moodle, yang diakses secara terbatas oleh mahasiswa di jaringan kampus. Contoh lainnya pada sistem informasi akademik adalah server database SQL yang menyimpan data mahasiswa dan staf, serta aplikasi web berbasis internal yang memungkinkan akses melalui jaringan kampus. Arsitektur baseline ini memberikan gambaran tentang infrastruktur dasar sebelum adanya pembaruan atau integrasi teknologi baru.
	
	\item \textbf{Arsitektur Teknologi Target, mencakup}:
	\begin{itemize}
		\item \textbf{Komponen teknologi dan hubungannya dengan sistem informasi}: Komponen teknologi dalam arsitektur target meliputi elemen yang berperan penting dalam proses informasi universitas. Contohnya adalah penggunaan server dan penyimpanan berbasis cloud yang mendukung LMS, yang juga terintegrasi dengan aplikasi video konferensi untuk pembelajaran hybrid. Contoh lain adalah penerapan sistem autentikasi tunggal yang memungkinkan mahasiswa dan staf menggunakan satu identitas untuk mengakses berbagai aplikasi universitas, seperti portal akademik, perpustakaan digital, dan layanan email.
		
		\item \textbf{Platform teknologi dan dekomposisinya, menunjukkan kombinasi teknologi yang diperlukan untuk mewujudkan "stack" teknologi tertentu}: Platform teknologi target merinci kombinasi teknologi yang bekerja bersama untuk mendukung aplikasi tertentu. Misalnya, stack teknologi yang terdiri dari React sebagai frontend, Node.js sebagai backend, dan MongoDB sebagai basis data untuk portal akademik yang digunakan mahasiswa. Contoh lainnya adalah penggunaan platform AWS dengan komponen S3 untuk penyimpanan data, EC2 untuk komputasi, dan RDS untuk basis data relasional, menciptakan arsitektur multi-tier untuk mendukung skenario hybrid learning.
		
		\item \textbf{Lingkungan dan lokasi dengan pengelompokan teknologi yang diperlukan dalam lingkungan komputasi (misalnya, pengembangan, produksi)}: Lingkungan komputasi dipisahkan untuk pengembangan, pengujian, dan produksi. Dalam konteks ini, lingkungan pengembangan memungkinkan tim IT untuk menguji fitur baru pada LMS tanpa mengganggu operasional. Contoh lainnya adalah penggunaan lingkungan staging yang mencerminkan lingkungan produksi, sehingga simulasi akses oleh pengguna dapat diuji terlebih dahulu sebelum peluncuran ke semua mahasiswa dan staf.
		
		\item \textbf{Beban pemrosesan yang diharapkan dan distribusi beban di seluruh komponen teknologi}: Beban pemrosesan memperkirakan jumlah permintaan yang harus dikelola oleh setiap komponen teknologi. Misalnya, saat ujian daring, server LMS harus mampu menangani ribuan mahasiswa yang mengakses platform bersamaan, sehingga penggunaan load balancer diperlukan untuk distribusi beban. Contoh lain adalah server email kampus yang menerima permintaan pengiriman email dalam jumlah besar setiap hari, memerlukan skalabilitas otomatis agar server dapat menangani lonjakan lalu lintas pada periode tertentu.
		
		\item \textbf{Komunikasi fisik (jaringan)}: Komunikasi fisik dalam arsitektur teknologi memastikan konektivitas antara berbagai perangkat dan sistem. Contohnya, jaringan kabel di ruang kelas utama memungkinkan akses internet yang stabil untuk perangkat multimedia yang digunakan dalam pembelajaran hybrid. Contoh lainnya adalah jaringan nirkabel kampus yang diprioritaskan untuk perangkat mobile mahasiswa dan staf agar dapat mengakses LMS dan materi kuliah dengan cepat dari seluruh area kampus.
		
		\item \textbf{Spesifikasi perangkat keras dan jaringan}: Spesifikasi perangkat keras dan jaringan disesuaikan untuk mendukung kebutuhan operasional hybrid learning. Misalnya, server dengan prosesor tinggi dan kapasitas RAM yang besar diperlukan untuk mendukung ribuan akses ke LMS pada waktu bersamaan. Contoh lainnya adalah penggunaan router dan switch berkapasitas tinggi yang mendukung komunikasi data cepat dan stabil di seluruh area kampus, memastikan infrastruktur jaringan dapat mendukung berbagai perangkat yang terhubung.
	\end{itemize}
	
	\item \textbf{Tampilan yang sesuai dengan sudut pandang yang dipilih untuk menjawab kekhawatiran utama pemangku kepentingan}: Tampilan arsitektur disusun berdasarkan sudut pandang pemangku kepentingan utama, memastikan bahwa arsitektur teknologi memenuhi kebutuhan masing-masing. Misalnya, dari sudut pandang dosen, tampilan arsitektur perlu mendukung akses mudah dan cepat ke materi serta tugas mahasiswa di LMS. Sementara itu, dari sudut pandang administrator IT, fokus utamanya adalah keamanan data dan keberlanjutan infrastruktur, memastikan bahwa semua data sensitif dilindungi dan infrastruktur mendukung pertumbuhan jumlah pengguna.
\end{itemize}


\section{Spesifikasi Persyaratan Arsitektur yang Diperbarui}
Persyaratan Arsitektur Teknologi yang ditambahkan ke dalam Spesifikasi Persyaratan Arsitektur pada Fase D meliputi:
\begin{itemize}
	\item Hasil Analisis Kesenjangan
	\item Persyaratan teknologi yang diperbarui
\end{itemize}

\subsection*{Contoh:}

\begin{itemize}
	\item \textbf{Hasil Analisis Kesenjangan}: Hasil analisis kesenjangan menunjukkan perbedaan antara arsitektur teknologi saat ini dan kebutuhan masa depan yang diinginkan. Misalnya, dalam konteks hybrid learning, analisis kesenjangan mungkin mengungkapkan bahwa kapasitas server saat ini tidak memadai untuk mendukung akses serentak dari ribuan mahasiswa selama ujian daring, sehingga diperlukan peningkatan kapasitas atau solusi distribusi beban. Contoh lain adalah identifikasi kebutuhan untuk integrasi antara Learning Management System (LMS) dan sistem video konferensi, di mana sistem saat ini mungkin tidak mendukung integrasi penuh yang diperlukan untuk pembelajaran daring dan tatap muka yang seamless.
	
	\item \textbf{Persyaratan Teknologi yang Diperbarui}: Persyaratan teknologi yang diperbarui menggambarkan spesifikasi dan kebutuhan teknologi yang diperlukan untuk mencapai target arsitektur yang mendukung seluruh aspek hybrid learning. Misalnya, dalam mendukung skenario hybrid, mungkin dibutuhkan teknologi jaringan yang lebih cepat dan stabil di seluruh area kampus untuk memastikan pengalaman pembelajaran yang tidak terganggu, baik di dalam kelas fisik maupun di lingkungan daring. Contoh lainnya adalah peningkatan kebutuhan perangkat keras pada server dan penyimpanan cloud, seperti penambahan unit penyimpanan dan peningkatan kapasitas RAM pada server, untuk mengakomodasi volume data yang meningkat akibat penambahan jumlah materi digital dan rekaman kelas.
\end{itemize}

\section{Tren Terkini dalam IT}
Teknologi yang muncul dan mempengaruhi fase Arsitektur Teknologi meliputi:
\begin{enumerate}
	\item Layanan Cloud
	\item Microservices
	\item Kontainer Perangkat Lunak
	\item DevOps
	\item Jaringan Seluler 5G
	\item Realitas Virtual/Augmented
	\item Kembar Digital
	\item Internet of Things
	\item Kecerdasan Buatan
	\item Printer 3D
	\item Edge Computing
	\item Blockchain
	\item Gerakan Web3 Terdesentralisasi
	\item Keamanan Siber
	\item Komputasi Kuantum
\end{enumerate}

\subsection*{Contoh:}

\begin{itemize}
	\item \textbf{Layanan Cloud}: Layanan cloud menyediakan infrastruktur, platform, dan perangkat lunak yang dapat diakses secara online, menggantikan kebutuhan untuk infrastruktur fisik. \emph{Contoh:} Amazon Web Services (AWS), Google Cloud Platform (GCP), Microsoft Azure.
	
	\item \textbf{Microservices}: Microservices adalah arsitektur pengembangan perangkat lunak yang memecah aplikasi menjadi komponen-komponen kecil yang dapat dikembangkan dan diterapkan secara independen. \emph{Contoh:} Aplikasi Netflix yang menggunakan microservices untuk memisahkan layanan streaming, rekomendasi, dan pembayaran.
	
	\item \textbf{Kontainer Perangkat Lunak}: Kontainer menyediakan lingkungan yang ringan dan konsisten untuk menjalankan aplikasi di berbagai platform tanpa konflik dependensi. \emph{Contoh:} Docker, Kubernetes.
	
	\item \textbf{DevOps}: DevOps menggabungkan pengembangan dan operasi untuk meningkatkan kecepatan pengembangan, pengujian, dan peluncuran perangkat lunak melalui otomatisasi. \emph{Contoh:} Penggunaan CI/CD pipelines dengan Jenkins atau GitLab CI.
	
	\item \textbf{Jaringan Seluler 5G}: Teknologi 5G menyediakan kecepatan dan kapasitas yang jauh lebih tinggi untuk komunikasi data, mendukung IoT dan aplikasi latensi rendah. \emph{Contoh:} Penyedia layanan 5G seperti Verizon, AT\&T, dan Huawei.
	
	\item \textbf{Realitas Virtual/Augmented}: Teknologi ini menciptakan pengalaman digital yang memperluas atau mereplikasi lingkungan dunia nyata atau virtual untuk pengguna. \emph{Contoh:} Oculus Rift untuk VR, dan aplikasi AR seperti Pokémon Go.
	
	\item \textbf{Kembar Digital (Digital Twin)}: Kembar digital adalah representasi digital dari objek atau sistem fisik yang dapat dimonitor dan dianalisis secara real-time. \emph{Contoh:} Kembar digital dari mesin pesawat terbang yang digunakan oleh General Electric (GE) untuk memantau performa.
	
	\item \textbf{Internet of Things (IoT)}: IoT memungkinkan objek-objek fisik untuk terhubung ke internet dan saling berkomunikasi. \emph{Contoh:} Perangkat rumah pintar seperti Amazon Echo dan Google Nest.
	
	\item \textbf{Kecerdasan Buatan (AI)}: AI memungkinkan komputer untuk melakukan tugas yang biasanya memerlukan kecerdasan manusia, seperti pengenalan gambar dan pemrosesan bahasa alami. \emph{Contoh:} Asisten virtual seperti Siri dan Google Assistant.
	
	\item \textbf{Printer 3D}: Printer 3D dapat membuat objek fisik dari desain digital dengan menyusun lapisan bahan secara berurutan. \emph{Contoh:} Printer 3D seperti Ultimaker untuk pembuatan prototipe cepat dalam industri manufaktur.
	
	\item \textbf{Edge Computing}: Edge computing mengurangi latensi dengan memproses data di dekat sumbernya, bukan di pusat data yang jauh. \emph{Contoh:} Perangkat IoT di pabrik yang memproses data sensor secara lokal sebelum mengirimkan hasilnya ke cloud.
	
	\item \textbf{Blockchain}: Blockchain adalah teknologi buku besar terdistribusi yang memungkinkan transaksi yang aman dan transparan. \emph{Contoh:} Platform cryptocurrency seperti Bitcoin dan Ethereum.
	
	\item \textbf{Gerakan Web3 Terdesentralisasi}: Web3 bertujuan untuk menciptakan internet terdesentralisasi dengan memanfaatkan blockchain untuk keamanan dan kontrol data pribadi. \emph{Contoh:} Aplikasi terdesentralisasi (DApps) seperti Uniswap dan OpenSea.
	
	\item \textbf{Keamanan Siber}: Keamanan siber adalah praktek melindungi sistem, jaringan, dan data dari serangan digital. \emph{Contoh:} Perangkat lunak antivirus seperti Norton, dan firewall yang digunakan di jaringan perusahaan.
	
	\item \textbf{Komputasi Kuantum}: Komputasi kuantum menggunakan prinsip mekanika kuantum untuk memecahkan masalah yang terlalu kompleks bagi komputer klasik. \emph{Contoh:} Komputer kuantum dari IBM Q System dan Google Quantum AI.
\end{itemize}

\section{Ringkasan}
Fase Arsitektur Teknologi memungkinkan kita untuk mendefinisikan dan memvalidasi arsitektur teknologi yang diperlukan untuk mendukung:
\begin{enumerate}
	\item Arsitektur Bisnis
	\item Arsitektur Data
	\item Arsitektur Aplikasi
\end{enumerate}
