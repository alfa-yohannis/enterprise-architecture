\chapter{Arsitektur Data}

\section{Tujuan}
Tujuan dari fase ini adalah untuk mengembangkan Arsitektur Data Target yang memungkinkan Arsitektur Bisnis dan Visi Arsitektur dapat tercapai, sambil menangani Permintaan Pekerjaan Arsitektur dan \textit{concerns} dari para \textit{stakeholder}. Selain itu, fase ini bertujuan untuk mengidentifikasi komponen Peta Jalan Arsitektur berdasarkan kesenjangan antara Arsitektur Data Dasar dan Arsitektur Data Target.

\section{Input (1)}
Input yang digunakan dalam fase ini meliputi:
\begin{enumerate}
	\item Permintaan Pekerjaan Arsitektur
	\item Penilaian Kapabilitas
	\item Rencana Komunikasi
	\item Model Organisasi untuk Arsitektur Perusahaan
	\item Kerangka Kerja Arsitektur yang Disesuaikan
	\item Prinsip Data, jika ada
	\item Pernyataan Pekerjaan Arsitektur
	\item Draf Spesifikasi Kebutuhan Arsitektur, mencakup:
	\begin{enumerate}
		\item Hasil Analisis Kesenjangan
		\item Kebutuhan teknis relevan lainnya
	\end{enumerate}
	\item Visi Arsitektur
	\item Repositori Arsitektur
	\item Draf Dokumen Definisi Arsitektur, mencakup:
	\begin{enumerate}
		\item Arsitektur Bisnis Dasar (rinci)
		\item Arsitektur Bisnis Target (rinci)
		\item Arsitektur Data Dasar (garis besar)
		\item Arsitektur Data Target (garis besar)
		\item Arsitektur Aplikasi Dasar (garis besar)
		\item Arsitektur Aplikasi Target (garis besar)
		\item Arsitektur Teknologi Dasar (garis besar)
		\item Arsitektur Teknologi Target (garis besar)
	\end{enumerate}
	\item Komponen Arsitektur Bisnis dari Peta Jalan Arsitektur
\end{enumerate}

\section{Langkah-langkah}
Langkah-langkah yang diambil dalam fase ini meliputi:
\begin{enumerate}
	\item Memilih model referensi, sudut pandang, dan alat
	\item Mengembangkan Deskripsi Arsitektur Data Dasar
	\item Mengembangkan Deskripsi Arsitektur Data Target
	\item Melakukan Analisis Kesenjangan
	\item Menentukan komponen peta jalan kandidat
	\item Menyelesaikan dampak pada Lanskap Arsitektur
	\item Melakukan tinjauan formal dari pemangku kepentingan
	\item Menyelesaikan Arsitektur Data
	\item Membuat Dokumen Definisi Arsitektur
\end{enumerate}

\subsection*{Contoh:}
Langkah-langkah yang diambil dalam fase ini meliputi:
\begin{enumerate}
	\item Memilih model referensi, sudut pandang, dan alat
	\begin{itemize}
		\item Contoh: Memilih model referensi pedagogis seperti Community of Inquiry (CoI) untuk pembelajaran hybrid, dengan sudut pandang dari dosen dan mahasiswa, serta alat seperti Moodle atau Google Classroom untuk manajemen kursus.
	\end{itemize}
	\item Mengembangkan Deskripsi Arsitektur Data \textit{Baseline}
	\begin{itemize}
		\item Contoh: Membuat peta data yang mencakup informasi tentang interaksi mahasiswa dalam lingkungan belajar daring dan luring, seperti data kehadiran dan partisipasi dalam diskusi.
	\end{itemize}
	\item Mengembangkan Deskripsi Arsitektur Data Target
	\begin{itemize}
		\item Contoh: Merancang sistem yang mendukung analitik pembelajaran untuk mengevaluasi keterlibatan dan kemajuan mahasiswa, termasuk pengintegrasian data dari platform pembelajaran dan sistem manajemen akademik.
	\end{itemize}
	\item Melakukan Analisis Kesenjangan
	\begin{itemize}
		\item Contoh: Menganalisis kesenjangan antara praktik pengajaran saat ini dan kebutuhan untuk model pembelajaran hybrid, seperti kurangnya data tentang keterlibatan mahasiswa dalam sesi tatap muka versus daring.
	\end{itemize}
	\item Menentukan komponen peta jalan kandidat
	\begin{itemize}
		\item Contoh: Mengidentifikasi proyek-proyek pengembangan kursus hybrid, seperti pelatihan dosen dalam penggunaan teknologi pembelajaran dan pengembangan materi pembelajaran yang dapat diakses secara daring.
	\end{itemize}
	\item Menyelesaikan dampak pada Lanskap Arsitektur
	\begin{itemize}
		\item Contoh: Menganalisis dampak transisi ke pembelajaran hybrid pada infrastruktur TI universitas dan kesiapan dosen untuk mengajar dalam format ini.
	\end{itemize}
	\item Melakukan tinjauan formal dari pemangku kepentingan
	\begin{itemize}
		\item Contoh: Mengadakan forum diskusi dengan dosen, mahasiswa, dan staf administrasi untuk meninjau arsitektur pembelajaran hybrid dan mengumpulkan masukan tentang tantangan yang dihadapi.
	\end{itemize}
	\item Menyelesaikan Arsitektur Data
	\begin{itemize}
		\item Contoh: Finalisasi desain arsitektur data yang mendukung evaluasi efektivitas pembelajaran hybrid dengan memperhitungkan umpan balik dari pemangku kepentingan dan analisis data yang ada.
	\end{itemize}
	\item Membuat Dokumen Definisi Arsitektur
	\begin{itemize}
		\item Contoh: Menyusun dokumen definisi arsitektur yang merinci struktur dan proses pembelajaran hybrid, kebijakan penggunaan teknologi, dan protokol untuk mengukur hasil pembelajaran.
	\end{itemize}
\end{enumerate}


\section{Output}
Output dari fase ini mencakup:
\begin{enumerate}
	\item Pernyataan Pekerjaan Arsitektur, diperbarui sesuai kebutuhan
	\item Prinsip-prinsip data yang divalidasi, atau prinsip-prinsip data baru
	\item Draf Dokumen Definisi Arsitektur dengan konten yang diperbarui
	\item Spesifikasi Kebutuhan Arsitektur yang telah didesain, termasuk yang telah diperbarui
	\item Komponen Arsitektur Data dari Peta Jalan Arsitektur
\end{enumerate}

\subsection{Output}
Output dari fase ini mencakup:
\begin{enumerate}
	\item Pernyataan Pekerjaan Arsitektur, diperbarui sesuai kebutuhan
	\begin{itemize}
		\item Contoh: Dokumen Pernyataan Pekerjaan Arsitektur yang merinci tanggung jawab dan tugas tim dalam mengembangkan lingkungan belajar hybrid, termasuk penyesuaian terhadap kebutuhan teknologi yang muncul selama proses pengajaran.
	\end{itemize}
	\item Prinsip-prinsip data yang divalidasi, atau prinsip-prinsip data baru
	\begin{itemize}
		\item Contoh: Prinsip data yang divalidasi untuk menjaga kerahasiaan data mahasiswa dan memastikan aksesibilitas konten pembelajaran secara adil, serta penambahan prinsip baru mengenai penggunaan data analitik untuk meningkatkan pengalaman belajar mahasiswa.
	\end{itemize}
	\item Draf Dokumen Definisi Arsitektur dengan konten yang diperbarui
	\begin{itemize}
		\item Contoh: Draf Dokumen Definisi Arsitektur yang mencakup detail terbaru mengenai platform pembelajaran yang digunakan, prosedur pengajaran hybrid, dan mekanisme evaluasi yang terintegrasi antara pembelajaran daring dan luring.
	\end{itemize}
	\item Spesifikasi Kebutuhan Arsitektur yang telah didesain, termasuk yang telah diperbarui
	\begin{itemize}
		\item Contoh: Spesifikasi kebutuhan yang mencakup alat dan teknologi yang diperlukan untuk mendukung pembelajaran hybrid, seperti perangkat lunak untuk pengelolaan kursus dan sistem umpan balik mahasiswa yang efektif.
	\end{itemize}
	\item Komponen Arsitektur Data dari Peta Jalan Arsitektur
	\begin{itemize}
		\item Contoh: Komponen arsitektur data yang menggambarkan struktur sistem penyimpanan data untuk pembelajaran hybrid, termasuk database untuk menyimpan catatan kehadiran, partisipasi dalam diskusi, dan hasil evaluasi, serta integrasi data dari berbagai platform yang digunakan.
	\end{itemize}
\end{enumerate}

\section{Prinsip Data}
Beberapa prinsip data yang dipegang dalam fase ini adalah:
\begin{enumerate}
	\item \textbf{Data adalah Aset}. Data adalah aset yang memiliki nilai bagi perusahaan dan dikelola sesuai dengan itu.
	\item \textbf{Data dapat dibagikan}. Pengguna dapat mengakses data yang diperlukan untuk melaksanakan tugas mereka; oleh karena itu, data dibagikan di seluruh fungsi dan organisasi perusahaan.
	\item \textbf{Data yang dapat diakses}. Data dapat diakses oleh pengguna untuk menjalankan fungsi mereka.
	\item \textbf{Definisi Kosakata dan Data yang Dibagikan}. Data didefinisikan secara konsisten di seluruh perusahaan, dan definisi tersebut dapat dipahami serta tersedia untuk semua pengguna.
	\item \textbf{Keamanan Data}. Data dilindungi dari penggunaan dan pengungkapan yang tidak sah. Selain aspek tradisional dari klasifikasi keamanan nasional, ini termasuk tetapi tidak terbatas pada perlindungan sebelum keputusan, informasi sensitif, pemilihan sumber sensitif, dan informasi kepemilikan.
\end{enumerate}

\section{Katalog, Matriks, dan Diagram Arsitektur Data}
Dalam fase ini, berbagai katalog, matriks, dan diagram yang digunakan meliputi:
\begin{table}[ht]
	\begin{tabular}{|c|c|c|}
		\hline
		\textbf{Katalog} & \textbf{Matriks} & \textbf{Diagram} \\ \hline
		Katalog Entitas Data/Komponen & Matriks Entitas Data/Fungsi Bisnis & Diagram Data Konseptual \\
		& Matriks Aplikasi/Data & Diagram Data Logis \\
		& & Diagram Penyebaran Data \\
		& & Diagram Siklus Hidup Data \\
		& & Diagram Keamanan Data \\
		& & Diagram Migrasi Data \\ \hline
	\end{tabular}
\end{table}

\section{Komponen Dokumen Definisi Arsitektur}
\label{sec:data_komponen_dokumen_definisi_arsitektur}
Topik yang harus dibahas dalam Dokumen Definisi Arsitektur yang terkait dengan Arsitektur Data adalah sebagai berikut:
\begin{itemize}
	\item Arsitektur Data Baseline, jika memungkinkan
	\item Arsitektur Data Target, termasuk model untuk data bisnis, data logis, dan proses manajemen data, serta matriks Entitas Data/Fungsi Bisnis
	\item Pandangan Arsitektur Data yang sesuai dengan sudut pandang yang dipilih, yang mengatasi kekhawatiran pemangku kepentingan kunci
\end{itemize}

\subsection*{Contoh:}
\begin{itemize}
	\item \textbf{Arsitektur Data Baseline, jika memungkinkan} \\
	Contoh: Pada fase awal implementasi hybrid learning, universitas mengumpulkan data mengenai penggunaan platform pembelajaran daring (LMS) dan kehadiran mahasiswa dalam pertemuan tatap muka. Data baseline ini mencakup statistik tentang frekuensi akses mahasiswa ke materi pembelajaran, tingkat partisipasi dalam diskusi online, dan kehadiran dalam kelas fisik. 
	
	\item \textbf{Arsitektur Data Target, termasuk model untuk data bisnis, data logis, dan proses manajemen data, serta matriks Entitas Data/Fungsi Bisnis} \\
	Contoh: Universitas menetapkan Arsitektur Data Target yang mencakup model untuk pengelolaan data mahasiswa, dosen, dan kurikulum. Model ini memetakan hubungan antara entitas seperti mahasiswa, kursus, dan dosen. Misalnya, matriks Entitas Data/Fungsi Bisnis menunjukkan bagaimana data tentang mahasiswa (seperti nilai dan kehadiran) digunakan untuk meningkatkan pengalaman belajar dalam konteks hybrid, misalnya, melalui pengembangan materi yang sesuai untuk pembelajaran daring dan tatap muka.
	
	\item \textbf{Pandangan Arsitektur Data yang sesuai dengan sudut pandang yang dipilih, yang mengatasi kekhawatiran pemangku kepentingan kunci} \\
	Contoh: Dalam konteks hybrid learning, pandangan arsitektur data mencakup perspektif pemangku kepentingan seperti mahasiswa, dosen, dan pengelola pendidikan. Misalnya, mahasiswa mungkin mengkhawatirkan aksesibilitas materi pembelajaran secara daring, sedangkan dosen mungkin fokus pada bagaimana data kehadiran dan partisipasi dapat membantu mereka dalam menyesuaikan metode pengajaran. Pandangan ini diintegrasikan ke dalam laporan analisis untuk memastikan bahwa semua kebutuhan dan kekhawatiran dipertimbangkan dalam pengembangan arsitektur data.
\end{itemize}

\section{Komponen Spesifikasi Kebutuhan Arsitektur}
\label{sec:data_komponen_spesifikasi_kebutuhan}
Kebutuhan Arsitektur Data yang mengisi Spesifikasi Kebutuhan Arsitektur pada Fase C meliputi:
\begin{itemize}
	\item Hasil Analisis Kesenjangan
	\item Kebutuhan interoperabilitas data (misalnya, skema XML, kebijakan keamanan)
	\item Area di mana Arsitektur Bisnis mungkin perlu berubah untuk mematuhi perubahan dalam Arsitektur Data
	\item Pembatasan pada Arsitektur Teknologi yang akan dirancang
	\item Kebutuhan bisnis yang telah diperbarui, jika ada
	\item Kebutuhan aplikasi yang telah diperbarui, jika ada
\end{itemize}

\begin{itemize}
	\item \textbf{Hasil Analisis Kesenjangan} \\
	Contoh: Setelah menganalisis data penggunaan LMS, ditemukan bahwa terdapat kesenjangan antara materi yang disediakan secara daring dan kebutuhan mahasiswa dalam kelas tatap muka. Hasil analisis ini menunjukkan perlunya pengembangan materi tambahan yang dapat diakses secara daring untuk mendukung pemahaman mahasiswa.
	
	\item \textbf{Kebutuhan interoperabilitas data (misalnya, skema XML, kebijakan keamanan)} \\
	Contoh: Dalam konteks hybrid learning, kebutuhan interoperabilitas data mencakup penggunaan format data standar seperti skema XML untuk memungkinkan pertukaran data antara berbagai sistem, seperti LMS dan sistem manajemen akademik. Kebijakan keamanan juga diperlukan untuk melindungi data pribadi mahasiswa selama proses ini.
	
	\item \textbf{Area di mana Arsitektur Bisnis mungkin perlu berubah untuk mematuhi perubahan dalam Arsitektur Data} \\
	Contoh: Untuk mematuhi perubahan dalam Arsitektur Data, universitas mungkin perlu meninjau kembali kebijakan pendaftaran kursus. Misalnya, jika kursus hybrid ditawarkan, prosedur pendaftaran mungkin perlu disesuaikan agar mahasiswa dapat memilih antara opsi pembelajaran daring atau tatap muka sesuai kebutuhan mereka.
	
	\item \textbf{Pembatasan pada Arsitektur Teknologi yang akan dirancang} \\
	Contoh: Dalam merancang Arsitektur Teknologi untuk mendukung hybrid learning, terdapat pembatasan seperti kebutuhan untuk mendukung akses dari berbagai perangkat (laptop, tablet, smartphone) dan memastikan kompatibilitas dengan berbagai platform pembelajaran daring. Ini juga termasuk pembatasan terkait bandwidth yang tersedia untuk mengakses materi pembelajaran secara efektif.
	
	\item \textbf{Kebutuhan bisnis yang telah diperbarui, jika ada} \\
	Contoh: Seiring dengan pengembangan sistem hybrid learning, kebutuhan bisnis universitas yang diperbarui mungkin mencakup peningkatan layanan dukungan akademik untuk membantu mahasiswa yang belajar secara daring, termasuk layanan bimbingan online dan akses ke sumber daya tambahan.
	
	\item \textbf{Kebutuhan aplikasi yang telah diperbarui, jika ada} \\
	Contoh: Kebutuhan aplikasi yang diperbarui mungkin termasuk pembaruan pada aplikasi mobile universitas untuk mendukung fitur baru seperti notifikasi kelas tatap muka, akses ke materi pembelajaran daring, dan integrasi dengan sistem pembayaran untuk biaya kuliah yang lebih mudah diakses oleh mahasiswa.
\end{itemize}

\section{Beberapa Hal Penting untuk Diperhatikan untuk Hybrid Learning and Teaching pada Tingkat Universitas}
Dalam konteks \textit{\textbf{hybrid teaching and learning}} di universitas, hal-hal berikut harus diperhatikan dengan cermat untuk memastikan kelancaran proses pembelajaran jarak jauh maupun tatap muka:

\begin{enumerate}
	\item \textbf{Migrasi data offline ke online}: Proses migrasi materi pembelajaran, seperti catatan fisik atau bahan ajar cetak, ke platform digital yang diakses melalui internet, seperti Learning Management System (LMS).
	\item \textbf{Transformasi data fisik dan tidak terstruktur menjadi digital dan terstruktur}: Contohnya adalah mengubah catatan kuliah atau tugas dalam bentuk tulisan tangan atau cetak menjadi dokumen digital yang dapat disimpan dalam format terstruktur seperti PDF atau dokumen Word.
	\item \textbf{Keamanan data}: Penting untuk melindungi data akademik, termasuk nilai, tugas, dan rekaman kuliah, menggunakan teknologi enkripsi, manajemen kata sandi yang baik, dan pengaturan hak akses untuk memastikan hanya pengguna yang berwenang yang dapat mengakses data tersebut.
	\item \textbf{Tata kelola data}: Menentukan kebijakan otorisasi dan pengaturan CRUD (create, read, update, delete) untuk data pembelajaran, serta menentukan berapa lama data akan disimpan sebelum dihapus sesuai dengan kebijakan universitas.
	\item \textbf{Pengelolaan data}: Memastikan bagaimana data pembelajaran seperti materi kuliah, tugas, dan hasil ujian disimpan, dibagikan kepada mahasiswa, diakses oleh dosen, dan diproses untuk evaluasi.
	\item \textbf{Cadangan dan pemulihan bencana}: Menyediakan sistem cadangan untuk data penting, seperti rekaman kuliah dan hasil penilaian, serta memastikan adanya mekanisme pemulihan data jika terjadi bencana atau kegagalan sistem.
	\item \textbf{Biaya data}: Mengelola biaya yang terkait dengan penyimpanan data, seperti biaya server untuk platform LMS, kecepatan pemrosesan, transfer data antar pengguna, dan kapasitas penyimpanan yang dibutuhkan untuk menyimpan data digital dalam skala besar.
\end{enumerate}



\section{Inovasi Terkait Data}
Beberapa inovasi terkait data meliputi:

\begin{enumerate}
	\item Berkas Teks dan Biner
	\item Data tidak terstruktur
	\item Data terstruktur
	\item Sistem Basis Data Relasional (RDBMS)
	\item Data Pemrosesan Transaksi Online (OLTP)
	\item Data Pemrosesan Analitik Online (OLAP)
	\item Data Lake
	\item Data Warehouse
	\item Business Intelligence
	\item BigData
	\item Data Mining
	\item NoSQL dan Data Tidak Terstruktur
	\item Basis Data Graf
	\item Basis Data Berkas
	\item Basis Data Kolom
	\item Basis Data Dalam Memori
	\item Pemrosesan Sequential vs Paralel
	\item Data Terpusat vs Terdistribusi
	\item Data Terpusat vs Federasi
	\item Data Terpusat vs Desentralisasi (gerakan Web3) seperti Filecoin Arweave, SOLID POD
\end{enumerate}

\subsection*{Contoh:}

\begin{enumerate}
	\item \textbf{Berkas Teks dan Biner}: 
	\begin{itemize}
		\item Contoh: File teks.
		\item Berkas biner seperti rekaman video kuliah dalam format MP4, rekaman suara dalam format WAV, dokumen terkompresi seperti DOCX.
	\end{itemize}
	
	\item \textbf{Data Tidak Terstruktur}: 
	\begin{itemize}
		\item Contoh: Komentar atau diskusi mahasiswa di forum pembelajaran pada platform Moodle.
		\item Rekaman video diskusi kelompok di Zoom, dan hasil scan tulisan tangan dalam format JPEG atau PNG.
	\end{itemize}
	
	\item \textbf{Data Terstruktur}: 
	\begin{itemize}
		\item Contoh: Data mahasiswa dalam tabel Microsoft Excel mencakup informasi seperti nama, nomor mahasiswa, dan nilai.
		\item Tabel di basis data SQL yang digunakan oleh sistem informasi akademik (SIA).
	\end{itemize}
	
	\item \textbf{Sistem Basis Data Relasional (RDBMS)}: 
	\begin{itemize}
		\item Contoh Produk: MySQL, PostgreSQL, atau Oracle Database yang digunakan untuk mengelola data akademik, seperti pendaftaran mata kuliah dan data keuangan mahasiswa.
	\end{itemize}
	
	\item \textbf{Data Pemrosesan Transaksi Online (OLTP)}: 
	\begin{itemize}
		\item Contoh: Sistem pembayaran online untuk biaya kuliah menggunakan payment gateway seperti Midtrans atau Xendit.
		\item Sistem pendaftaran mata kuliah secara real-time melalui SIA.
	\end{itemize}
	
	\item \textbf{Data Pemrosesan Analitik Online (OLAP)}: 
	\begin{itemize}
		\item Contoh Produk: Power BI atau Tableau digunakan untuk menghasilkan laporan analitik seperti kinerja akademik mahasiswa dan tren kehadiran.
	\end{itemize}
	
	\item \textbf{Data Lake}: 
	\begin{itemize}
		\item Contoh: AWS S3 atau Google Cloud Storage yang menyimpan berbagai jenis data pembelajaran seperti rekaman kuliah, tugas, dan hasil ujian dalam format mentah.
	\end{itemize}
	
	\item \textbf{Data Warehouse}: 
	\begin{itemize}
		\item Contoh Produk: Amazon Redshift, Google BigQuery, atau Snowflake digunakan untuk menyimpan data terstruktur, seperti data kinerja akademik untuk pelaporan atau analisis jangka panjang.
	\end{itemize}
	
	\item \textbf{Business Intelligence}: 
	\begin{itemize}
		\item Contoh Produk: Microsoft Power BI atau IBM Cognos digunakan untuk analisis dan visualisasi data pembelajaran, seperti performa akademik dan statistik penggunaan platform e-learning.
	\end{itemize}
	
	\item \textbf{Big Data}: 
	\begin{itemize}
		\item Contoh Produk: Hadoop atau Apache Spark digunakan untuk memproses data dalam jumlah besar, seperti interaksi mahasiswa dengan materi kuliah, jumlah klik, dan waktu yang dihabiskan pada setiap materi.
	\end{itemize}
	
	\item \textbf{Data Mining}: 
	\begin{itemize}
		\item Contoh Produk: RapidMiner atau WEKA digunakan untuk menambang data interaksi pembelajaran guna menemukan pola keterlibatan mahasiswa yang memprediksi prestasi akademik.
	\end{itemize}
	
	\item \textbf{NoSQL dan Data Tidak Terstruktur}: 
	\begin{itemize}
		\item Contoh Produk: MongoDB atau Cassandra digunakan untuk menyimpan data tidak terstruktur seperti rekaman diskusi Zoom atau log aktivitas mahasiswa.
	\end{itemize}
	
	\item \textbf{Basis Data Graf}: 
	\begin{itemize}
		\item Contoh Produk: Neo4j atau Amazon Neptune digunakan untuk memetakan dan menganalisis hubungan antar mahasiswa, dosen, dan mata kuliah dalam komunitas akademik.
	\end{itemize}
	
	\item \textbf{Basis Data Berkas}: 
	\begin{itemize}
		\item Contoh Produk: Hadoop HDFS atau GlusterFS digunakan untuk menyimpan berkas seperti rekaman kuliah atau bahan ajar multimedia.
	\end{itemize}
	
	\item \textbf{Basis Data Kolom}: 
	\begin{itemize}
		\item Contoh Produk: Apache Cassandra atau HBase digunakan untuk menyimpan data berskala besar, seperti data historis interaksi mahasiswa dengan LMS.
	\end{itemize}
	
	\item \textbf{Basis Data Dalam Memori}: 
	\begin{itemize}
		\item Contoh Produk: Redis atau Memcached digunakan untuk mempercepat akses data, misalnya saat ujian daring atau akses cepat ke materi yang sering digunakan.
	\end{itemize}
	
	\item \textbf{Pemrosesan Sequential vs Paralel}: 
	\begin{itemize}
		\item Contoh: Pemrosesan sequential untuk menghitung nilai mahasiswa satu per satu setelah ujian.
		\item Pemrosesan paralel dengan Apache Spark untuk menganalisis data aktivitas ribuan mahasiswa sekaligus.
	\end{itemize}
	
	\item \textbf{Data Terpusat vs Terdistribusi}: 
	\begin{itemize}
		\item Contoh Terpusat: Data akademik disimpan di server pusat universitas menggunakan Microsoft SQL Server.
		\item Contoh Terdistribusi: Data dikelola oleh masing-masing fakultas dengan basis data terpisah, misalnya menggunakan MongoDB untuk penyimpanan terdistribusi.
	\end{itemize}
	
	\item \textbf{Data Terpusat vs Federasi}: 
	\begin{itemize}
		\item Contoh Terpusat: Data disimpan di server pusat universitas, seperti Google Cloud SQL.
		\item Contoh Federasi: Data disimpan terpisah di berbagai kampus, tetapi terhubung melalui federasi data di PostgreSQL.
	\end{itemize}
	
	\item \textbf{Data Terpusat vs Desentralisasi (gerakan Web3)}: 
	\begin{itemize}
		\item Contoh Terpusat: Data mahasiswa disimpan di server universitas yang dikelola departemen IT.
		\item Contoh Desentralisasi: Menggunakan Filecoin atau SOLID POD, di mana data disimpan secara terdesentralisasi dan setiap individu mengelola data pribadinya.
	\end{itemize}
\end{enumerate}

\section{Aktivitas Kelas dan Tugas}
Buat model data dan dokumen \textbf{As-Is} yang terpengaruh oleh kapabilitas yang dipilih. Selain itu, buat model data dan dokumen \textbf{To-Be} yang diperlukan untuk mewujudkan kapbilitas terpilih. Tuangkan dalam bentuk atau perbaharui Dokumen Definisi Arsitektur (Subbab \ref{sec:isi_dokumen_definisi_arsitektur} dan \ref{sec:data_komponen_dokumen_definisi_arsitektur}) dan Spesifikasi Persyaratan Arsitektur (Subbab \ref{sec:data_komponen_spesifikasi_kebutuhan} dan  \ref{sec:data_komponen_spesifikasi_kebutuhan}).
