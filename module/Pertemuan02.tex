\chapter{Studi Kasus: Kerja Jarak Jauh}

\section{Pendahuluan}

Covid-19 merupakan pemicu utama perdebatan antara Kerja Dari Kantor (WFO) dan Kerja Jarak Jauh. Kontroversi ini berlanjut karena sebagian orang mendukung kerja jarak jauh sementara yang lain menolaknya. Isu ini tetap relevan hingga saat ini, dengan perusahaan-perusahaan yang semakin banyak menerapkan model kerja hybrid. Masalah umum seperti kemacetan lalu lintas, kekhawatiran lingkungan seperti polusi dan perubahan iklim, serta preferensi pekerja berkontribusi pada diskusi yang sedang berlangsung. Masalah kerja jarak jauh adalah isu dunia nyata dengan implikasi yang signifikan.

\section{Penelitian Awal}

Penelitian awal dilakukan di Universitas MTI Pradita, termasuk tugas mahasiswa mengenai topik ini dan survei dengan 91 responden (target: 100). Data tersedia di \url{https://zenodo.org/records/8116799} dan memberikan gambaran tentang masalah kerja jarak jauh. Penelitian ini berguna sebagai dasar untuk menerapkan Arsitektur Perusahaan untuk mengatasi tantangan kerja jarak jauh. Makalah terkait telah dipublikasikan dan dapat diakses di \url{https://ieeexplore.ieee.org/abstract/document/10327330}.

\section{Hasil: Manfaat Kerja Jarak Jauh}

Kerja jarak jauh menawarkan beberapa manfaat:
\begin{itemize}
	\item \textbf{Efektif Biaya}: Mengurangi biaya transportasi bagi karyawan dan biaya operasional bagi perusahaan.
	\item \textbf{Penghematan Waktu}: Waktu perjalanan dialihkan untuk kegiatan lain.
	\item \textbf{Keseimbangan Kerja-Hidup}: Waktu lebih untuk keluarga dan hobi meningkatkan kesehatan mental.
	\item \textbf{Peningkatan Produktivitas}: Fleksibilitas dalam waktu dan tempat meningkatkan hasil kerja dan mengurangi gangguan kantor.
\end{itemize}

\section{Hasil: Tantangan dalam Kerja Jarak Jauh}

Meskipun manfaatnya, kerja jarak jauh menghadapi beberapa tantangan:
\begin{itemize}
	\item \textbf{Jenis Pekerjaan}: Beberapa pekerjaan, seperti pekerjaan pabrik atau penelitian laboratorium, tidak cocok untuk kerja jarak jauh.
	\item \textbf{Masalah Teknis}: Masalah termasuk internet lambat, masalah perangkat keras/perangkat lunak, dan infrastruktur yang tidak memadai.
	\item \textbf{Investasi yang Diperlukan}: Biaya paket data, koneksi internet, dan perangkat keras/perangkat lunak.
	\item \textbf{Isu Etiket}: Tantangan dengan kontak atau tugas di luar jam kerja dan komunikasi yang tidak sopan.
	\item \textbf{Gangguan Lingkungan}: Kondisi rumah yang tidak mendukung mempengaruhi kerja.
\end{itemize}

Tantangan tambahan meliputi:
\begin{itemize}
	\item \textbf{Isu Motivasi}: Karyawan mungkin kekurangan motivasi tanpa pengawasan.
	\item \textbf{Isu Kepercayaan}: Pengawas mungkin meragukan kemampuan karyawan untuk bekerja secara mandiri.
	\item \textbf{Masalah Komunikasi}: Perbedaan dalam kebiasaan komunikasi dibandingkan dengan interaksi langsung.
	\item \textbf{Isu Sosialisasi}: Kesulitan dalam membangun kerja tim dan jaringan.
\end{itemize}

\section{Solusi untuk Tantangan}

Mengatasi tantangan kerja jarak jauh melibatkan beberapa solusi:
\begin{itemize}
	\item \textbf{Solusi Teknis}: Meningkatkan koneksi internet, menyediakan perangkat keras/perangkat lunak, dan memelihara infrastruktur.
	\item \textbf{Sumber Daya Manusia}: Meningkatkan keterampilan non-teknis seperti etiket, kemandirian, motivasi, komunikasi, dan kepercayaan diri. Tingkatkan keterampilan teknis dengan pelatihan penggunaan perangkat lunak dan pemecahan masalah TI.
	\item \textbf{Proses Bisnis dan Tata Kelola}: Sesuaikan proses bisnis untuk kerja jarak jauh. Alihkan penilaian kinerja untuk fokus pada hasil daripada kehadiran atau jam kerja.
\end{itemize}

Solusi tambahan meliputi:
\begin{itemize}
	\item \textbf{Pendanaan}: Kalibrasi ulang dan alokasikan ulang sumber daya keuangan.
	\item \textbf{Interaksi Sosial}: Selenggarakan acara sosial secara rutin untuk mendorong kerja sama, jaringan, dan persahabatan.
	\item \textbf{Infrastruktur dan Dukungan}: Sediakan sumber daya atau bantuan untuk kerja jarak jauh yang efektif.
\end{itemize}

\section{Visi Studi Kasus}
\textbf{Tujuan:} Universitas mampu beroperasi secara fleksibel dengan model kerja hybrid -- Kerja Dari Kantor dan Kerja Dari Mana Saja, dengan \textbf{objektif/target} sebagai berikut:
\begin{enumerate}
	\item Mengubah proses bisnis sesuai dengan tujuan di atas.
	\item Mengubah tata kelola: termasuk pembagian tanggung jawab/wewenang, aturan dalam menilai kinerja sumber daya manusia, dan sebagainya.
	\item Menyiapkan teknologi yang memampukan kerja jarak jauh.
	\item Menyiapkan sumber daya manusia yang siap untuk kerja jarak jauh.
	\item Penyesuaian anggaran untuk mendukung kerja jarak jauh.
\end{enumerate}

%\section{Persyaratan Kelulusan}
%
%Untuk memenuhi persyaratan kelulusan, mahasiswa harus:
%\begin{itemize}
%	\item Menyelesaikan tugas akhir (proyek).
%	\item Menerbitkan satu makalah di jurnal internasional yang terindeks oleh Scopus.
%	\item Menerbitkan dua makalah di jurnal nasional yang terindeks oleh SINTA (platform indeks publikasi nasional).
%	\item Menyajikan dua makalah di konferensi internasional yang terindeks oleh Scopus (misalnya, konferensi IEEE atau ACM).
%\end{itemize}

\section{Aktivitas Kelas dan Tugas}
Berdasarkan temuan-temuan pada tugas pertemuan sebelumnya, tetapkan \textbf{kemampuan} apa yang perlu ditambahkan ke organisasi Anda untuk meningkatkan kualitas bisnisnya. Selanjutnya, temukan setidaknya 30 sumber pustaka, termasuk makalah akademik, laporan, white papers, dll. Ceritakan kemampuan yang Anda temukan di dalam makalah.

\section{Contoh Beberapa Kapabilitas atau Solusi Transisi ke Model Hybrid untuk Kegiatan Kelas dan Perkuliahan}

Seandainya, pihak universitas sepakat untuk mengadopsi model kerja hybrid untuk meningkatkan kualitas dan pengalaman proses belajar mengajarnya, berikut adalah hal-hal yang dapat dilakukan untuk mendukung model kerja tersebut.

\subsection{Platform Pembelajaran dan Teknologi}
\begin{itemize}
	\item \textbf{Platform Pembelajaran Daring:} Gunakan platform pembelajaran daring seperti Moodle, Blackboard, atau Canvas untuk menyampaikan materi kuliah, tugas, dan penilaian. Pastikan platform ini dapat diakses dengan mudah oleh mahasiswa dari berbagai perangkat.
	\item \textbf{Sistem Kelas Virtual:} Implementasikan sistem kelas virtual seperti Zoom, Microsoft Teams, atau Google Meet untuk sesi perkuliahan langsung, diskusi kelompok, dan konsultasi dosen.
	\item \textbf{Alat Pembelajaran Interaktif:} Manfaatkan alat seperti kuis online, polling, dan papan tulis digital untuk meningkatkan keterlibatan mahasiswa selama kuliah.
\end{itemize}

\subsection{Pengajaran dan Kurikulum}
\begin{itemize}
	\item \textbf{Desain Kurikulum Hybrid:} Rancang kurikulum yang mendukung format hybrid dengan kombinasi perkuliahan tatap muka dan online. Tentukan bagian mana yang dapat dilakukan secara daring dan mana yang memerlukan pertemuan langsung.
	\item \textbf{Modul Pembelajaran Asinkron:} Buat modul pembelajaran asinkron yang memungkinkan mahasiswa mengakses materi dan menyelesaikan tugas pada waktu mereka sendiri. Ini mencakup video kuliah, bacaan, dan bahan ajar lainnya.
	\item \textbf{Evaluasi Berkala:} Implementasikan sistem evaluasi berkala yang mencakup kuis online, tugas individu, dan proyek kelompok untuk memantau kemajuan mahasiswa.
\end{itemize}

\subsection{Interaksi dan Keterlibatan}
\begin{itemize}
	\item \textbf{Sesi Tanya Jawab dan Diskusi:} Adakan sesi tanya jawab dan diskusi daring secara rutin untuk menjaga interaksi antara dosen dan mahasiswa. Gunakan breakout rooms dalam platform video untuk diskusi kelompok kecil.
	\item \textbf{Jam Kantor Virtual:} Sediakan waktu jam kantor virtual di mana mahasiswa dapat berkonsultasi langsung dengan dosen melalui video call atau chat.
	\item \textbf{Forum Diskusi:} Gunakan forum diskusi dalam platform pembelajaran untuk memungkinkan mahasiswa bertanya, berdiskusi, dan berkolaborasi secara online.
\end{itemize}

\subsection{Manajemen Kelas}
\begin{itemize}
	\item \textbf{Pendaftaran dan Jadwal Kelas:} Manajemen jadwal kelas dan pendaftaran dapat dilakukan secara daring melalui sistem manajemen akademik untuk memastikan aksesibilitas dan koordinasi yang efektif.
	\item \textbf{Catatan Kehadiran:} Gunakan sistem absensi digital untuk memantau kehadiran mahasiswa pada sesi perkuliahan daring, termasuk integrasi dengan platform pembelajaran untuk kehadiran otomatis.
	\item \textbf{Akses Materi:} Pastikan semua materi kuliah dan bahan ajar dapat diakses oleh mahasiswa melalui platform pembelajaran, dan sediakan salinan materi untuk diunduh jika diperlukan.
\end{itemize}

\subsection{Penilaian dan Umpan Balik}
\begin{itemize}
	\item \textbf{Ujian Online:} Gunakan alat ujian online untuk administrasi ujian yang aman dan terkelola dengan baik. Implementasikan fitur seperti pengacakan pertanyaan dan waktu terbatas untuk menjaga integritas ujian.
	\item \textbf{Umpan Balik Berkelanjutan:} Berikan umpan balik yang cepat dan konstruktif melalui platform pembelajaran atau email, dan dorong mahasiswa untuk memberikan umpan balik mengenai pengalaman pembelajaran daring mereka.
	\item \textbf{Penilaian Otentik:} Terapkan penilaian otentik yang relevan dengan dunia nyata, seperti proyek berbasis kasus atau studi lapangan virtual, untuk mengukur pemahaman dan keterampilan mahasiswa secara lebih holistik.
\end{itemize}

\subsection{Dukungan Teknologi dan Pelatihan}
\begin{itemize}
	\item \textbf{Pelatihan Dosen:} Berikan pelatihan kepada dosen mengenai penggunaan teknologi pembelajaran daring dan strategi pengajaran yang efektif dalam format hybrid.
	\item \textbf{Dukungan TI:} Sediakan dukungan TI yang responsif untuk mengatasi masalah teknis yang dihadapi mahasiswa dan dosen selama sesi perkuliahan daring.
	\item \textbf{Panduan dan Tutorial:} Buat panduan dan tutorial yang jelas untuk mahasiswa mengenai cara menggunakan platform pembelajaran, mengakses materi, dan berpartisipasi dalam kelas virtual.
\end{itemize}

\subsection{Kesejahteraan Mahasiswa}
\begin{itemize}
	\item \textbf{Keseimbangan Kerja-Kuliah:} Dorong mahasiswa untuk menjaga keseimbangan antara studi dan kehidupan pribadi mereka dengan menyediakan fleksibilitas dalam jadwal dan beban tugas.
	\item \textbf{Dukungan Kesehatan Mental:} Sediakan layanan dukungan kesehatan mental secara daring, termasuk konseling dan sumber daya untuk membantu mahasiswa menghadapi stres dan tantangan yang mungkin timbul dari format pembelajaran hybrid.
	\item \textbf{Kegiatan Sosial Virtual:} Selenggarakan kegiatan sosial virtual untuk menjaga keterhubungan mahasiswa dan membangun komunitas meskipun secara daring.
\end{itemize}

\subsection{Data dan Infrastruktur}
\begin{itemize}
	\item \textbf{Manajemen Data:} Gunakan sistem manajemen data berbasis cloud untuk menyimpan dan mengelola data akademik dan administratif secara aman dan terpusat. Pastikan bahwa data mahasiswa, dosen, dan administrasi dapat diakses dengan mudah namun tetap aman.
	\item \textbf{Keamanan Data:} Implementasikan protokol keamanan data yang ketat untuk melindungi informasi sensitif dari akses yang tidak sah. Gunakan enkripsi data dan autentikasi multi-faktor untuk meningkatkan keamanan.
	\item \textbf{Konektivitas Infrastruktur:} Pastikan infrastruktur TI di universitas mendukung konektivitas yang stabil dan cepat untuk mendukung kegiatan pembelajaran daring. Investasikan dalam jaringan yang handal dan redundansi sistem.
\end{itemize}

\subsection{Proses Bisnis Terkait Proses Pembelajaran}
\begin{itemize}
	\item \textbf{Automasi Proses:} Automasi proses administrasi seperti pendaftaran kursus, penilaian, dan pelaporan akademik untuk mengurangi beban kerja administratif dan meningkatkan efisiensi.
	\item \textbf{Integrasi Sistem:} Integrasikan sistem pembelajaran daring dengan sistem manajemen akademik dan keuangan untuk memudahkan koordinasi dan pengelolaan data.
	\item \textbf{Proses Onboarding Mahasiswa Baru:} Rancang proses onboarding yang efisien untuk mahasiswa baru yang mencakup orientasi daring, pelatihan penggunaan platform pembelajaran, dan pengenalan kepada dosen serta staf.
\end{itemize}
