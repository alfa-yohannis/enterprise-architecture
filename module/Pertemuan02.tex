\chapter{Studi Kasus: Remote Work}

\section{Introduction}

Covid-19 was the main trigger for the debate between Work From Office (WFO) and Remote Working. This controversy continues as some people are in favor of remote working while others are against it. This issue remains relevant today, with companies increasingly implementing hybrid working models. General issues such as traffic congestion, environmental concerns like pollution and climate change, and worker preferences contribute to the ongoing discussion. The problem of remote working is a real-world issue with significant implications.

\section{Preliminary Research}

Initial research was conducted at MTI Pradita University, including student assignments on the topic and a survey with 91 respondents (target: 100). The data is available at \url{https://zenodo.org/records/8116799} and provides an illustration of remote working problems. This research is useful as a basis for implementing Enterprise Architecture to address remote working challenges. A related paper has been published and can be accessed at \url{https://ieeexplore.ieee.org/abstract/document/10327330}.

\section{Results: Benefits of Remote Working}

Remote working offers several benefits:
\begin{itemize}
	\item \textbf{Cost-effective}: Reduces transportation costs for employees and operational costs for companies.
	\item \textbf{Time-saving}: Commuting time is repurposed for other activities.
	\item \textbf{Work-life balance}: More time for family and hobbies leads to improved mental health.
	\item \textbf{Increased productivity}: Flexibility in time and place enhances work results and reduces office distractions.
\end{itemize}

\section{Results: Challenges in Remote Working}

Despite its benefits, remote working presents several challenges:
\begin{itemize}
	\item \textbf{Nature of Work}: Some jobs, such as factory work or laboratory research, are not suited for remote work.
	\item \textbf{Technical Issues}: Issues include slow internet, hardware/software problems, and inadequate infrastructure.
	\item \textbf{Investment Required}: Costs for data packages, internet connections, and hardware/software.
	\item \textbf{Etiquette Issues}: Challenges with contacts or assignments outside working hours and rude communication.
	\item \textbf{Environmental Interference}: Non-conducive home conditions affecting work.
\end{itemize}

Additional challenges include:
\begin{itemize}
	\item \textbf{Motivational Issues}: Employees may lack motivation without supervision.
	\item \textbf{Trust Issues}: Supervisors may doubt employees' ability to work independently.
	\item \textbf{Communication Problems}: Differences in communication habits compared to direct interactions.
	\item \textbf{Socialization Issues}: Difficulty in building teamwork and networking.
\end{itemize}

\section{Solutions to the Challenges}

Addressing the challenges of remote working involves several solutions:
\begin{itemize}
	\item \textbf{Technical Solutions}: Improve internet connections, procure hardware/software, and maintain infrastructure.
	\item \textbf{Human Resources}: Enhance non-technical skills such as etiquette, independence, motivation, communication, and confidence. Improve technical skills with training on software usage and IT troubleshooting.
	\item \textbf{Business Processes and Governance}: Adapt business processes for remote work. Shift performance appraisal to focus on outcomes rather than attendance or hours.
\end{itemize}

Additional solutions include:
\begin{itemize}
	\item \textbf{Funding}: Recalibrate and reallocate financial resources.
	\item \textbf{Social Interaction}: Regularly host social events to foster cooperation, networking, and friendship.
	\item \textbf{Infrastructure and Support}: Provide resources or assistance for effective remote work.
\end{itemize}

\section{Visi Studi Kasus}
\textbf{Tujuan:} Perusahaan mampu bekerja secara fleksibel yang mencakup moda kerja hybrid -- Work From Office dan Work From Anywhere, dengan \textbf{objektif/target} sebagai berikut:
\begin{enumerate}
	\item Mengubah proses bisnis disesuaikan dengan tujuan di atas.
	\item Mengubah tata kelola: termasuk pembagian tanggung jawab/wewenang, aturan dalam menilai kinerja sumber daya manusia, dan sebagainya.
	\item Menyiapkan teknologi yang memampukan kerja jarak jauh.
	\item Menyiapkan sumber daya manusia yang siap kerja jarak jauh.
	\item Penyesuain anggaran untuk mendukung kerja jarak jauh.
\end{enumerate}

\section{Graduation Requirements}

To meet graduation requirements, students must:
\begin{itemize}
	\item Complete a final assignment (project).
	\item Publish one paper in an international journal indexed by Scopus.
	\item Publish two papers in national journals indexed by SINTA (national publication index platform).
	\item Present two papers at international conferences indexed by Scopus (e.g., IEEE or ACM conferences).
\end{itemize}


\section{Aktivitas Kelas dan Tugas}
Berdasarkan temuan-temuan ada pada tugas pertemuan sebelumnya, tetapkan \textbf{kemampuan} apa yang perlu ditambahkan ke perusahaan Anda bekerja untuk meningkatkan kualitas bisnisnya. Selanjutlnya, temukan setidaknya 30 sumber pustaka, termasuk makalah akademik, laporan, white papers, dll.  Ceritakan kemampuan yang Anda temukan di dalam makalah.
