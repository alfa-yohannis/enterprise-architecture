\chapter{Fase G: Tata Kelola Implementasi}

\section{Tujuan}
\begin{enumerate}
	\item Memastikan kesesuaian proyek implementasi dengan Arsitektur Target.
	\item Melaksanakan fungsi Tata Kelola Arsitektur yang sesuai untuk solusi dan permintaan perubahan arsitektur yang didorong oleh implementasi.
\end{enumerate}

\section{Input}
\begin{itemize}
	\item Permintaan untuk Pekerjaan Arsitektur
	\item Penilaian Kapabilitas
	\item Model Organisasi untuk Arsitektur Perusahaan
	\item Kerangka Arsitektur yang Disesuaikan
	\item Pernyataan Pekerjaan Arsitektur
	\item Visi Arsitektur
	\item Repositori Arsitektur
	\item Dokumen Definisi Arsitektur
	\item Spesifikasi Kebutuhan Arsitektur
	\item Roadmap Arsitektur
	\item Model Tata Kelola Implementasi
	\item Kontrak Arsitektur
	\item Permintaan untuk Pekerjaan Arsitektur yang diidentifikasi dalam Fase E dan F
	\item Rencana Implementasi dan Migrasi
\end{itemize}

\section{Langkah-Langkah}
\begin{enumerate}
	\item Konfirmasi ruang lingkup dan prioritas untuk penerapan dengan manajemen pengembangan.
	\item Identifikasi sumber daya dan keterampilan untuk penerapan.
	\item Pandu pengembangan solusi penerapan.
	\item Lakukan Tinjauan Kepatuhan Arsitektur Perusahaan.
	\item Implementasikan operasi bisnis dan TI.
	\item Lakukan tinjauan pasca-implementasi dan tutup implementasi.
\end{enumerate}

\section{Output}
\begin{enumerate}
	\item Kontrak Arsitektur (ditandatangani)
	\item Penilaian Kepatuhan
	\item Permintaan Perubahan
	\item Analisis Dampak - Rekomendasi Implementasi
	\item Solusi yang sesuai dengan arsitektur diterapkan, termasuk:
	\begin{itemize}
		\item Sistem yang diimplementasikan sesuai dengan arsitektur
		\item Repositori Arsitektur yang terisi
		\item Rekomendasi kepatuhan arsitektur dan pengecualian
		\item Rekomendasi tentang persyaratan pengiriman layanan
		\item Rekomendasi tentang metrik kinerja
		\item Perjanjian Tingkat Layanan (SLA)
		\item Visi Arsitektur, diperbarui setelah implementasi
		\item Dokumen Definisi Arsitektur, diperbarui setelah implementasi
		\item Arsitektur Transisi, diperbarui setelah implementasi
		\item Model operasional bisnis dan TI untuk solusi yang diimplementasikan
	\end{itemize}
\end{enumerate}

\section{Kontrak Arsitektur}
\begin{enumerate}
	\item Kontrak Arsitektur dalam Fase G adalah kesepakatan bersama mengenai hasil, kualitas, dan kelayakan antara mitra pengembangan dan sponsor.
	\item Implementasi yang berhasil dari kesepakatan ini membutuhkan pendekatan tata kelola yang efektif.
	\item Tata kelola memastikan pemantauan berkelanjutan, pemeriksaan integritas, dan audit semua kegiatan yang berkaitan dengan arsitektur.
	\item Kepatuhan terhadap prinsip, standar, dan persyaratan dari arsitektur yang ada atau yang sedang dikembangkan ditegakkan.
	\item Risiko dalam pengembangan, implementasi, dan aspek operasional arsitektur diidentifikasi dan ditangani.
	\item Proses dan praktik memastikan akuntabilitas, tanggung jawab, dan disiplin dalam pengembangan dan penggunaan semua artefak arsitektur.
\end{enumerate}

\section{Konten Umum Kontrak Desain dan Pengembangan Arsitektur}
\begin{itemize}
	\item Pendahuluan dan latar belakang
	\item Sifat dari kesepakatan
	\item Ruang lingkup arsitektur
	\item Prinsip dan persyaratan arsitektur strategis
	\item Persyaratan kepatuhan
	\item Proses dan peran pengembangan serta manajemen arsitektur
	\item Ukuran Arsitektur Target
	\item Tahapan hasil yang terdefinisi
	\item Rencana kerja bersama yang diprioritaskan
	\item Batas waktu pelaksanaan
	\item Pengiriman arsitektur dan metrik bisnis
\end{itemize}

\section{Konten Umum Kontrak Arsitektur Pengguna Bisnis}
\begin{itemize}
	\item Pendahuluan dan latar belakang
	\item Sifat dari kesepakatan
	\item Ruang lingkup
	\item Persyaratan strategis
	\item Persyaratan kepatuhan
	\item Adopter arsitektur
	\item Batas waktu pelaksanaan
	\item Metrik bisnis arsitektur
	\item Arsitektur layanan (termasuk SLA)
	\begin{itemize}
		\item Kontrak ini juga digunakan untuk mengelola perubahan arsitektur perusahaan dalam Fase H.
	\end{itemize}
\end{itemize}

\section{Penilaian Kepatuhan}
\begin{itemize}
	\item Ikhtisar kemajuan dan status proyek
	\item Ikhtisar arsitektur/proyek desain
	\item Checklist arsitektur yang telah selesai:
	\begin{itemize}
		\item Checklist perangkat keras dan sistem operasi
		\item Checklist layanan perangkat lunak dan middleware
		\item Checklist aplikasi
		\item Checklist manajemen informasi
		\item Checklist keamanan
		\item Checklist manajemen sistem
		\item Checklist rekayasa sistem
		\item Checklist metode dan alat
	\end{itemize}
\end{itemize}

\section{Ringkasan}
\begin{enumerate}
	\item Tata Kelola Implementasi mendefinisikan batasan arsitektur dan mendapatkan tanda tangan pada Kontrak Arsitektur.
	\item Serahkan kontrak dan semua dokumentasi kepada tim implementasi.
	\item Kelola arsitektur selama implementasi dengan tinjauan kepatuhan dan pemantauan risiko.
	\item Lakukan tinjauan pasca-implementasi.
\end{enumerate}
