\chapter{Fase G: Tata Kelola Implementasi}

\section{Tujuan}
\begin{enumerate}
	\item Memastikan kesesuaian proyek implementasi dengan Arsitektur Target.
	\item Melaksanakan fungsi Tata Kelola Arsitektur yang sesuai untuk solusi dan permintaan perubahan arsitektur yang didorong oleh implementasi.
\end{enumerate}

\section{Input}
\begin{itemize}
	\item Permintaan untuk Pekerjaan Arsitektur
	\item Penilaian Kapabilitas
	\item Model Organisasi untuk Arsitektur Perusahaan
	\item Kerangka Arsitektur yang Disesuaikan
	\item Pernyataan Pekerjaan Arsitektur
	\item Visi Arsitektur
	\item Repositori Arsitektur
	\item Dokumen Definisi Arsitektur
	\item Spesifikasi Kebutuhan Arsitektur
	\item Roadmap Arsitektur
	\item Model Tata Kelola Implementasi
	\item Kontrak Arsitektur
	\item Permintaan untuk Pekerjaan Arsitektur yang diidentifikasi dalam Fase E dan F
	\item Rencana Implementasi dan Migrasi
\end{itemize}

\section{Langkah-Langkah}
\begin{enumerate}
	\item Konfirmasi ruang lingkup dan prioritas untuk penerapan dengan manajemen pengembangan.
	\item Identifikasi sumber daya dan keterampilan untuk penerapan.
	\item Pandu pengembangan solusi penerapan.
	\item Lakukan Tinjauan Kepatuhan Arsitektur Perusahaan.
	\item Implementasikan operasi bisnis dan TI.
	\item Lakukan tinjauan pasca-implementasi dan tutup implementasi.
\end{enumerate}

\subsection*{Contoh:}

\begin{enumerate}
	\item \textbf{Konfirmasi ruang lingkup dan prioritas untuk penerapan dengan manajemen pengembangan:} 
	Langkah ini memastikan ruang lingkup proyek penerapan pembelajaran hybrid telah disepakati antara tim pengembangan dan manajemen universitas, termasuk prioritas fitur dan hasil yang diharapkan. Misalnya, ruang lingkup awal mungkin difokuskan pada penerapan Learning Management System (LMS) yang mendukung penyampaian materi kuliah, pengumpulan tugas, dan forum diskusi daring. Sebagai contoh lain, universitas dapat memprioritaskan infrastruktur video konferensi yang stabil dan terintegrasi untuk mendukung kelas hybrid secara sinkron.
	
	\item \textbf{Identifikasi sumber daya dan keterampilan untuk penerapan:} 
	Langkah ini mengidentifikasi sumber daya seperti perangkat keras (server, perangkat jaringan), perangkat lunak (LMS, platform video konferensi), serta keterampilan yang diperlukan untuk mendukung sistem pembelajaran hybrid. Misalnya, tim IT universitas memerlukan keterampilan dalam integrasi LMS dengan sistem akademik yang ada untuk pengelolaan data mahasiswa. Sebagai contoh lain, pelatihan dosen mengenai penggunaan LMS dan teknologi video konferensi perlu dilakukan untuk memastikan kelancaran transisi ke model pembelajaran hybrid.
	
	\item \textbf{Pandu pengembangan solusi penerapan:} 
	Tim pengembangan harus mengarahkan solusi agar sesuai dengan kebutuhan pembelajaran hybrid. Misalnya, dalam pengembangan LMS, tim memastikan bahwa fitur pengumpulan tugas dan forum diskusi tersedia dan mudah diakses oleh mahasiswa. Contoh lainnya adalah integrasi platform video konferensi dengan LMS untuk memungkinkan mahasiswa bergabung dalam kelas secara daring atau tatap muka sesuai kebutuhan.
	
	\item \textbf{Lakukan Tinjauan Kepatuhan Arsitektur Perusahaan:} 
	Tinjauan ini dilakukan untuk memastikan bahwa solusi yang diterapkan sejalan dengan standar arsitektur dan kebijakan universitas. Misalnya, sebelum menerapkan LMS, tim memeriksa bahwa sistem ini memenuhi standar keamanan data untuk melindungi informasi akademik mahasiswa. Sebagai contoh lain, tinjauan dilakukan terhadap kebijakan akses video konferensi untuk memastikan hanya mahasiswa terdaftar yang dapat mengakses materi dan sesi kelas.
	
	\item \textbf{Implementasikan operasi bisnis dan TI:} 
	Operasi yang mendukung pembelajaran hybrid diterapkan agar sistem dapat berjalan sesuai dengan tujuan pembelajaran dan dukungan teknologi yang stabil. Misalnya, universitas memperbarui prosedur operasional agar dosen dapat mengelola kelas tatap muka dan daring secara bersamaan menggunakan LMS dan platform video konferensi. Contoh lainnya adalah penerapan proses otomatis untuk pencatatan kehadiran mahasiswa, baik yang hadir secara fisik maupun daring.
	
	\item \textbf{Lakukan tinjauan pasca-implementasi dan tutup implementasi:} 
	Langkah ini mengevaluasi hasil implementasi sistem pembelajaran hybrid untuk memastikan tujuan proyek tercapai sebelum penutupan proyek. Misalnya, universitas mengadakan tinjauan pasca-implementasi untuk mengevaluasi efektivitas LMS dalam mendukung pembelajaran hybrid, mengumpulkan masukan dari dosen dan mahasiswa. Contoh lain adalah evaluasi kepuasan pengguna terhadap platform video konferensi, yang diukur dari kemudahan akses dan efektivitas dalam mendukung interaksi kelas.
\end{enumerate}

\section{Output}
\begin{enumerate}
	\item Kontrak Arsitektur (disetujui)
	\item Penilaian Kepatuhan (\textit{compliance assessment})
	\item Permintaan Perubahan
	\item Analisis Dampak - Rekomendasi Implementasi
	\item Solusi yang sesuai dengan arsitektur diterapkan, termasuk:
	\begin{itemize}
		\item Sistem yang diimplementasikan sesuai dengan arsitektur
		\item Repositori Arsitektur yang terisi
		\item Rekomendasi kepatuhan arsitektur dan pengecualian
		\item Rekomendasi tentang persyaratan pengiriman layanan
		\item Rekomendasi tentang metrik kinerja
		\item Perjanjian Tingkat Layanan (SLA)
		\item Visi Arsitektur, diperbarui setelah implementasi
		\item Dokumen Definisi Arsitektur, diperbarui setelah implementasi
		\item Arsitektur Transisi, diperbarui setelah implementasi
		\item Model operasional bisnis dan TI untuk solusi yang diimplementasikan
	\end{itemize}
\end{enumerate}

\section{Kontrak Arsitektur}

\subsection{Gambaran Umum}
Kontrak Arsitektur yang dihasilkan pada Fase G adalah kesepakatan bersama antara mitra pengembangan dan sponsor mengenai hasil, kualitas, dan kelayakan dari arsitektur. Keberhasilan implementasi kontrak ini bergantung pada tata kelola arsitektur yang efektif. Dengan pendekatan tata kelola dalam manajemen kontrak, hasil berikut dapat dipastikan:

\begin{itemize}
	\item Sistem pemantauan berkelanjutan untuk memverifikasi integritas, mengelola perubahan, mendukung pengambilan keputusan, dan mengaudit semua kegiatan yang terkait dengan arsitektur dalam organisasi.
	\item Kepatuhan terhadap prinsip, standar, dan persyaratan arsitektur yang ada atau yang sedang dikembangkan.
	\item Identifikasi risiko dalam semua aspek pengembangan dan implementasi arsitektur, meliputi pengembangan internal sesuai standar, kebijakan, teknologi, dan produk yang diterima, serta aspek operasional arsitektur untuk memastikan kelangsungan bisnis dalam lingkungan yang tangguh.
	\item Serangkaian proses dan praktik yang menjamin akuntabilitas, tanggung jawab, dan disiplin dalam pengembangan dan penggunaan semua artefak arsitektur.
	\item Pemahaman formal mengenai organisasi tata kelola yang bertanggung jawab atas kontrak, tingkat wewenang mereka, dan ruang lingkup arsitektur di bawah pengawasan badan ini.
\end{itemize}

Standar TOGAF mengidentifikasi dua contoh kontrak:
\begin{enumerate}
	\item Kontrak Desain dan Pengembangan Arsitektur
	\item Kontrak Arsitektur Pengguna Bisnis
\end{enumerate}

\subsection{Isi isi Kontrak Desain dan Pengembangan Arsitektur}
\begin{itemize}
	\item Pendahuluan dan latar belakang
	\item Sifat dari kesepakatan
	\item Ruang lingkup arsitektur
	\item Prinsip dan persyaratan arsitektur strategis
	\item Persyaratan kepatuhan
	\item Proses dan peran pengembangan serta manajemen arsitektur
	\item Ukuran Arsitektur Target
	\item Tahapan hasil yang terdefinisi
	\item Rencana kerja bersama yang diprioritaskan
	\item Jangka waktu pelaksanaan
	\item Hasil arsitektur dan metrik bisnis
\end{itemize}

\subsection{Isi dari Kontrak Arsitektur Pengguna Bisnis}
\begin{itemize}
	\item Pendahuluan dan latar belakang
	\item Sifat dari kesepakatan
	\item Ruang lingkup
	\item Persyaratan strategis
	\item Persyaratan kepatuhan
	\item Pengguna arsitektur
	\item Jangka waktu pelaksanaan
	\item Metrik bisnis arsitektur
	\item Arsitektur layanan (termasuk SLA)
	\begin{itemize}
		\item Kontrak ini juga digunakan untuk mengelola perubahan arsitektur perusahaan pada Fase H.
	\end{itemize}
\end{itemize}

\section{Penilaian Kepatuhan}

\subsection{Konten Umum dari Penilaian Kepatuhan}
\begin{itemize}
	\item Gambaran umum kemajuan dan status proyek
	\item Gambaran umum arsitektur/desain proyek
	\item Daftar periksa arsitektur yang telah selesai:
	\begin{itemize}
		\item Daftar periksa perangkat keras dan sistem operasi
		\item Daftar periksa layanan perangkat lunak dan middleware
		\item Daftar periksa aplikasi
		\item Daftar periksa manajemen informasi
		\item Daftar periksa keamanan
		\item Daftar periksa manajemen sistem
		\item Daftar periksa rekayasa sistem
		\item Daftar periksa metode dan alat
	\end{itemize}
\end{itemize}

\section{Contoh Kontrak Desain dan Pengembangan Arsitektur untuk Pembelajaran dan Pengajaran Hybrid}
\label{sec:contoh_kontrak_desain_dan_pengembangan_arsitektur}
\begin{enumerate}
	
	\item \textbf{Pendahuluan dan Latar Belakang} \\
	Kontrak ini menetapkan kerangka kerja untuk desain dan pengembangan arsitektur pembelajaran dan pengajaran hybrid yang disesuaikan untuk mendukung tujuan universitas dalam menyediakan pendidikan yang fleksibel, mudah diakses, dan berkualitas tinggi. Kontrak ini bertujuan untuk meningkatkan pengalaman pendidikan dengan mengintegrasikan metode pembelajaran daring dan tatap muka guna memenuhi kebutuhan beragam mahasiswa dan meningkatkan hasil belajar.
	
	\item \textbf{Sifat dari Kesepakatan} \\
	Kesepakatan ini menguraikan tanggung jawab dan komitmen dari semua pihak yang terlibat, termasuk fakultas, staf administrasi, departemen TI, dan mitra teknologi eksternal, untuk secara kolaboratif merancang, mengembangkan, dan memelihara arsitektur pembelajaran hybrid yang kuat. Kontrak ini berfungsi sebagai pedoman untuk mencapai dan mempertahankan lingkungan pembelajaran hybrid yang berkualitas.
	
	\item \textbf{Ruang Lingkup Arsitektur} \\
	Ruang lingkup arsitektur ini mencakup pengembangan dan integrasi platform pembelajaran digital, kelas virtual, sistem manajemen konten, dan alat-alat untuk mendukung pengajaran secara tatap muka dan jarak jauh. Ini mencakup aspek penyampaian pengajaran, keterlibatan mahasiswa, penilaian, dan dukungan bagi fakultas dalam model hybrid.
	
	\item \textbf{Prinsip dan Persyaratan Arsitektur Strategis} \\
	Prinsip utama yang memandu arsitektur ini meliputi aksesibilitas, skalabilitas, keamanan, dan adaptabilitas. Arsitektur ini harus mendukung akses yang lancar ke sumber belajar, menyediakan fleksibilitas untuk skala sesuai dengan pendaftaran, memastikan keamanan dan privasi data, serta memungkinkan adaptasi terhadap perkembangan teknologi dan kebutuhan pedagogis di masa mendatang.
	
	\item \textbf{Persyaratan Kepatuhan} \\
	Arsitektur harus mematuhi kebijakan universitas, standar pendidikan, peraturan privasi data (seperti GDPR), dan pedoman aksesibilitas (misalnya, WCAG). Semua platform dan alat yang digunakan harus memenuhi persyaratan ini untuk melindungi data mahasiswa dan fakultas serta memastikan akses yang setara bagi semua pengguna.
	
	\item \textbf{Proses dan Peran dalam Pengembangan dan Manajemen Arsitektur} \\
	Proses pengembangan mencakup tahap perencanaan, desain, implementasi, dan evaluasi, dengan peran yang ditugaskan kepada manajer proyek, perancang pembelajaran, staf TI, perwakilan fakultas, dan konsultan eksternal. Setiap peran bertanggung jawab atas tugas tertentu, mulai dari pengumpulan persyaratan awal hingga pemeliharaan dan dukungan sistem yang berkelanjutan.
	
	\item \textbf{Ukuran Target Arsitektur} \\
	Arsitektur ini akan mendukung basis pengguna sekitar 10.000 mahasiswa dan 1.000 anggota fakultas dan staf di beberapa kampus dan lingkungan pembelajaran daring. Ini akan dirancang untuk menangani beban lalu lintas puncak, termasuk akses simultan selama periode ujian dan kelas virtual berskala besar.
	
	\item \textbf{Tahapan Hasil yang Terdefinisi} \\
	Tahapan hasil utama mencakup:
	\begin{itemize}
		\item Cetak biru arsitektur awal dan persetujuan.
		\item Pengembangan dan pengujian modul pembelajaran inti dan platform.
		\item Tahap pilot untuk kursus terpilih.
		\item Implementasi penuh di seluruh departemen.
		\item Evaluasi dan pembaruan berkala berdasarkan umpan balik pengguna.
	\end{itemize}
	
	\item \textbf{Rencana Kerja Bersama yang Diprioritaskan} \\
	Rencana kerja bersama memprioritaskan pembangunan sistem dasar (misalnya, sistem manajemen pembelajaran), integrasi alat hybrid, dan pelatihan fakultas dalam menggunakan arsitektur baru. Setelah itu, fokus akan bergeser pada dukungan metode pembelajaran asinkron dan sinkron, pengoptimalan pengalaman pengguna, serta pengumpulan umpan balik untuk perbaikan berkelanjutan.
	
	\item \textbf{Jangka Waktu Implementasi} \\
	Arsitektur akan diimplementasikan selama \textbf{dua tahun} akademik, dengan tahun pertama didedikasikan untuk perencanaan, desain, dan uji coba, sedangkan tahun kedua berfokus pada implementasi penuh dan skala besar. Pencapaian meliputi tinjauan kemajuan triwulanan dan evaluasi komprehensif setelah fase pilot.
	
	\item \textbf{Hasil dari Arsitektur dan Metrik Bisnis} \\
	Metrik keberhasilan mencakup:
	\begin{itemize}
		\item Tingkat adopsi pengguna di antara mahasiswa dan fakultas: target \textbf{80\%} tingkat adopsi dalam tahun pertama.
		\item Waktu aktif (\textit{uptime}) sistem dan waktu respons selama periode puncak: pertahankan \textbf{99.9\% waktu aktif} dan waktu respons di bawah \textbf{2 detik} selama waktu lalu lintas tinggi (misalnya, ujian).
		\item Survei kepuasan pengguna yang berfokus pada keterlibatan dan kemudahan penggunaan: capai skor kepuasan rata-rata \textbf{4.0 dari 5} dalam survei yang dilakukan setiap semester.
		\item Metrik kinerja akademik yang membandingkan hasil pembelajaran hybrid dengan metode tradisional: tujuan peningkatan \textbf{5-10\%} dalam keterlibatan mahasiswa dan tingkat penyelesaian kursus.
		\item Kepatuhan terhadap standar aksesibilitas dan privasi data: pertahankan \textbf{100\% kepatuhan} dengan pedoman GDPR dan WCAG melalui audit rutin.
	\end{itemize}
	
\end{enumerate}


\section{Contoh Kontrak Arsitektur Bisnis untuk Pembelajaran dan Pengajaran Hybrid}
\label{sec:contoh_kontrak_arsitektur_bisnis}

\begin{enumerate}
	\item \textbf{Pendahuluan dan Latar Belakang} \\
	Kontrak Arsitektur Bisnis ini dibuat untuk mendukung tujuan universitas dalam menyediakan pembelajaran dan pengajaran hybrid yang fleksibel dan berkualitas tinggi. Kontrak ini mengatur persyaratan dan tanggung jawab dari semua pihak yang terlibat dalam implementasi arsitektur yang memungkinkan integrasi antara metode pembelajaran daring dan tatap muka, yang dirancang untuk meningkatkan pengalaman belajar dan memenuhi kebutuhan beragam mahasiswa.
	
	\item \textbf{Sifat dari Kesepakatan} \\
	Kesepakatan ini mencakup komitmen dari fakultas, staf administrasi, departemen teknologi informasi, dan mitra teknologi eksternal untuk berkolaborasi dalam pengembangan, penerapan, dan pemeliharaan arsitektur hybrid yang mendukung kegiatan akademik di universitas. Kontrak ini juga bertujuan untuk mengoptimalkan proses pembelajaran dan pengajaran dengan teknologi yang mendukung fleksibilitas, aksesibilitas, dan efektivitas pendidikan.
	
	\item \textbf{Ruang Lingkup} \\
	Ruang lingkup kontrak ini mencakup seluruh sistem dan proses pembelajaran hybrid, termasuk integrasi platform pembelajaran digital, kelas virtual, sistem manajemen konten, dan alat-alat penilaian yang digunakan dalam proses pembelajaran daring dan tatap muka. Kontrak ini juga mencakup dukungan untuk perangkat keras dan infrastruktur yang diperlukan untuk mendukung pembelajaran hybrid di kampus.
	
	\item \textbf{Persyaratan Strategis} \\
	Persyaratan strategis arsitektur mencakup aksesibilitas, skalabilitas, keamanan, dan interoperabilitas untuk mendukung pertumbuhan dan adaptasi terhadap perubahan kebutuhan pembelajaran. Arsitektur harus memungkinkan akses yang mudah dan aman bagi seluruh mahasiswa dan staf, memungkinkan peningkatan kapasitas sesuai kebutuhan, serta memastikan arsitektur yang kompatibel dengan berbagai teknologi dan platform yang mendukung tujuan pembelajaran hybrid jangka panjang.
	
	\item \textbf{Persyaratan Kepatuhan} \\
	Arsitektur ini harus mematuhi kebijakan internal universitas, standar pendidikan nasional, peraturan privasi data seperti GDPR, serta pedoman aksesibilitas seperti WCAG. Kepatuhan ini bertujuan untuk melindungi data pribadi mahasiswa dan staf serta memastikan akses yang adil dan setara bagi seluruh pengguna arsitektur.
	
	\item \textbf{Pengguna Arsitektur} \\
	Pengguna utama dari arsitektur ini meliputi mahasiswa, dosen, dan staf administrasi yang terlibat dalam proses pembelajaran dan pengajaran. Mahasiswa menggunakan arsitektur untuk mengakses materi pembelajaran, mengikuti kelas daring, dan berpartisipasi dalam kegiatan interaktif. Dosen menggunakan arsitektur untuk menyampaikan materi, mengelola kelas, dan berinteraksi dengan mahasiswa. Staf administrasi bertanggung jawab dalam mengelola data pengguna dan mendukung keberlangsungan operasional arsitektur.
	
	\item \textbf{Jangka Waktu Pelaksanaan} \\
	Implementasi arsitektur akan dilakukan selama \textbf{dua tahun} akademik, dengan tahap awal berfokus pada perencanaan dan desain, serta tahap kedua pada pengujian dan penerapan penuh di seluruh fakultas. Kontrak ini mencakup evaluasi berkala setiap kuartal untuk memastikan kemajuan yang sesuai dengan jadwal dan tujuan yang telah ditetapkan.
	
	\item \textbf{Metrik Bisnis Arsitektur} \\
	Metrik keberhasilan dari arsitektur ini meliputi:
	\begin{itemize}
		\item Tingkat adopsi oleh pengguna, yaitu mahasiswa dan dosen, dengan target minimal \textbf{80\%} pada akhir tahun pertama.
		\item Ketersediaan sistem dengan waktu aktif \textbf{99.9\%} dan waktu respons kurang dari \textbf{2 detik} selama jam sibuk seperti periode ujian.
		\item Kepuasan pengguna, dengan target skor rata-rata \textbf{4.0 dari 5} pada survei setiap semester mengenai pengalaman belajar dan mengajar.
		\item Peningkatan hasil akademik, dengan sasaran peningkatan \textbf{5-10\%} pada keterlibatan mahasiswa dan tingkat kelulusan kursus yang menggunakan sistem hybrid.
		\item Kepatuhan terhadap standar aksesibilitas dan privasi data dengan audit rutin untuk memastikan \textbf{100\% kepatuhan} dengan GDPR dan WCAG.
	\end{itemize}
	
	\item \textbf{Arsitektur Layanan (termasuk SLA)} \\
	Arsitektur layanan dalam konteks ini mencakup perjanjian tingkat layanan (SLA) yang menjamin bahwa platform pembelajaran hybrid tetap beroperasi dalam standar yang disepakati, termasuk waktu respons dukungan teknis dan pemulihan sistem. SLA menetapkan komitmen untuk pemulihan cepat dalam kasus gangguan layanan dan menjamin tingkat pelayanan sesuai kebutuhan pendidikan.
	
\end{enumerate}


\section{Contoh Penilaian Kepatuhan untuk Arsitektur Pembelajaran dan Pengajaran Hybrid }
\label{sec:contoh_penilaian_kepatuhan}
\begin{enumerate}
	
	 \item \textbf{Gambaran Umum Kemajuan dan Status Proyek} \\
	Laporan ini memberikan gambaran umum tentang kemajuan proyek implementasi arsitektur hybrid di universitas. Penilaian ini mencakup status setiap tahap proyek, termasuk:
	\begin{itemize}
		\item \textbf{Tahap Perencanaan:} Penyelesaian pengumpulan persyaratan pengguna, termasuk kebutuhan khusus dosen dan mahasiswa untuk pembelajaran daring dan tatap muka.
		\item \textbf{Tahap Desain:} Prototipe sistem manajemen konten (CMS) telah diselesaikan dengan integrasi kelas virtual dan alat penilaian.
		\item \textbf{Tahap Pengembangan:} Platform hybrid telah dikonfigurasi dengan basis data pengguna yang mencakup data mahasiswa, dosen, dan staf.
		\item \textbf{Tahap Implementasi:} Platform diujicobakan pada dua fakultas untuk memastikan stabilitas dan mengumpulkan umpan balik pengguna sebelum peluncuran penuh.
	\end{itemize}
	Pemantauan kemajuan ini memastikan bahwa proyek berjalan sesuai dengan jadwal dan mencapai target yang ditetapkan, dengan pembaruan mingguan untuk tim proyek dan pihak universitas.
	
	\item \textbf{Gambaran Umum Arsitektur/Desain Proyek} \\
	Bagian ini menjelaskan desain arsitektur pembelajaran hybrid yang terdiri dari:
	\begin{itemize}
		\item \textbf{Platform Pembelajaran:} Platform utama menggunakan sistem manajemen pembelajaran (LMS) yang diintegrasikan dengan Google Classroom dan Moodle untuk mendukung pembelajaran asinkron dan sinkron.
		\item \textbf{Kelas Virtual:} Menggunakan Zoom dan Microsoft Teams untuk mengakomodasi pembelajaran daring yang sinkron, dengan integrasi kalender untuk memudahkan pengelolaan jadwal kelas.
		\item \textbf{Sistem Manajemen Konten (CMS):} CMS mendukung penyimpanan dan akses ke materi perkuliahan, catatan kelas, video pembelajaran, serta tugas yang diunggah oleh mahasiswa dan dosen.
		\item \textbf{Alat Penilaian dan Pengawasan:} Mencakup sistem ujian daring yang diintegrasikan dengan alat proctoring otomatis untuk menjaga integritas akademik selama ujian jarak jauh.
		\item \textbf{Dashboard Analitik:} Sistem analitik real-time untuk pelacakan kinerja mahasiswa, seperti tingkat kehadiran, partisipasi kelas, dan perkembangan akademik, guna memberikan umpan balik langsung kepada dosen.
	\end{itemize}
	Komponen teknologi ini dirancang untuk memungkinkan integrasi penuh antara metode pembelajaran daring dan tatap muka, dengan kemampuan beradaptasi terhadap perkembangan kebutuhan pembelajaran hybrid di universitas.
	
	
	\item \textbf{Daftar Periksa Arsitektur yang Telah Selesai} \\
	Daftar periksa ini memastikan bahwa setiap elemen arsitektur telah memenuhi persyaratan yang ditetapkan dan diuji untuk kepatuhan terhadap standar. Daftar periksa meliputi checkpoint berikut:
	
	\begin{enumerate}
		\item \textbf{Daftar Periksa Perangkat Keras dan Sistem Operasi} 
		\begin{itemize}
			\item Ketersediaan perangkat keras yang cukup untuk mendukung arsitektur hybrid.
			\item Kompatibilitas perangkat keras dengan sistem operasi yang digunakan.
			\item Ketersediaan cadangan (backup) perangkat keras untuk pemulihan cepat.
			\item Pengaturan jaringan yang mendukung koneksi stabil, terutama selama kegiatan daring.
			\item Pemantauan kinerja sistem secara real-time untuk mencegah downtime.
		\end{itemize}
		
		\item \textbf{Daftar Periksa Layanan Perangkat Lunak dan Middleware} 
		\begin{itemize}
			\item Kompatibilitas platform pembelajaran dengan perangkat pengguna (PC, tablet, smartphone).
			\item Integrasi middleware yang memungkinkan komunikasi antar sistem (misalnya, sistem manajemen pembelajaran dengan platform kelas virtual).
			\item Ketersediaan API untuk menghubungkan perangkat lunak pihak ketiga yang digunakan dalam pembelajaran.
			\item Sistem keamanan pada middleware untuk mencegah akses tidak sah.
		\end{itemize}
		
		\item \textbf{Daftar Periksa Aplikasi} 
		\begin{itemize}
			\item Ketersediaan dan fungsionalitas aplikasi pembelajaran yang diperlukan.
			\item Kompatibilitas aplikasi dengan berbagai perangkat dan sistem operasi.
			\item Pengaturan pembaruan otomatis untuk menjaga aplikasi tetap up-to-date.
			\item Kemampuan aplikasi untuk mendukung pembelajaran sinkron dan asinkron.
		\end{itemize}
		
		\item \textbf{Daftar Periksa Manajemen Informasi} 
		\begin{itemize}
			\item Penyimpanan data pengguna yang aman sesuai dengan standar privasi.
			\item Kontrol akses untuk memastikan hanya pihak yang berwenang yang dapat mengakses data sensitif.
			\item Ketersediaan data cadangan dan strategi pemulihan jika terjadi kehilangan data.
			\item Kemampuan untuk mengelola data akademik, seperti nilai dan kehadiran, secara terintegrasi.
		\end{itemize}
		
		\item \textbf{Daftar Periksa Keamanan} 
		\begin{itemize}
			\item Implementasi enkripsi untuk data dalam perjalanan dan data yang disimpan.
			\item Sistem otentikasi multi-faktor (MFA) bagi pengguna.
			\item Pemindaian kerentanan secara berkala untuk mendeteksi risiko keamanan.
			\item Prosedur respons insiden jika terjadi pelanggaran keamanan.
			\item Kepatuhan dengan standar keamanan seperti GDPR dan pedoman universitas.
		\end{itemize}
		
		\item \textbf{Daftar Periksa Manajemen Sistem} 
		\begin{itemize}
			\item Ketersediaan alat pemantauan sistem untuk melacak kinerja platform.
			\item Kebijakan pemeliharaan preventif untuk menjaga kestabilan sistem.
			\item Sistem notifikasi untuk memberi peringatan saat terjadi anomali.
			\item Prosedur pemulihan bencana dan pemulihan data.
		\end{itemize}
		
		\item \textbf{Daftar Periksa Rekayasa Sistem} 
		\begin{itemize}
			\item Pengujian kompatibilitas sistem dengan berbagai perangkat.
			\item Pengujian beban (load testing) untuk memastikan kinerja optimal pada periode sibuk.
			\item Ketersediaan fitur pencadangan otomatis untuk konfigurasi sistem.
			\item Dokumentasi lengkap dari setiap komponen sistem dan prosedur pemeliharaan.
		\end{itemize}
		
		\item \textbf{Daftar Periksa Metode dan Alat} 
		\begin{itemize}
			\item Evaluasi efektivitas metode pengajaran yang digunakan dalam platform.
			\item Kesesuaian alat yang digunakan dengan tujuan pembelajaran hybrid.
			\item Kemampuan alat untuk mendukung kolaborasi dan interaksi antar pengguna.
			\item Prosedur untuk mengevaluasi umpan balik dari pengguna dan melakukan penyesuaian metode atau alat.
		\end{itemize}
	\end{enumerate}
	
\end{enumerate}


\section{Ringkasan}
\begin{enumerate}
	\item Tata Kelola Implementasi mendefinisikan batasan arsitektur dan mendapatkan persetujuan pada Kontrak Arsitektur.
	\item Serahkan kontrak dan semua dokumentasi kepada tim implementasi.
	\item Kelola arsitektur selama implementasi dengan tinjauan kepatuhan dan pemantauan risiko.
	\item Lakukan tinjauan pasca-implementasi.
\end{enumerate}

\section{Aktivitas Kelas dan Tugas}

Buatlah dokumen Kontrak Desain dan Pengembangan Arsitektur (Subbab \ref{sec:contoh_kontrak_desain_dan_pengembangan_arsitektur}), Kontrak Arsitektur Bisnis (Subbab \ref{sec:contoh_kontrak_arsitektur_bisnis}), dan Penilaian Kepatuhan (Subbab \ref{sec:contoh_penilaian_kepatuhan}) untuk memastikan implementasi menghasilkan Kapabilitas yang dikehendaki.